\author{João Gonçalves}
\newcommand{\authorr}{Teresa Nogueira}
\newcommand{\studentID}{99995}
\newcommand{\studentIDD}{100029}
\newcommand{\supervisorone}{}
\newcommand{\supervisortwo}{}
\newcommand{\department}{Engenharia Eletrotécnica e de Computadores}
\newcommand{\exam}{Fundamentos de Energia Elétrica}

\title{%
Fundamentos de Energia Elétrica\\
\large (Alguns tópicos \href{https://github.com/Kons-5}{\raisebox{0 em}{\large \faGithub}})}
\date{Setembro 2023}

\documentclass[a4paper, 10pt]{article}
\usepackage{packages}

%----------------------------------TITLE PAGE -----------------------------------
\makeatletter
\def\maketitle{
  \begin{center}\leavevmode
        \normalfont
        \includegraphics[width=0.5\columnwidth]{img/title-page/IST.pdf}
        \vskip 0.05cm   
        \textsc{\large \department}\\
        \vskip 0.5cm
        \rule{0.95\linewidth}{0.2 mm} %\\
        {\large \exam}\\[0.5 cm]
        {\huge \bfseries \@title \par} 
        \vspace{1em}
        \begin{tikzpicture}
            \node[inner sep=6.66pt] (image) at (0,0) {
                \includegraphics[scale=0.4]{img/title-page/000.jpg}
            };
            \draw[dashed] (image.south west) rectangle (image.north east);
        \end{tikzpicture}
        \vspace{-0.5em}
        \captionof*{figure}{\color{gray}Imagem: \textit{Máquina síncrona}}
        \vspace{0.5cm}
        \rule{0.95\linewidth}{0.2 mm} \\[0.75 cm]
        %\fontsize{9pt}{11pt}\selectfont
        \begin{minipage}[t]{\textwidth}
            \begin{flushleft} \large
                \emph{Autores:}\\
                \normalsize \textbf{\@author} : \studentID \\
                \fontsize{9pt}{11pt}\selectfont $\hookrightarrow$ jrazevedogoncalves@tecnico.ulisboa.pt \\
                \normalsize \textbf{\authorr} : \studentIDD \\
                \fontsize{9pt}{11pt}\selectfont $\hookrightarrow$ maria.teresa.ramos.nogueira@tecnico.ulisboa.pt
            \end{flushleft}
	   \end{minipage}%
    \vfill
	{\Large \@date\par}
   \end{center}
   %\vfill
   %\null
   \cleardoublepage
  }
\makeatother
%-------------------------------- ENDTITLE PAGE ----------------------------------

%---------------------------------- HEADER ------------------------------------
%\fancyhf{}
\renewcommand{\headrulewidth}{1pt}% Header rule width
\renewcommand{\footrulewidth}{0pt}% No footer rule
\setlength\headheight{26pt} 
\fancyhead[L]{\raisebox{0.1\height}[0pt][0pt]{\textit{Fundamentos de Energia Elétrica}}}
\fancyhead[R]{\raisebox{0.1\height}[0pt][0pt]{2023/2024}}
%-------------------------------- END HEADER ----------------------------------

\setcounter{tocdepth}{4}
\setcounter{secnumdepth}{4}
% \setcounter{secnumdepth}{-2}

\renewcommand{\figurename}{Fig.}
\renewcommand{\tablename}{Tab.}
\renewcommand{\contentsname}{Índice}
\settocbibname{\raisebox{0em}{Referências}}
\setlength{\bibsep}{0.15em}%reduzir espaço entre refs.

\pgfplotsset{compat=1.18}

\begin{document}
    \sloppy
    %% title page
    \pagenumbering{gobble}
    \maketitle
    
    %% toc
    \setcounter{tocdepth}{3}
    \tableofcontents
    \setcounter{tocdepth}{4}
    
    %% body
    \newpage
    \pagestyle{fancy}
    \pagenumbering{arabic}
    
    \setlength{\abovedisplayskip}{4pt}
    \setlength{\belowdisplayskip}{4pt}

    \clearpage
    \section{Conceitos Básicos}%
       %//==============================--@--==============================//%
\subsection{Energia e Potência. Diagrama de carga}
\label{subsec:energy-and-power}

A relação básica entre energia e potência exprime-se por:
$$
    P = \frac{dW}{dt} \qquad\rightleftarrows\qquad W = \int P\, dt
$$
onde $W$ denota a energia (expressa em joule, J) e $P$ a potência (expressa em watt, W).

\vspace{1em}
\noindent A carga de um Sistema de Energia Elétrica (SEE), varia significativamente ao longo do dia, acompanhando a atividade humana. Para um diagrama de carga define-se
$$
    h_d = \frac{W}{P_\text{max}}\, [\text{unidades de tempo}] \qquad\quad\qquad f_d = \frac{P_\text{med}}{P_\text{max}}\, [\%], \mkern9mu \text{em que } P_\text{med} = \frac{1}{T} \int_T P\, dt
$$
a \textit{utilização diária da ponta}, $h_d$, como a relação entre a energia e a potência máxima; e também o \textit{fator de carga diário}, $f_d$, como relação entre a potência média e a potência máxima.

\begin{figure}[H]
    \centering
    \scalebox{0.9}{%
        \begin{tikzpicture}
            \begin{axis}[
                xlabel={Horas}, xmin=0, xmax=24,
                xtick={0,2,4,6,8,10,12,14,16,18,20,22,24},
                ylabel={Carga $[$MW$]$}, ymin=0, ymax=50,
                ytick={0,5,10,15,20,25,30,35,40,45,50},
                grid=both,
                minor tick num=1,
                width=10cm,
                height=6.5cm
            ]
            
            \addplot [thick, line width=1.5pt, fill=gray!20, fill opacity=0.45] table [col sep=comma, x=Time, y=Load] {src/1/data.csv} \closedcycle;

            \addplot [mark=*, mark size=2pt, only marks] coordinates {(13, 41)};
            
            \node[above right, fill=white, fill opacity=0.6, text opacity=1, font=\large] at (13,41) {$P_\text{max}$};

            \node[above right, font=\large] at (10,10) {$W = \displaystyle \int P\, dt$};

            \draw [dashed, line width=1.25pt] (axis cs:0, 30.12) -- (axis cs:24, 30.12);

            \node[above, fill=white, fill opacity=0.6, text opacity=1, font=\large] at (4,30) {$P_\text{med}$};
            \end{axis}
        \end{tikzpicture}
    }
    \caption{Exemplo de diagrama de carga (diário)}
    \label{fig:diagrama-carga}
\end{figure}
%//==============================--@--==============================//%
\subsection{Potência em Sistemas de Energia Elétrica}
\label{subsec:power-SEE}

Os SEE atualmente funcionam quase na sua totalidade em corrente alternada --- com uma frequência de $50$ Hz na Europa e de $60$ Hz nos EUA e no Brasil ---, existindo casos especiais em que se utiliza a corrente contínua.\cite{paiva2005}

\renewcommand{\thefootnote}{\fnsymbol{footnote}}
\footnotetext[4]{%
    Em alguns países, como por exemplo no Japão, coexistem as duas frequências.
}
\renewcommand{\thefootnote}{\arabic{footnote}}

%//==============================--@--==============================//%
\subsubsection{Potência Instantânea, Potência Ativa e Reativa, Potência Complexa e Aparente}

Considerando um sistema monofásico constituído por um gerador que dá origem a uma tensão $v$ aos terminais de uma carga representada por uma impedância constante $Z$, define-se a \textit{potência instantânea} transferida do gerador para a carga como
$$
    \begin{aligned}
        p = v i &= \overbrace{\sqrt{2}V \cos(\omega t + \alpha_v)}^{\text{tensão }v} \cdot \overbrace{\sqrt{2}I \cos(\omega t + \alpha_i)}^{\text{corrente }i}, \qquad \forall (\alpha_v, \alpha_i) \in [0,2\pi] \times [0,2\pi]: \phi \in [-\pi/2,\pi/2] \\
        &= VI \cos(\phi) - VI \cos(2\omega t - \phi)
    \end{aligned}
$$
em que $\phi = \alpha_v - \alpha_i$ (desfasagem da tensão face à corrente). Massajando a expressão conclui-se
$$
    p = \underbrace{VI \cos(\phi) [1 - \cos(2 \omega t)]}_{p_1} - \underbrace{VI \sin(\phi) \sin(2\omega t)}_{p_2}
$$
Repare-se que a componente $p_1$ tem valor médio constante e $p_2$ tem valor médio nulo. A \textit{potência ativa} é o valor médio da potência instantânea, e corresponde portanto à potência efetivamente transferida:
$$
    P = VI \cos(\phi) \quad [\text{W}]
$$
A \textit{potência reativa} é o valor máximo da componente da potência que oscila entre o gerador e a carga, cujo valor médio é nulo, resultante da variação da energia magnética/elétrica armazenada nos elementos da carga:
$$
    Q = VI \sin(\phi) \quad [\text{VAr}]
$$
\noindent \textbf{Nota}: A grandeza $\cos(\phi)$ designa-se por \textit{fator de potência}, $fp \delequal P/S = \cos(\phi)$.

\clearpage
\vspace*{-2.25em}
\begin{centering}
    \begin{minipage}[b]{0.5\linewidth}
        \begin{figure}[H]
            \centering
            \scalebox{0.925}{%
            \begin{circuitikz}[scale=1.25]
                % Voltage source
                \draw (0,0) to[sV] (0,2.75) -- (4,2.75);
                % Arbitrary impedance (Z)
                \draw (4,2.75) node[circ] {} to[generic, l=$Z$, v=$v$] (4,0) node[circ] {}  -- (0,0);
                % Draw the current's arrow
                \draw[-latex, black, line width=0.9pt] (1.25,2.9) -- (2.75,2.9) node[midway,above] {$i$};
                % Label
                \draw (-0.35,0.7) node[xshift=0mm, yshift=2mm] {$(\omega)$};
            \end{circuitikz}}
            \caption{Sistema monofásico em corrente alternada}
        \end{figure}
    \end{minipage}%
    \begin{minipage}[b]{0.5\linewidth}
        \begin{figure}[H]
            \centering
            \scalebox{0.925}{%
            \begin{tikzpicture}[scale=1.25, >=latex]
                % Draw real and imaginary axes
                \draw[->] (-0.25,0) -- (2.5,0) node[right] {Re};
                \draw[->] (0,-0.25) -- (0,2.5) node[above] {Im};
                
                \draw [line width=1.25pt] (0,0) -- (2,0) -- (2,2) -- cycle;
        
                \draw [->, line width=1.25pt] (0,0) -- (2.025,2.025);
                
                \node[below] at (1,0) {$P$};
                \node[right] at (2,1) {$Q$};
                \node[above left] at (1,1) {$\mathbf{S}$};
                \node[right] at (2.2,2.4) {$|\mathbf{S}| = \sqrt{P^2 + Q^2} = VI$};
                \node[left] at (-1.15,2) {};

                \draw (0.75,0) arc (0:45:0.75); 
                \node[right] at (0.715,0.35) {$\phi$};
            \end{tikzpicture}}
            \caption{Representação das potências}
        \end{figure}
    \end{minipage}
\end{centering}

\noindent A \textit{potência complexa} é definida pelo produto do fasor tensão pelo conjugado do fasor corrente:
$$
    \mathbf{S} = \mathbf{V} \mathbf{I}^* = P + jQ \quad\text{em que}\quad
    \left\{\begin{aligned}
        \mathbf{V} &= V e^{j\alpha_v} \\
        \mathbf{I} &= I e^{j\alpha_i}
    \end{aligned}\right.
$$
A \textit{potência aparente} é definida como o módulo da potência complexa, i.e.,
$$
    S = |\mathbf{S}| = \sqrt{P^2 + Q^2} = VI \quad [\text{VA}]
$$
Além disto, a potência aparente representa a potência média ao longo do tempo para circuitos que se comportam como puramente resistivos ($\phi=0$).
%//==============================--@--==============================//%
\subsection{Sistema Elétrico Trifásico}

\noindent \textbf{Motivação}: A energia elétrica é produzida, transportada e distribuida em \textit{sistemas elétricos trifásicos}. A natureza pulsante da potência em sistemas monofásicos produz efeitos indesejaveis na operação dos sistemas elétricos. Nos motores elétricos, esta potência traduz-se num binário pulsante, por exemplo. A corrente trifásica elimina estas pulsações de potência e binário.

\begin{figure}[H]
    \centering
    \begin{circuitikz}[scale=0.625, transform shape]
        \draw (3.75,13.75) to[sinusoidal voltage source, sources/symbol/rotate=auto] (6.25,13.75);
        \draw (3.75,12.5) to[sinusoidal voltage source, sources/symbol/rotate=auto] (6.25,12.5);
        \draw (3.75,11.25) to[sinusoidal voltage source, sources/symbol/rotate=auto] (6.25,11.25);
        \draw [](3.75,13.75) to[short] (3.75,8.75);
        \draw [](6.25,13.75) to[short] (12.5,13.75);
        \draw [](6.25,12.5) to[short] (12.5,12.5);
        \draw [](6.25,11.25) to[short] (12.5,11.25);
        \draw (3.75,11.25) to[short, -*] (3.75,11.25);
        \draw (3.75,12.5) to[short, -*] (3.75,12.5);
        \draw (12.5,13.75) to[R, l=$Z$] (16.25,13.75);
        \draw (12.5,12.5) to[R, l=$Z$] (16.25,12.5);
        \draw (12.5,11.25) to[R, l=$Z$] (16.25,11.25);
        \draw [](16.25,13.75) to[short] (16.25,8.75);
        \draw (16.25,11.25) to[short, -*] (16.25,11.25);
        \draw (16.25,12.5) to[short, -*] (16.25,12.5);
        \draw [dashed] (2.5,8.75) -- (17.5,8.75);
        \draw [, dashed] (12.25,15) rectangle  (16.5,10);
        \draw [, dashed] (6.75,10) rectangle  (3.25,15);
        \draw (6.75,11.25) to[short, -*] (6.75,11.25);
        \draw (6.75,12.5) to[short, -*] (6.75,12.5);
        \draw (6.75,13.75) to[short, -*] (6.75,13.75);
        \draw (12.25,13.75) to[short, -*] (12.25,13.75);
        \draw (12.25,12.5) to[short, -*] (12.25,12.5);
        \draw (12.25,11.25) to[short, -*] (12.25,11.25);

        % Add overbrace and label
        \draw [decorate,decoration={brace,amplitude=5pt}] (3.25,15.15) -- (6.75,15.15) node[midway,above=12pt,font=\large] {Gerador trifásico};
        \draw [decorate,decoration={brace,amplitude=5pt}] (6.90,15.15) -- (12.10,15.15) node[midway,above=8pt,font=\large] {\parbox{3.85cm}{Linha de transmissão \\ trifásica}};
        \draw [decorate,decoration={brace,amplitude=5pt}] (12.25,15.15) -- (16.5,15.15) node[midway,above=10pt,font=\large] {Carga trifásica};

        % Label
        \draw (3.75,10.25) node[xshift=2.5cm, yshift=1.5mm] {$(\omega)$};
    \end{circuitikz}
    \caption{Sistema trifásico simétrico}
\end{figure}

%//==============================--@--==============================//%
\subsubsection{Tensão e Corrente}

\noindent Um gerador trifásico com os 3 enrolamentos em estrela produz 3 forças eletromotizes com frequência angular $w =2\pi f$, desfasados $\pm 2\pi/3 (= \pm 120^\degree)$. A fase de referência, possui argumento nulo.


\vspace{-0.5em}
\begin{centering}
    \begin{minipage}[b]{0.5\linewidth}
        \begin{figure}[H]
        \centering
            \scalebox{0.925}{%
            \begin{tikzpicture}[scale=0.825]
                \begin{axis}[
                domain=0:1, 
                samples=400, 
                axis lines=middle, 
                axis line style={-}, 
                xtick=\empty, 
                ytick=\empty, 
                xticklabel=\empty, 
                yticklabel=\empty,
                ymax = 1.15,
                xmax = 1.1,
                ]
                
                \addplot[black,thick] {sin(deg(2*pi*x))}; 
                \addplot[black,thick] {sin(deg(2*pi*x - 2*pi/3))}; 
                \addplot[black,thick] {sin(deg(2*pi*x + 2*pi/3))}; 
                
                \draw[dashed] (axis cs:1,1) -- (axis cs:1,-1); 
                \node at (axis cs:1.05,-0.1) {$2\pi$};
                \node at (axis cs:0.25,1.1) {$e_a$};
                \node at (axis cs:0.58,1.1) {$e_b$};
                \node at (axis cs:0.91,1.1) {$e_b$};
             \end{axis}
        \end{tikzpicture}}
        \caption{Variação no tempo das f.e.m}
    \end{figure}
    \end{minipage}%
    \begin{minipage}[b]{0.5\linewidth}
        \begin{figure}[H]
        \centering
            \scalebox{0.925}{%
            \begin{tikzpicture}[scale=1.25,>=stealth]
                 \coordinate (O) at (0,0);
        
                  \draw[->,thick] (O) -- ++(240:2cm) node[right] {$E_b$}; 
                  \draw[->,thick] (O) -- ++(360:2cm) node[below] {$E_a$}; 
                  \draw[->,thick] (O) -- ++(480:2cm) node[left] {$E_c$};
                
                  \draw[->, thick] (240:0.3cm) arc (240:360:0.3cm) node[midway, below] {$\frac{2\pi}{3}$};   
                  \draw[->, thick] (360:0.3cm) arc (360:480:0.3cm) node[midway, above] {$\frac{2\pi}{3}$};
                  \draw[->,thick] (480:0.3cm) arc (480:600:0.3cm) node[midway, left] {$\frac{2\pi}{3}$};
            \end{tikzpicture}}
        \caption{Diagrama de fasores}
        \end{figure}
    \end{minipage}
\end{centering}

%% next page
\clearpage
\begin{minipage}[c]{0.4\textwidth}
    \begin{figure}[H]
        \centering
        \begin{tikzpicture}[scale=1.5,>=stealth]
            \coordinate (O) at (0,0);
            
            \draw[->,thick] (O) -- ++(240:2cm) node[below] {$V_b$}; 
            \draw[->,thick] (O) -- ++(360:2cm) node[right] {$V_a$}; 
            \draw[->,thick] (O) -- ++(480:2cm) node[above] {$V_c$};
    
            \draw[->,thick] (O) -- ++(210:0.75cm) node[above] {$I_b$};
            \draw[->,thick] (O) -- ++(330:0.75cm) node[below] {$I_a$};
            \draw[->,thick] (O) -- ++(450:0.75cm) node[right] {$I_c$};
    
            \draw (360:0.3cm) arc (360:330:0.3cm) node[midway, right, below=, font=\small] {$\phi$};
            \draw (240:0.3cm) arc (240:210:0.3cm) node[midway, left, font=\small] {$\phi$};
            \draw (480:0.3cm) arc (480:450:0.3cm) node[midway, above, font=\small] {$\phi$};
    
            % Calculate the coordinates of the tips of V_a, V_b, and V_c
            \coordinate (tipA) at ($(O)+(360:2cm)$);
            \coordinate (tipB) at ($(O)+(240:2cm)$);
            \coordinate (tipC) at ($(O)+(480:2cm)$);
            
            % Draw vectors V_ab, V_ca, and V_bc
            \draw[->,thick,dashed] (tipB) -- (tipA) node[midway, below=0.75mm] {$V_{ab}$};
            \draw[->,thick,dashed] (tipA) -- (tipC) node[midway, above] {$V_{ca}$};
            \draw[->,thick,dashed] (tipC) -- (tipB) node[midway, left] {$V_{bc}$};
    
            % Draw the angle 
            \draw (O) ++(360:1.45cm) arc (180:210:0.55cm);
            \node at ($(O)+(351:1.3cm)$) {$\frac{\pi}{6}$};
        \end{tikzpicture}
        \caption{Fasores de tensão (simples e composta) e fasores de corrente num sistema trifásico simétrico.}
    \end{figure}
\end{minipage} \hfill
\begin{minipage}[c]{0.525\textwidth} \small
    \noindent As \textit{tensões simples} ou \textit{fase-neutro} são então: 
    $$ 
    \begin{aligned} 
        v_a &= \sqrt{2} V \sin(\omega t) \\ 
        v_b &= \sqrt{2} V \sin(\omega t - 2\pi/3) \\ 
        v_c &= \sqrt{2} V \sin(\omega t + 2\pi/3) 
    \end{aligned} 
    \qquad\rightleftarrows\qquad 
    \begin{aligned} 
        \mathbf{V}_{a} &= V e^{j0} \\ 
        \mathbf{V}_{b} &= V e^{-j2\pi/3} \\ 
        \mathbf{V}_{c} &= V e^{j2\pi/3} 
    \end{aligned} 
    $$ 
    
    Num sistema trifásico, define-se o valor das \textit{tensões fase-fase} (ou \textit{tensões entre fases} ou \textit{tensões compostas}): 
    $$ 
    \begin{aligned} 
        \mathbf{V}_{ab} &= \mathbf{V}_a - \mathbf{V}_b \\ 
        \mathbf{V}_{bc} &= \mathbf{V}_b - \mathbf{V}_c \\ 
        \mathbf{V}_{ca} &= \mathbf{V}_c - \mathbf{V}_a 
    \end{aligned} 
    \;\rightarrow\; 
    \boxed{\begin{aligned} 
        V_L &= V_{ab} = V_{bc} = V_{ca} = 2V \cos(\pi/6) \\ 
            &= \sqrt{3} V \mkern9mu\text{\small (valor eficaz das compostas)}
    \end{aligned}} 
    $$ 
    
    \noindent Uma vez que a carga é simétrica, as correntes escrevem-se: 
    $$ \begin{aligned} 
        i_a &= \sqrt{2} I \sin(\omega t - \phi) \\ 
        i_b &= \sqrt{2} I \sin(\omega t - \phi - 2\pi/3) \\ 
        i_c &= \sqrt{2} I \sin(\omega t - \phi + 2\pi/3) 
    \end{aligned} 
    \qquad\rightleftarrows\qquad 
    \begin{aligned} 
        \mathbf{I}_{a} &= I e^{-j\phi} \\ 
        \mathbf{I}_{b} &= I e^{-j(2\pi/3 + \phi)} \\ 
        \mathbf{I}_{c} &= I e^{j(2\pi/3 - \phi)} 
    \end{aligned} 
    $$
\end{minipage}

\begin{mdframed} % miminhos
    \noindent \textbf{Notas}:
    \begin{enumerate}[noitemsep,nolistsep,leftmargin=*,label=\arabic*.,font=\small\bfseries]\small
        \item ``A soma das correntes nas três fases é nula, logo não é necessário um condutor a conectar o neutro do gerador com o da carga. Os dois neutros estão ao potencial da terra, quer no gerador quer na carga.''\cite{paiva2005}
        
        \item ``Num sistema trifásico simétrico, todas as tensões simples podem ser medidas em relação a um neutro, que tem o mesmo potencial (zero) ao longo de todo o sistema.''\cite{paiva2005}
    \end{enumerate}
\end{mdframed}

%//==============================--@--==============================//%
\subsubsection{Potência Instantânea, Potência Ativa e Reativa, e Potência Complexa}

A potência transferida do gerador para a carga será a soma das potências instantâneas por fase:
$$
    p = v_a i_a + v_b i_b + v_c i_c = 3 VI \cos(\phi)
$$
Verifica-se que a \textit{potência trifásica instantânea} é constante, e igual a 3 vezes a potência ativa por fase.

Deste modo, a \textit{potência ativa trifásica} escreve-se:
$$
    P = 3 VI \cos(\phi) = \sqrt{3} V_L I \cos (\phi) \quad [\text{W}]
$$
A \textit{potência reativa trifásica} é definida como a soma algébrica das potências reativas em cada fase:
$$
    Q = 3 VI \sin(\phi) = \sqrt{3} V_L I \sin (\phi) \quad [\text{VAr}]
$$
A \textit{potências complexa e aparente para sistemas trifásicos} é dada por:
$$
\begin{aligned}
    \mathbf{S} &= 3\mathbf{V} \mathbf{I}^* = P + jQ\\
    S &= \sqrt{P^2 + Q^2} = 3VI = \sqrt{3} V_L I \quad [\text{VA}]
\end{aligned}
$$

%//==============================--@--==============================//%
\subsubsection{Carga Ligada em Triângulo}

\noindent Outra forma de ligar a carga é em \underline{triângulo}, situação em que cada impedância de carga \textbf{Z} está sujeita à tensão entre fases.

\begin{centering}
    \begin{minipage}[b]{0.4\linewidth}
        \begin{figure}[H]
        \centering
        \resizebox{0.75\textwidth}{!}{%
        \begin{circuitikz}[>=stealth]
        % Nodes
        \draw (-2,2) to[short, -*] (-2,2);
        \draw (-2,0) to[short, -*] (-2,0);
        \draw (-2,-2) to[short, -*] (-2,-2);
        
        %Lines
        \draw [](-2,0) to[short] (1,0);
        \draw (1,0) to[short, -*] (1,0);
        \draw (1,0) to[R,*-*] (2.5,2);
        \draw (1,0) to[R,*-*] (2.5,-2);
        \draw (2.5,2) to[R,*-*] (2.5,-2);
        \draw (-2,2) to[short, -*] (2.5,2);
        \draw (-2,-2) to[short, -*] (2.5,-2);
        
        \node at (3,0) {$Z_\Delta$};
        \node at (1.1,1) {$Z_\Delta$};
        \node at (1.1,-1) {$Z_\Delta$};
        
        \node[coordinate,label=left:$a$] at (-2,2) {};
        \node[coordinate,label=left:$b$] at (-2,0) {};
        \node[coordinate,label=left:$c$] at (-2,-2) {};
        
        % Arrows
        \draw[->,thick](-2, 1.8) to[short] (-2,0.2);
        \draw[->,thick](-2, -0.2) to[short] (-2,-1.8);
        \draw[->,thick](-2.5, -1.8) to[short] (-2.5,1.8);
        
        \node at (-2.9,0) {$V_{ca}$};
        \node at (-1.7,1) {$V_{ab}$};
        \node at (-1.7,-1) {$V_{bc}$};
        
        \draw[->,thick](0, 2) to[short] (0.1,2) node[above] {$I_a$};
        \draw[->,thick](0, 0) to[short] (0.1,0) node[above] {$I_b$};
        \draw[->,thick](0, -2) to[short] (0.1,-2) node[above] {$I_c$};
        
        \draw[->,thick](2.5, -1.5) to[short] (2.5,-1.3) node[right=0.1] {$I_{ca}$};
        
        
        \end{circuitikz}
        }%
        \caption{Carga ligada em triângulo.}
        \end{figure}
\end{minipage}%
    \begin{minipage}[b]{0.6\linewidth}
        \begin{mdframed}
            \noindent As correntes $\bar{I}_{ab}$, $\bar{I}_{bc}$ e $\bar{I}_{ca}$ são:
            \vspace{0.5em}
            $$
                \bar{I}_{ab} = \dfrac{\overline{V}_{ab}}{\overline{Z}_\Delta}\quad
                \bar{I}_{bc} = \dfrac{\overline{V}_{bc}}{\overline{Z}_\Delta}\quad
                \bar{I}_{ac} = \dfrac{\overline{V}_{ac}}{\overline{Z}_\Delta}
            $$
            \vspace{0.5em}
            \noindent A corrente na linha $I_a$ é, por conseguinte:
            $$
                \boxed{\bar{I}_a = \dfrac{\overline{V}_{ab} - \overline{V}_{ca}}{\overline{Z}_\Delta} = \dfrac{3\overline{V}_a}{\overline{Z}_\Delta}}
            $$
            \noindent A amplitude da corrente é \textbf{3 vezes maior} que na ligação da carga em estrela.
        \end{mdframed}
    \end{minipage}
\end{centering}

\noindent A \underline{potência absorvida} pela carga ligada em triângulo é \textbf{3 vezes maior} que a ligada em estrela, para o mesmo valor de impedância de carga.

%//==============================--@--==============================//%
\paragraph{Conversão entre Ligação em Estrela e Ligação em Triângulo (wye $\rightleftarrows$ delta)}

\begin{table}[ht]
    \centering
    \caption{Transformações Y-$\Delta$ e $\Delta$-Y}

    \setlength{\tabcolsep}{1cm}
    
    \begin{tabular}{cc}
        \begin{circuitikz}
            \ctikzset{resistors/scale=0.75}
            
            % Define interest points
            \coordinate (A) at (0,0);
            \coordinate (B) at (4,0);
            \coordinate (C) at (2,{-2*sqrt(3)});
            \coordinate (G) at (2,{-(2/3)*sqrt(3)});
    
            % Draw the skeleton
            \draw[dashed] (A) -- (G) -- (B);
            \draw[dashed] (G) -- (C);
    
            \node[circ] at (A) {};
            \node[circ] at (B) {};
            \node[circ] at (C) {};
            \node[circ] at (G) {};

            \node[left] at (A) {$a$};
            \node[right] at (B) {$b$};
            \node[below] at (C) {$c$};
    
            % Draw the impedances
            \draw (A) to[R, a=$Z_{ab}$] (B) to[R, a=$Z_{bc}$] (C) to[R, a=$Z_{ca}$] (A);
        \end{circuitikz}
        &
        \begin{circuitikz}
            \ctikzset{resistors/scale=0.75}
            
            % Define interest points
            \coordinate (A) at (0,0);
            \coordinate (B) at (4,0);
            \coordinate (C) at (2,{-2*sqrt(3)});
            \coordinate (G) at (2,{-(2/3)*sqrt(3)});
    
            % Draw the skeleton
            \draw[dashed] (A) -- (B) -- (C) -- (A);
            \node[circ] at (A) {};
            \node[circ] at (B) {};
            \node[circ] at (C) {};
            \node[circ] at (G) {};

            \node[left] at (A) {$a$};
            \node[right] at (B) {$b$};
            \node[below] at (C) {$c$};
            
            % Draw the impedances
            \draw (A) to[R, l=$Z_{a}$] (G) to[R, l=$Z_{b}$] (B);
            \draw (C) to[R, l=$Z_{c}$] (G);
        \end{circuitikz} \\
        $\begin{aligned}
            \boxed{\textbf{Y} \to \mathbf{\Delta}}\\
            Z_{ab} &= \frac{Z_aZ_b+Z_bZ_c+Z_cZ_a}{Z_c} \\
            Z_{bc} &= \frac{Z_aZ_b+Z_bZ_c+Z_cZ_a}{Z_a} \\
            Z_{ca} &= \frac{Z_aZ_b+Z_bZ_c+Z_cZ_a}{Z_b}
        \end{aligned}$
        & 
        $\begin{aligned}
            \boxed{\mathbf{\Delta} \to \textbf{Y}}\\
            Z_a &= \frac{Z_{ab}Z_{ca}}{Z_{ab}+Z_{bc}+Z_{ca}} \\
            Z_b &= \frac{Z_{bc}Z_{ab}}{Z_{ab}+Z_{bc}+Z_{ca}} \\
            Z_c &= \frac{Z_{ca}Z_{cb}}{Z_{ab}+Z_{bc}+Z_{ca}}
        \end{aligned}$ 
    \end{tabular}
\end{table}

%//==============================--@--==============================//%
\vspace*{-1em}%
\subsection{Valores por Unidade}

O uso de \textit{valores por unidade} (p.u.) consiste na quantificação das grandezas elétricas como frações de valores de base, designados por valores nominais ou de plena carga. O valor p.u. de uma grandeza obtém-se por
$$
    \text{valor p.u.} = \frac{\text{valor da grandeza}}{\text{valor da base}}
$$
\textbf{Nota}: O valor da grandeza pode ser um fasor/número complexo ou um valor instantâneo em unidades SI, enquanto o valor de base é um número real adequado. O valor de base pode ser \underline{postulado} ou \underline{derivado}.

%//==============================--@--==============================//%
\subsubsection{Sistemas Monofásicos}

\begin{mdframed}
    \noindent Num sistema monofásico, postula-se: 
    \begin{center}
        a \textit{base de tensão} $[$kV$]$, $V_b$ e a \textit{base de potência} $[$MVA$]$, $S_b$
    \end{center}
    \noindent Os valores de base derivados são:
    \begin{itemize}[noitemsep,label=-,font=\bfseries]
        \item \textit{base de corrente} $[$kA$]$, $$ I_b = S_b/V_b $$
        \item \textit{base de impedância} $[\Omega]$, $$ Z_b = V_b/I_b = V^2_b/S_b $$
        \item \textit{base de admitância} $[$S$]$, $$ Y_b = I_b/V_b = S_b/V^2_b$$
    \end{itemize}
\end{mdframed}

%//==============================--@--==============================//%
\subsubsection{Sistemas Trifásicos}

\begin{mdframed}
Analogamente, postula-se para base a tensão \underline{entre fases}, $V_b$, e a potência aparente \underline{trifásica}, $S_b$.

\vspace{1em}
\noindent Temos a relação $S_b = \sqrt{3} V_b I_b$, de onde se derivam as restantes:
\begin{itemize}[noitemsep,label=-,font=\bfseries]
        \item \textit{base de corrente} $[$kA$]$, $$ I_b = S_b/\sqrt{3}V_b $$
        \item \textit{base de impedância} $[\Omega]$, $$ Z_b = V_b/\sqrt{3}I_b = V^2_b/S_b $$
        \item \textit{base de admitância} $[$S$]$, $$ Y_b = \sqrt{3}I_b/V_b = S_b/V^2_b$$
    \end{itemize}
\end{mdframed}

%//==============================--@--==============================//%
\subsection{Transmissão de Energia}
\subsubsection{Corrente Alternada}
Considerando uma linha de transmissão de energia modelada por um elemento indutivo com reatância $X_L$, pretende-se estabelecer a relação entre as potências ativa e reativa que transitam na linha e as tensões nos nós entre as quais ela está ligada.

\vspace{-0.5em}
\begin{centering}
    \begin{minipage}[b]{0.5\linewidth}
        \begin{figure}[H]
        \centering
        \resizebox{0.8\textwidth}{!}{%
        \begin{circuitikz}[>=stealth]
        
        % Inductor
        \draw (0,0) to[cute inductor] (3,0);
        \node at (1.5,0.4) {$j X_L$};
        
        % Nodes
        \draw (0, 0) to[short, -*] (0,0);
        \draw (3,0) to[short, -*] (3,0);
        \draw (0,-2) to[short, -*] (0,-2);
        \draw (3,-2) to[short, -*] (3,-2);
        
        % Wire
        \draw (0, -2) to[short, -*] (3,-2);
        
        % Arrows
        \draw[->,thick](0, -0.2) to[short] (0,-1.8);
        \node at (-0.3,-1) {$V_1$};
        
        \draw[->,thick](3, -0.2) to[short] (3,-1.8);
        \node at (3.3,-1) {$V_2$};
        
        \draw[->,thick](0.5, 0) to[short] (0.55,0) node[above] {$I$};
        
        \end{circuitikz}
        }%
        \caption{Transmissão de energia --- elemento indutivo.}
        \end{figure}
\end{minipage}%
    \begin{minipage}[b]{0.5\linewidth}
        \begin{mdframed}
            A corrente que percorre a linha, definida como positiva no sentido $1 \rightarrow 2$:
            $$
                I = \dfrac{V_1 - V_2}{j X_L}
            $$
            A potência complexa $\mathbf{S}_{12}$:
            $$
            \begin{aligned}
                &\mathbf{S}_{12} = \mathbf{V_1} \mathbf{I^*} = \mathbf{V_1}\dfrac{ \mathbf{V}_1^* -  \mathbf{V}_2^*}{-j X_L} =\\ &\dfrac{ V_1^2 - \mathbf{V}_1\mathbf{V}_2^*}{-j X_L}\quad\text{em que}\quad
                \left\{\begin{aligned}
                    \mathbf{V}_1 &= V_1 e^{j\theta_1} \\
                    \mathbf{V}_2 &= V_2 e^{j\theta_2}
                \end{aligned}\right.
            \end{aligned}
            $$
        \end{mdframed}
    \end{minipage}
\end{centering}

\noindent Seja $\theta = \theta_1 - \theta_2$ o ângulo de desfasagem entre as tensões no nó 1 e no nó 2:
$$
\mathbf{S_{12}} = j\dfrac{ V_1^2 -  V_1 V_2 e^{j\theta}}{X_L} = \underbrace{\dfrac{V_1 V_2 \sin(\theta)}{X_L}}_{P_{12}} + \underbrace{j \dfrac{V_1^2 - V_1 V_2 \cos(\theta)}{X_L}}_{Q_{12}}
$$
De forma análoga se deduzem as potência ativa e reativa na receção, positivas no sentido $2 \rightarrow 1$:
$$
P_{21} = - \dfrac{V_1 V_2 \sin(\theta)}{X_L}\qquad 
Q_{21} = \dfrac{V_2^2 - V_1 V_2 \cos(\theta)}{X_L}
$$
Somando as duas equações vem que:
$$
\begin{aligned}
    &P_L = P_{12} + P_{21} = 0\\
    &Q_L = Q_{12} + Q_{21} = \dfrac{V_1^2 + V_2^2 - 2 V_1 V_2 \cos(\theta)}{X_L}
\end{aligned}
$$
em que $P_L$ e $Q_L$ representam as perdas de potência ativa e reativa na linha. Uma vez que desprezamos a resistência, as perdas de potência ativa são nulas. As perdas de potência reativa não correspondem a perdas energéticas, no entanto, \underline{o balanço da potência reativa deve de ser fechado}\footnotemark[1] como a potência ativa.

\footnotetext[1]{%
    Caso contrário, o desequilíbrio provocará flutuações nas tensões em diferentes partes do sistema, que levam a ineficiências energéticas. 
}

\begin{mdframed}
   \noindent \textbf{Notas}:
    \begin{enumerate}[nolistsep,noitemsep,leftmargin=*,label=\arabic*.,font=\small\bfseries]\small
        \item O sentido do trânsito de potência ativa é determinado pelo ângulo de desfasagem $\theta$ entre as tensõe de cada nó.
        
        \item As amplitudes das tensões $V_1$ e $V_2$ influenciam o sentido de trânsito da potência reativa. Caso as tensões sejam de igual amplitude nos dois extremos $V_1 = V_2 = V_n$ (tensão nominal), então:
        $$
            Q_{\textit{med}} = \dfrac{Q_{12} - Q_{21}}{2} = 0
        $$
        O mesmo não acontece com o balanço de potência reativa $Q_L = Q_{12} + Q_{21}$, uma vez que os valores nos extremos não se anulam, i.e., $Q_{12} = Q_{21} = V^2_n (1-\cos(\theta))/X_L$. O que leva a 
        $$
            Q_L = \frac{2V^2_n (1-\cos(\theta))}{X_L}
        $$
    \end{enumerate} 
\end{mdframed}

%//==============================--@--==============================//%
\subsection{Caracterização das Cargas}

As cargas típicas têm \underline{caráter indutivo}, e agrupam-se em quatro tipos: (i) motores, (ii) iluminação, (iii) aquecimento e refrigeração e (iv) aparelhos eletrónicos. Normalmente são caracterizadas pela potência ativa $P_C$ e pela potência reativa $Q_C$ ou fator de potência $\cos(\phi)$ (alternativamente poderá utilizar-se $\tan(\phi)$):
$$
    \cos(\phi) = \frac{P_C}{\sqrt{P^2_C + Q^2_C}} \quad \land \quad \tan(\phi) = \frac{P_C}{Q_C}
$$

%//==============================--@--==============================//%

    \clearpage
    \section{Transformador e Máquina Assíncrona}%
       %//==============================--@--==============================//%
\subsection{Equações de Maxwell}
\label{subsec:maxwell-eq}

É necessário um conjunto de quatro vetores para descrever os fenómenos do campo eletromagnético:
\begin{itemize}
    \item[] o campo elétrico, $\mathbf{E}$ (unidades: V/m, volt por metro)
    \item[] o campo de indução magnética, $\mathbf{B}$ (unidades: T, tesla)
    \item[] o campo de deslocamento elétrico, $\mathbf{D}$ (unidades: C/m$^2$ , coulomb por metro quadrado)
    \item[] o campo magnético, $\mathbf{H}$ (unidades: A/m, ampère por metro)
\end{itemize}
\noindent Entre estes, os dois primeiros têm significado físico especial, uma vez que podem ser determinados experimentalmente e medidos.

Para fenómenos eletromagnéticos variáveis no tempo consideraram-se as equações de Maxwell:
$$
    \begin{cases}%
        \text{rot } \mathbf{E} = -\dfrac{\partial \mathbf{B}}{\partial t} & \text{\small (Lei de Indução)} \\
        \text{div } \mathbf{D} = \rho & \text{\small (Lei de Gauss)} \\
        \text{rot } \mathbf{H} = \mathbf{J} + \dfrac{\partial \mathbf{D}}{\partial t} & \text{\small (Lei de Ampère com a correção de Maxwell)} \\
        \text{div } \mathbf{B} = 0 & \text{\small (Lei de Gauss para o magnetismo)}
    \end{cases}
$$
em que $\mathbf{D} = \varepsilon \mathbf{E}$ (onde $\varepsilon$ é a permitividade do meio, em F/m), $\mathbf{J} = \sigma \mathbf{E}$ (onde $\sigma$ é a condutividade do condutor, em S/m) e $\mathbf{B} = \mu \mathbf{H}$ (onde $\mu$ representa a permeabilidade do meio, em H/m).

%//==============================--@--==============================//%
\subsection{Circuitos Magnéticos}
\label{subsec:circuitos-magneticos}

É importante revisitar os conceitos fundamentais dos circuitos magnéticos: 
$$
    \begin{aligned}
        \text{\underline{Tensão magnética}: }& u_m \delequal \int_C \mathbf{H} \cdot d\mathbf{r} \\
        \text{\underline{Fluxo magnético}: }& \Phi_m \delequal \int\!\!\int_S \mathbf{B} \cdot \mathbf{n}\; dS\\
        \text{\underline{Relutância magnética}: }& R_m \delequal \frac{u_m}{\Phi_m}
    \end{aligned}
$$
Salienta-se que o fluxo magnético obedece à \textit{Lei dos Cortes} --- análoga ao KCL --- uma vez que $\text{div } \mathbf{B} = 0$, temos que $\sum_i^n \Phi_{m_i} = 0$ numa superfície fechada (ou ``nó''); enquanto, em geral, a tensão magnética \textbf{não} obedece a nenhuma lei semelhante ao KVL.

\vspace{0.5em}
\noindent As \hyperref[subsec:maxwell-eq]{equações de Maxwell acima} condensam duas leis essenciais a este capítulo:
\begin{enumerate}
    \item[] \textbf{Lei de Indução}: A expressão <<$\text{rot } \mathbf{E} = -\partial \mathbf{B}/\partial t$>> diz que um campo magnético que varia com o tempo é sempre acompanhado por um campo elétrico não-conservativo que varia espacialmente, e vice-versa.
    $$
        \iint_{S_N} \text{rot } \mathbf{E} \cdot \mathbf{n}\, dS = -\iint_{S_N} \dfrac{\partial \mathbf{B}}{\partial t} \cdot \mathbf{n}\, dS \iff \oint_{\partial S_N} \mathbf{E} \cdot d\mathbf{r} = -\dfrac{d}{dt} \iint_{S_N} \mathbf{B} \cdot \mathbf{n}\, dS 
    $$
    $$
        \boxed{ \therefore \text{e.m.f.} = -\frac{d \Psi}{dt} = -N\frac{d \Phi_m}{dt} }
    $$
    em que $\Psi = N\Phi_m$ é definido como o \textit{fluxo magnético ligado}.

    \item[] \textbf{Lei de Ampère}: A expressão <<$\text{rot } \mathbf{H} = \mathbf{J} + \partial \mathbf{D}/\partial t$>> afirma que campos magnéticos podem ser criados de duas formas: através de correntes elétricas, que é a lei de Ampère original, e por campos elétricos que variam no tempo, que é a correção proposta por Maxwell (nesta UC não consideramos este fenómeno).
    $$
        \iint_{S_N} \text{rot } \mathbf{H} \cdot \mathbf{n}\, dS = \iint_{S_N} \mathbf{J} \cdot \mathbf{n}\, dS \iff \oint_{\partial S_N} \mathbf{H} \cdot d\mathbf{r} = I_{\partial S} = N I > 0 \text{ (quando $\mathbf{H}$ é concordante com $d\mathbf{r}$)}
    $$
\end{enumerate}

\noindent O fluxo ligado é uma quantidade proporcional à corrente --- de onde se define o coeficiente de indução,
$$
    \Psi = N \Phi = \iint_{S_N} \mathbf{B} \cdot \mathbf{n}\, dS = L I
$$

\renewcommand{\thefootnote}{\fnsymbol{footnote}}
\footnotetext[4]{%
    A constante $N$ representa, naturalmente, o número de espiras do enrolamento em questão.
}
\renewcommand{\thefootnote}{\arabic{footnote}}

%//==============================--@--==============================//%
\subsection{Transformador}
\label{subsec:transformador}

%//==============================--@--==============================//%
\subsubsection{Funcionamento do Transformador}
\label{subsec:transformador-funcionamento}

%//==============================--@--==============================//%
\paragraph{Transformador Ideal}
\label{subsubsec:transformador-ideal}

Para o transformador monofásico ideal assumimos duas aproximações: enrolamentos com resistência nula e circuito magnético com relutância igualmente nula (isto implica que não existe dispersão).

\vspace{-0.75em}
\begin{minipage}[c]{0.35\linewidth}
    \begin{figure}[H]
        \centering
        \begin{circuitikz}[>=stealth, scale=0.9, american]
            \draw (0,0) node[transformer] (T) {};
            \draw (T.A1) -- ++(-1,0) node[circ] {};
            \draw (T.A2) -- ++(-1,0) node[circ] {};
            \draw (T.B1) -- ++(1,0) node[circ] {};
            \draw (T.B2) -- ++(1,0) node[circ] {};
    
            \node[left=4.5mm,above=4mm] at (T.outer dot A2) {$N_1$};
            \node[right=4.5mm,above=4mm] at (T.outer dot B2) {$N_2$};
    
            \node[draw,dotted,fit=(T.inner dot A1)(T.inner dot A2)(T.inner dot B1)(T.inner dot B2),inner sep=1.25mm] (dottedrect) {};
            
            \node[above] at (dottedrect.north) {$\Phi$};
            \draw[->] (dottedrect.north) -- ++(0.075,0);
    
            % voltage drops
            \draw[->] ([xshift=-1cm, yshift=-1.5mm]T.A1) -- ([xshift=-1cm, yshift=+1.5mm]T.A2) node[midway,left] {$v_1$};
            \draw[->] ([xshift=+1cm, yshift=-1.5mm]T.B1) -- ([xshift=+1cm, yshift=+1.5mm]T.B2) node[midway,right] {$v_2$};
    
            % currents
            \draw[->] ([xshift=-0.50cm]T.A1) -- ([xshift=-0.05cm]T.A1) node[above] {$i_1$};
            \draw[->] (T.B1) -- ([xshift=+0.25cm]T.B1) node[above] {$i_2$};

            % add transformer dots
            \node[circ,scale=0.75,yshift=-2.75mm] at (T.outer dot A1) {};
            \node[circ,scale=0.75,yshift=-2.75mm] at (T.outer dot B1) {};
        \end{circuitikz}
        \caption{Transformador ideal.}
        \label{fig:ideal-transformer}
    \end{figure}
\end{minipage}
\begin{minipage}[c]{0.6\linewidth}
    \noindent Neste primeiro modelo temos as seguintes relações:
    $$
        \begin{dcases}
            v_1 = N_1 \frac{d \Phi}{dt}\\
            v_2 = N_2 \frac{d \Phi}{dt}
        \end{dcases}
        \implies
        \begin{dcases}
            \mathbf{V}_1 = j\omega N_1 \Phi \\
            \mathbf{V}_2 = j\omega N_2 \Phi
        \end{dcases}
        \implies \frac{\mathbf{V}_1}{\mathbf{V}_2} = \frac{V_1}{V_2} = \frac{N_1}{N_2}
    $$
\end{minipage}

\vspace{0.75em}
\noindent Uma vez que a resistência dos enrolamentos é nula e a reatância de dispersão também é nula, \underline{não há perdas} de potência ativa nem de potência reativa. \underline{A potência complexa é igual nos dois lados} do transformador:
$$
    \mathbf{S}_1 = \mathbf{S}_2 \iff \mathbf{V}_1 \mathbf{I}_1^* = \mathbf{V}_2 \mathbf{I}_2^* \implies \frac{\mathbf{I}_1}{\mathbf{I}_2} = \frac{I_1}{I_2} = \frac{N_2}{N_1}
$$
É útil definir a relação de transformação $m$, o quociente entre o número de espiras do primário (enrolamento que recebe energia) e do secundário (enrolamento que cede energia):
$$
    m = \frac{\mathbf{V}_1}{\mathbf{V}_2} = \frac{N_1}{N_2} = \frac{V_{n1}}{V_{n2}} \; \text{kV/kV}
$$
onde $V_{n1}$ é a tensão nominal primária e $V_{n2}$ a tensão nominal secundária.

\renewcommand{\thefootnote}{\fnsymbol{footnote}}
\footnotetext[4]{%
    A tensão nominal, também conhecida como tensão nominal de operação, é um valor específico de tensão elétrica que um equipamento, dispositivo ou sistema elétrico é projetado para operar de forma ideal e segura.
}
\renewcommand{\thefootnote}{\arabic{footnote}}

%//==============================--@--==============================//%
\paragraph{Não Idealidades do Transformador e Corrente de Magnetização}
\label{subsubsec:corrente-magnetizacao}

O núcleo do transformador é normalmente constituído por ferro, que possui uma característica B-H não linear: a partir de um certo valor dos campos manifesta-se saturação. Acresce-se ainda o fenómeno da histerese, i.e., as trajetórias B-H são distintas para valores crescentes ou decrescente do campo magnético.

\begin{minipage}[b]{0.275\linewidth}
   \begin{figure}[H]
        \centering
        \scalebox{0.6}{%
            \begin{tikzpicture}
                \begin{axis}[very thick,
                             samples = 100,
                             xlabel = H,
                             ylabel = B,
                             xmin = -6,
                             xmax = 6,
                             ymin = -4,
                             ymax = 4,
                             axis x line = middle,
                             axis y line = middle,
                             ticks = none]
                    \addplot[dashed] plot (\x, 2.5);
                    \addplot[dashed] plot (\x,-2.5);
                    \addplot[red, name path=A] plot (\x, {5/(1 + exp(-1.7*\x+1.5))-2.5});
                    \addplot[red, name path=B] plot (\x, {5/(1 + exp(-1.7*\x-1.5))-2.5});
                    \addplot[red!20] fill between[of=A and B];
                \end{axis}
            \end{tikzpicture}
        }
        \caption{Característica magnética do núcleo do transformador}
        \label{fig:hysteresis-iron}
    \end{figure} 
\end{minipage}\hfill
\begin{minipage}[b]{0.65\linewidth}
    O ponto de funcionamento na curva B-H está normalmente localizado próximo ao cotovelo que marca o início da saturação.

    \hspace{1em} O fluxo magnético alternado dá origem a perdas no núcleo de ferro devidas à histerese e às correntes de fuga. ``As primeiras resultam da energia necessária para orientar os domínios magnéticos do material na direção do campo; as segundas devem-se ao efeito de Joule resultante das correntes induzidas no ferro''\cite{paiva2005}.

    \hspace{1em} Uma vez que a permeabilidade do ferro não é infinita, a relutância do circuito magnético não é nula. Introduz-se a \textit{corrente de magnetização} necessária para criar o campo magnético $\mathbf{H}$, fornecida pela rede/gerador que alimenta o transformador.
\end{minipage}

\vspace{0.75em}
\noindent A componente fundamental da corrente de magnetização, à frequência nominal, pode medir-se através de um ensaio em vazio do transformador (que \hyperref[subsec:analise-transformador]{veremos em seguida}).

O esquema equivalente representa-se na \hyperref[fig:esquema-equiv-transformador]{Fig. 12}: as componentes em fase e em quadratura da corrente de magnetização circulam através da condutância $G_m$ e suscetância $B_m$, respetivamente. A componente transversal do esquema equivalente, pertence à modelação da corrente de magnetização. As componentes das impedâncias longitudinais são devido à resistência dos condutores e à reatância de dispersão. 

\vspace{-0.5em}
\begin{theo}[Perdas no Transformador]{def:perdas-transformador}
    Cumulativamente, podemos calcular as perdas no transformador conforme a diferença
    $$
        \text{Perdas} = P_1 - P_2 = P_{cu} + P_{fe}
    $$
    em que $P_1$ e $P_2$ representam a potência ativa no primário e no secundário, respetivamente. A distribuição destas perdas corresponde à contribuição do circuito elétrico (nos enrolamentos de cobre) e do circuito magnético (no núcleo de ferro):
    $$
        P_{cu} = R_1 I_1^2 + R_2 I_2^2 \simeq R_t I_2^2 \quad\land\quad P_{fe} \simeq G_m V_1^2
    $$
\end{theo}

%//==============================--@--==============================//%
\clearpage
\paragraph{Esquema Equivalente do Transformador}
\label{subsubsec:equiv-esquema}

Um primeiro esquema equivalente do transformador pode ser dado por:

\begin{figure}[H]
\centering
    %\resizebox{0.8\textwidth}{!}{%
    \ctikzset{bipoles/resistor/height=0.20}
    \ctikzset{bipoles/resistor/width=0.5}
        \begin{circuitikz}[>=stealth,american,scale=0.95]
            
            % Resistor
            \draw (2,0) to[R] (4,0);
            \node at (3,0.4) {$R_1$};
            
            \draw (5.5,-1) to[R] (5.5, -2);
            \node at (5,-1.5) {$G_m$};
            
            \draw (9,0) to[R] (11, 0);
            \node at (10,0.4) {$R_2$};
            
            % Inductor
            \draw (4,0) to[cute inductor] (6,0);
            \node at (5,0.5) {$jX_1$};
            
            \draw (8,0) to[cute inductor] (8,-3);
            \draw (9,-3) to[cute inductor] (9,0);
            
            \draw (6.5,-1) to[cute inductor] (6.5,-2);
            \node at (7.15,-1.5) {$jB_m$};
            
            \draw (11,0) to[cute inductor, -*] (13,0);
            \node at (12,0.5) {$jX_2$};
            
            % Wires
            \draw (6,0) to[short, *-] (8, 0);
            \draw (5.5,-0.5) to[short, -] (6.5, -0.5);
            \draw (5.5,-2.5) to[short, -] (6.5, -2.5);
            \draw (5.5,-1) to[short, -] (5.5, -0.5);
            \draw (6.5,-1) to[short, -] (6.5, -0.5);
            \draw (5.5,-2) to[short, -] (5.5, -2.5);
            \draw (6.5,-2) to[short, -] (6.5, -2.5);
            \draw (6,0) to[short, *-] (6, -0.5);
            \draw (6,-3) to[short, *-] (6, -2.5);
            \draw (6,-3) to[short, -] (8, -3);
            \draw (1,-3) to[short, *-] (6, -3);
            \draw (9,-3) to[short, -*] (13, -3);
            
            % Nodes
            \draw (2, 0) to[short, -*] (1,0);
            \draw (6,0) to[short, -*] (6,0);
            \draw [fill=black] (7.87, -1.1)node(a){} circle (1pt);
            \draw [fill=black] (9.13, -1.1)node(a){} circle (1pt);
            
            % Arrows
            \draw[->,thick](1, -0.2) to[short] (1,-2.8);
            \node at (0.7,-1.5) {$V_1$};
            
            \draw[->,thick](13, -0.2) to[short] (13,-2.8);
            \node at (13.3, -1.5) {$V_2$};
            
            \draw[->,thick](9.3, -0.4) to[short] (9.3,-2.6) node[right] {$E_2$};
            \draw[->,thick](7.7, -0.4) to[short] (7.7,-2.6) node[left] {$E_1$};
            
            \draw[->,thick](1.7, 0) to[short] (1.8,0) node[above] {$I_1$};
            \draw[->,thick](7, 0) to[short] (7.1,0) node[above] {$I_2'$};
            \draw[->,thick](12.6, 0) to[short] (12.7,0) node[above] {$I_2$};
            \draw[->,thick](6, -0.3) to[short] (6,-0.4);
            \node at (6.3, -0.25) {$I_m$};
            
        \end{circuitikz}
    %}%
    
    \caption{Esquema Equivalente do transformador.}
    \label{fig:esquema-equiv-transformador}
\end{figure}
\vspace{-1em}
$$
    \left\{\begin{aligned}
        E_1 &= V_1 -  \left(RI_1 + jX_1I_1\right)\\
        E_2 &= V_2 +  RI_2 + jX_2I_2
    \end{aligned}\right.\qquad
    \left\{\begin{aligned}
        I_1 &= I_m + I_2'\\
        I_2' &= \frac{N_2}{N_1} I_2
    \end{aligned}\right.   
$$
Se tomarmos as tensões de base, do lado do primário e do secundário, pelas respetivas tensões nominais, i.e., $V_{b1} = V_{n1}$ e $V_{b2} = V_{n2}$, a relação do transformador em valores p.u. é
$$
    m = \frac{V_{{n1}_{pu}}}{V_{{n2}_{pu}}} = \frac{V_{n1}}{V_{b1}} \frac{V_{b2}}{V_{n2}} = 1.0\; \text{p.u.}
$$
Esta conclusão indica que o transformador ideal \underline{pode ser removido do esquema equivalente} da rede, uma vez que a relação de transformação é unitária. 

Chegamos assim ao esquema equivalente em T:
\begin{figure}[H]
    \centering
    %\resizebox{0.65\textwidth}{!}{%
    \ctikzset{bipoles/resistor/height=0.20}
    \ctikzset{bipoles/resistor/width=0.5}
        \begin{circuitikz}[>=stealth,american,scale=0.95]
            
            % Resistor
            \draw (2,0) to[R] (4,0);
            \node at (3,0.4) {$R_1$};
            
            \draw (5.5,-1) to[R] (5.5, -2);
            \node at (5,-1.5) {$G_m$};
            
            \draw (6,0) to[R] (8, 0);
            \node at (7,0.4) {$R_2$};
            
            % Inductor
            \draw (4,0) to[cute inductor] (6,0);
            \node at (5,0.5) {$jX_1$};
            
            \draw (6.5,-1) to[cute inductor] (6.5,-2);
            \node at (7.15,-1.5) {$jB_m$};
            
            \draw (8,0) to[cute inductor, -*] (10,0);
            \node at (9,0.5) {$jX_2$};
            
            % Wires
            \draw (5.5,-0.5) to[short, -] (6.5, -0.5);
            \draw (5.5,-2.5) to[short, -] (6.5, -2.5);
            \draw (5.5,-1) to[short, -] (5.5, -0.5);
            \draw (6.5,-1) to[short, -] (6.5, -0.5);
            \draw (5.5,-2) to[short, -] (5.5, -2.5);
            \draw (6.5,-2) to[short, -] (6.5, -2.5);
            \draw (6,0) to[short, *-] (6, -0.5);
            \draw (6,-3) to[short, *-] (6, -2.5);
            \draw (1,-3) to[short, *-] (6, -3);
            \draw (6,-3) to[short, -*] (10, -3);
            
            % Nodes
            \draw (2, 0) to[short, -*] (1,0);
            \draw (6,0) to[short, -*] (6,0);
        
            
            % Arrows
            \draw[->,thick](1, -0.2) to[short] (1,-2.8);
            \node at (0.7,-1.5) {$V_1$};
            
            \draw[->,thick](10, -0.2) to[short] (10,-2.8);
            \node at (10.3, -1.5) {$V_2$};
            
            \draw[->,thick](1.7, 0) to[short] (1.8,0) node[above] {$I_1$};
            \draw[->,thick](9.6, 0) to[short] (9.7,0) node[above] {$I_2$};
            \draw[->,thick](6, -0.3) to[short] (6,-0.4);
            \node at (6.3, -0.25) {$I_m$};
            
        \end{circuitikz}
    %}%
    
    \caption{Esquema Equivalente em T do transformador.}
\end{figure}

\noindent O fluxo no núcleo mantém-se constante pelo que as admitâncias do ramo transversal que modelam a corrente de magnetização se podem considerar constantes. \underline{A corrente de magnetização é pequena} (alternativamente, a impedância do ramo transversal é muito maior que impedância longitudinal dos dois lado), logo, o ramo transversal pode ser levado para um dos extremos, resultando no esquema em L:
\begin{figure}[H]
    \centering
    %\resizebox{0.55\textwidth}{!}{%
    \ctikzset{bipoles/resistor/height=0.20}
    \ctikzset{bipoles/resistor/width=0.5}
        \begin{circuitikz}[>=stealth,american,scale=0.95]
            
            % Resistor
            \draw (5.5,-1) to[R] (5.5, -2);
            \node at (5,-1.5) {$G_m$};
            
            \draw (6,0) to[R] (8, 0);
            \node at (7,0.4) {$R_t$};
            
            % Inductor
            \draw (6.5,-1) to[cute inductor] (6.5,-2);
            \node at (7.15,-1.5) {$jB_m$};
            
            \draw (8,0) to[cute inductor, -*] (10,0);
            \node at (9,0.5) {$jX_t$};
            
            % Wires
            \draw (5.5,-0.5) to[short, -] (6.5, -0.5);
            \draw (5.5,-2.5) to[short, -] (6.5, -2.5);
            \draw (5.5,-1) to[short, -] (5.5, -0.5);
            \draw (6.5,-1) to[short, -] (6.5, -0.5);
            \draw (5.5,-2) to[short, -] (5.5, -2.5);
            \draw (6.5,-2) to[short, -] (6.5, -2.5);
            \draw (6,0) to[short, *-] (6, -0.5);
            \draw (6,-3) to[short, *-] (6, -2.5);
            \draw (3,-3) to[short, *-] (6, -3);
            \draw (6,-3) to[short, -*] (10, -3);
            
            % Nodes
            \draw (6, 0) to[short, -*] (3,0);
            \draw (6,0) to[short, -*] (6,0);
        
            
            % Arrows
            \draw[->,thick](3, -0.2) to[short] (3,-2.8);
            \node at (2.7,-1.5) {$V_1$};
            
            \draw[->,thick](10, -0.2) to[short] (10,-2.8);
            \node at (10.3, -1.5) {$V_2$};
            
            \draw[->,thick](3.7, 0) to[short] (3.8,0) node[above] {$I_1$};
            \draw[->,thick](9.6, 0) to[short] (9.7,0) node[above] {$I_2$};
            \draw[->,thick](6, -0.3) to[short] (6,-0.4);
            \node at (6.3, -0.25) {$I_m$};
            
        \end{circuitikz}
    %}%
    
    \caption{Esquema Equivalente em L do transformador.}
\end{figure}
$$
    \left\{\begin{aligned}
        R_t &= R_1 + R_2\\
        X_t &= X_1 + X_2
    \end{aligned}\right.\qquad
    Z_t = R_t + jX_t
$$

%//==============================--@--==============================//%
\subsubsection{Análise do Transformador}
\label{subsec:analise-transformador}
%//==============================--@--==============================//%
\paragraph{Ensaio em Vazio}
Aplica-se a tensão nominal ao enrolamento que produz a menor corrente de magnetização, e o outro enrolamento em aberto.

\begin{figure}[H]
    \centering
    \ctikzset{bipoles/resistor/height=0.20}
    \ctikzset{bipoles/resistor/width=0.5}
    \begin{circuitikz}[=>stealth,american]
        % circuito
        \draw (0,0) to [short, *-] (3,0);

        \draw (3,-0.5) to [short] (3,0);
        \draw (2.5,-0.5) to [short] (3.5,-0.5);
        \draw (2.5,-0.5) to [R, l_=$G_m$] (2.5,-2.5);
        \draw (3.5,-0.5) to [cute inductor, l=$jB_m$] (3.5,-2.5);
        \draw (2.5,-2.5) to [short] (3.5,-2.5);
        \draw (3,-2.5) to [short] (3,-3);

        \draw (3,-3) to [short] (0,-3) node[circ] {};

        % tensão e corrente de magnetização
        \draw[->] (0,-0.25) -- (0,-2.75) node[midway,left] {$V_n$};
        \draw[->] (3,-0.25) -- (3,-0.30) node[midway,right] {$I_m$};
    \end{circuitikz}
    \caption{Transformador com o secundário em aberto.}
    \label{fig:ensaio-vazio-transformador}
\end{figure}

\noindent Os valores de $G_m$ e $B_m$ obtêm-se através das medidas da tensão aplicada $V_n$, corrente de magnetização $I_m$ e potência de perdas em vazio $P_0$:
$$
    G_m = \frac{P_0}{V_n^2} 
    \mkern48mu
    B_m = -\sqrt{\left(\frac{I_m}{V_n}\right)^2-G_m^2}
$$
\textbf{Nota:} A suscetância $B_m$ é negativa de forma a que a reatância respetiva seja indutiva.
%//==============================--@--==============================//%
\paragraph{Ensaio em Curto-Circuito}

\noindent Através de um ensaio em curto-circuito do transformador pode medir-se o módulo da impedância de $Z_t$ (normalmente designada por impedância de curto circuito $Z_{cc}$). Aplica-se um curto circuito a um dos enrolamentos e alimenta-se o outro com uma tensão reduzida (tensão de curto-circuito $V_{cc}$), de modo a que a corrente fique no seu valor nominal (de modo a não fritar o circuito).

\begin{figure}[H]
    \centering
    \begin{circuitikz}[>=stealth, scale=0.95]
        % Inductor
        \draw (0,0) to[R] (6,0);
        \node at (3,0.5) {$Z_t = Z_{cc}$};
        
        % Nodes
        \draw (0, 0) to[short, -*] (0,0);
        \draw (0,-2) to[short, -*] (0,-2);
        
        % Wire
        \draw (0, -2) to[short, -] (6,-2);
        \draw (6, 0) to[short, -] (6, -2);
        
        % Arrows
        \draw[->,thick](0, -0.2) to[short] (0,-1.8);
        \node at (-0.3,-1) {$V_{cc}$};
        
        \draw[->,thick](0.5, 0) to[short] (0.55,0) node[above] {$I_n$};
    \end{circuitikz}
    \caption{Transformador em curto-circuito.}
    \label{fig:ensaio-cc-transformador}
\end{figure}

\vspace{-1em}
\noindent\textbf{Nota:} O paralelo de duas resistências é aproximadamente a mais pequena. Como o ramo transversal possui uma impedância muito elevada, \underline{é possível desprezar o ramo e a respetiva corrente de magnetização}. %adoro-te mimo 
$$
    \left\{\begin{aligned}
        V_{cc} &= Z_{cc} I_n\\
        I_n &= 1.0\; \text{p.u.}
    \end{aligned}\right.\quad\rightarrow\quad
    \boxed{V_{cc} = Z_{cc}} 
$$
\noindent Para decompor a impedância de curto-circuito nas suas componentes resistiva e reativa, é preciso conhecer a potência $P_{cc}$:
$$
    R_t = \dfrac{P_{cc}}{I_n^2} = P_{cc}\qquad
    Z_{cc} = V_{cc} = \sqrt{R_t^2 + X_t^2}\qquad
    X_t = \sqrt{Z_t^2 - R_t^2}
$$

\begin{figure}[H]
    \centering
    \scalebox{0.925}{%
        \begin{tikzpicture}[scale=1.25, >=latex]
            % Draw real and imaginary axes
            \draw[->] (-0.25,0) -- (2.5,0) node[right] {Re};
            \draw[->] (0,-0.25) -- (0,2.5) node[above] {Im};
            
            \draw [line width=1.25pt] (0,0) -- (2,0) -- (2,2) -- cycle;
        
            \draw [->, line width=1.25pt] (0,0) -- (2.025,2.025);
            
            \node[below] at (1,0) {$R_t$};
            \node[right] at (2,1) {$X_t$};
            \node[above left] at (1,1) {$\mathbf{Z}$};
            \node[right] at (2.2,2.4) {$Z = \sqrt{R_t^2 + X_t^2}$};
            \node[left] at (-1.15,2) {};
        
            \draw (0.75,0) arc (0:45:0.75); 
        \end{tikzpicture}
    }
    \caption{Decomposição da impedância total equivalente.}
\end{figure}

%//==============================--@--==============================//%
\subsubsection{Configurações Especiais do Transformador}
\label{subsec:transformador-config-especiais}

%//==============================--@--==============================//%
\paragraph{Transformador com Três Enrolamentos}

Trata-se de um transformador com três enrolamentos à volta do mesmo núcleo como se representa na \hyperref[fig:transformador-3-enrolamentos]{Fig. 18}.

\begin{figure}[H]
    \centering
    \scalebox{0.725}{%
    \begin{circuitikz}[american]
        %% Configure circuitikz
        \ctikzset{inductor=cute}    
        \ctikzset{inductors/coils=4}
        \ctikzset{bipoles/resistor/height=0.25}
        \ctikzset{bipoles/resistor/width=0.5}
        \ctikzset{resistors/zigs=4}
    
        %% circuit
        \draw (0,0) to[short,*-] (1,0) to[generic,l^=$Z_1$,-*] (4,0);
        \draw (4,0) to[generic,l^=$Z_2$] (6,2) to[short,-*] (8,2);
        \draw (4,0) to[generic,l^=$Z_3$] (6,-2) to[short,-*] (7,-2);
        \draw (0,-4) to[short,*-*] (8,-4);

        %% optional
        \draw[gray] (4,0) to[short,*-] (4,-1.5) to[short] (3.5,-1.5) to[short] (4.5,-1.5);
        \draw[gray] (3.5,-1.5) to[R,l_=$G_m$] (3.5,-3);
        \draw[gray] (4.5,-1.5) to[L,l=$jB_m$] (4.5,-3);
        \draw[gray] (3.5,-3) to[short] (4.5,-3);
        \draw[gray] (4,-3) to[short,-*] (4,-4);

        %% labels and arrows
        \draw[->,>=stealth] (0,-0.25) -- (0,-3.75) node[midway,left] {$\mathbf{V}_1$};
        \draw[->,>=stealth] (7,-2.25) -- (7,-3.75) node[midway,left] {$\mathbf{V}_3$};
        \draw[->,>=stealth] (8,1.75) -- (8,-3.75) node[midway,right] {$\mathbf{V}_2$};
    \end{circuitikz}}

    \caption{Transformador com três enrolamentos.}
    \label{fig:transformador-3-enrolamentos}
\end{figure}

\vspace{-0.25em}
\noindent No caso de se pretender representar a admitância de magnetização, liga-se entre o nó fictício e o neutro.

As impedâncias do sistemas monofásico equivalente, podem ser obtidas através de três ensaios de curto-circuito (primário-secundário, primário-terciário e secundário-terciário), nos quais se medem $Z_{12}$, $Z_{13}$ e $Z_{23}$, respetivamente:
$$
    \begin{cases}
        Z_{12} = Z_1 + Z_2\\
        Z_{13} = Z_1 + Z_3\\
        Z_{23} = Z_2 + Z_3
    \end{cases}
$$
De onde resulta que
$$
    \left\{
    \begin{aligned}
        Z_1 &= \frac{Z_{12} + Z_{13} - Z_{23}}{2} \\
        Z_2 &= \frac{Z_{12} + Z_{23} - Z_{13}}{2} \\
        Z_3 &= \frac{Z_{13} + Z_{23} - Z_{12}}{2}
    \end{aligned}\right.
$$
%//==============================--@--==============================//%
\paragraph{Autotransformador}

Num autotransformador existe apenas um enrolamento, existindo portanto uma ligação elétrica e magnética como se \hyperref[fig:autotransformador]{representa abaixo}.

\vspace{0.25em}
\begin{minipage}[b]{0.35\linewidth}
    \begin{figure}[H]
        \centering
            \begin{circuitikz}
                %% Configure circuitikz
                \ctikzset{inductor=cute}    
                \ctikzset{inductors/coils=4}
                \ctikzset{bipoles/resistor/height=0.25}
                \ctikzset{bipoles/resistor/width=0.5}
                \ctikzset{resistors/zigs=4}
    
                % Circuito equivalente
                \draw (-1,0) to [short,*-] (2,0);
                \draw (2,0) to [L, -*] (2,-2);
                \draw (2,-2) to [L, *-*] (2,-4);
                \draw (2,-2) to [short,-*] (4,-2);
                \draw (-1,-4) to [short,*-*] (4,-4);
    
                % Labels and arrows
                \node[circ] at (1.75,-0.75) {}; \node[circ] at (1.75,-2.75) {};
                \draw[->,>=stealth] (-1,-0.25) -- (-1,-3.75) node[midway,left] {$\mathbf{V}^{'}_1$};
                \draw[->,>=stealth] (1.5,-0.25) -- (1.5,-1.75) node[midway,left] {$\mathbf{V}_1$};
                \draw[->,>=stealth] (4,-2.25) -- (4,-3.75) node[midway,right] {$\mathbf{V}_2$};
                \draw[->,>=stealth] (0.475,0) -- (0.525,0) node[midway,above] {$\mathbf{I}_1$};
                \draw[->,>=stealth] (2,-2.35) -- (2,-2.25) node[midway,right] {$\mathbf{I}_2$};
                \draw[->,>=stealth] (2.95,-2) -- (3.05,-2) node[midway,above] {$\mathbf{I}^{'}_2$};
            \end{circuitikz}
        \caption{Autotransformador.}
        \label{fig:autotransformador}
    \end{figure}
\end{minipage}\hfill
\begin{minipage}[b]{0.55\linewidth}
    \begin{mdframed}
        Sendo válidas as seguintes relações:
        $$
                \frac{V_1}{V_2} = \frac{N_1}{N_2} = m 
                \quad\land\quad
                \frac{I_1}{I_2} = \frac{N_2}{N_1} = \frac{1}{m}
        $$
        A potência aparente fornecida ao primário é dada por
        $$
            \begin{aligned}
                S^{'}_1 &= V^{'}_1 I_1 = (V_1 + V_2) I_1 = V_1 I_1 \frac{m+1}{m} \\
                &= S_1 \frac{m+1}{m}
            \end{aligned}
        $$
        E a potência cedida por este ao secundário é
        $$
            \begin{aligned}
                S^{'}_2 &= V_2 I^{'}_2 = V_2 (I_1 + I_2) = V_2 I_2 \frac{m+1}{m} \\
                &= S_2 \frac{m+1}{m}
            \end{aligned}
        $$
    \end{mdframed}
\end{minipage}

\vspace{0.5em}
\noindent Conclui-se que a potência nominal do autotransformador é mais elevada que a configuração com dois enrolamentos separados. Acresce ainda o facto de que tem um maior rendimento energético, uma vez que a corrente em cada enrolamento é a mesma nas duas configurações.

Esta vantagem resulta numa redução de custo, especialmente quando a relação de transformação se aproxima de 1 (normalmente usa-se quando a relação de transformação é menor que \texttt{3:1})\cite{paiva2005}.

As desvantagens do autotransformador incluem a falta de isolamento galvânico entre os enrolamentos e uma corrente de curto-circuito mais elevada uma vez que a impedância de curto-circuito é menor.

%//==============================--@--==============================//%
\clearpage
\subsubsection{Transformador Trifásico}
\label{subsec:transformador-trifasico}

Para sistemas trifásicos é usual utilizar um banco de transformadores (conjunto de três transformadores monofásicos), ou um transformador trifásico como se apresenta na \hyperref[fig:transformador-trifasico]{Fig. X}.

\vspace{0.25em}
\begin{minipage}[c]{0.375\linewidth}
    \begin{figure}[H]
        \centering
        \tikzset{
            terminal_a/.pic = {%
                \coordinate (-in) at (-3mm,0);
                \coordinate (-out) at (-3mm,-4.5mm);
        
                \path[fill] (-in) circle (2pt);
                \draw[thick] (-in)--(0,0)--++(0:0.95cm) arc[start angle=90, delta angle=-180, radius=.75mm]; 
                \draw[thick] (0,-1.5mm) arc[start angle=90, delta angle=180, radius=.75mm]--++(0:0.95cm) arc[start angle=90, delta angle=-180, radius=.75mm]; 
                \fill (-out) circle (2pt);
                \draw[thick] (-out) -- ++(0:3mm);
                },           
            terminal_b/.pic = {%
                \coordinate (-in) at (-3mm,0);
                \coordinate (-out) at (-3mm,-7.5mm);
        
                \path[fill] (-in) circle (2pt);
                \draw[thick] (-in)--(0,0)--++(0:0.95cm) arc[start angle=90, delta angle=-180, radius=.75mm]; 
                \draw[thick] (0,-1.5mm) arc[start angle=90, delta angle=180, radius=.75mm]--++(0:0.95cm) arc[start angle=90, delta angle=-180, radius=.75mm]; 
                \draw[thick] (0,-4.5mm) arc[start angle=90, delta angle=180, radius=.75mm]--++(0:0.95cm) arc[start angle=90, delta angle=-180, radius=.75mm]; 
                \fill (-out) circle (2pt);
                \draw[thick] (-out) -- ++(0:3mm);
                },
            field/.pic = {
                \draw[thick,-Stealth] (0,0) -- (90:7mm) node[above] {\tikzpictext};
                }
        }
    
        \scalebox{0.9}{
        \begin{tikzpicture}[>=stealth, scale=0.95]
            \draw (0,0) rectangle (7,5);
            \draw (1,1) rectangle (3,4);
            \draw (4,1) rectangle (6,4);
        
            \foreach \i/\j in {0/a,3/b,6/c}{
                \pic (upper-\j) at (\i,3.8) {terminal_a};
                \pic (lower-\j) at (\i,2) {terminal_b};
                \pic at ([xshift=5mm]\i,2.2) {field};
                \node at ([xshift=5mm, yshift=10mm]\i,2.2) {$\Phi_\j$};
                }
        \end{tikzpicture}}
        
        \caption{Transformador trifásico.}
        \label{fig:transformador-trifasico}
    \end{figure}
\end{minipage}\hfill
\begin{minipage}[c]{0.525\linewidth}
     Comparando as duas configurações, é natural que o transformador trifásico requeira menos materiais (ferro, cobre, etc) que o banco de três transformadores, sendo portanto mais económico. No entanto, perde em termos de fiabilidade, dado que é mais difícil de reparar (no banco de transformadores só se substitui um ponto de falha normalmente).

     Os fluxos magnéticos no núcleo também gozam da mesma simetria que as tensões simples, tendo uma soma nula a qualquer instante. Não é necessário um circuito magnético de retorno, à semelhança do que acontece às correntes nos sistemas trifásicos simétricos.
\end{minipage}

\vspace{0.75em}
\noindent Consideramos os seguintes tipos de ligação para os transformadores trifásicos: Y/Y, Y/$\Delta$, $\Delta$/Y e $\Delta$/$\Delta$, existindo ainda a ligação em \textit{zig-zag} que foge do âmbito da Unidade Curricular.

A relação de transformação de um transformador Y/Y e $\Delta$/$\Delta$ é um número real, uma vez que as tensões no primário e no secundário estão em fase. A polaridade dos enrolamentos é de \underline{extrema importância} em transformadores trifásicos (assinalado com uma bola preta nas \hyperref[fig:tipos-ligacao-transformador-trifasico]{figuras abaixo}).

\begin{figure}[H]
    \centering

    %% Configure circuitikz
    \ctikzset{inductor=cute}    
    \ctikzset{inductors/coils=4}
    
    \begin{subfigure}[b]{0.5\linewidth}
        \centering
        \scalebox{0.6}{%
        \begin{circuitikz}
            %% Left inductors
            \draw (0,0) to [L, *-*] (0,2);
            \draw (0,0) to [L, *-*] (2,-2);
            \draw (0,0) to [L, *-*] (-2,-2);

            \node[below=1mm] at (0,0) {$n$}; % neutral
            \node[above] at (0,2) {$a$};
            \node[left] at (-2,-2) {$c$};
            \node[right] at (2,-2) {$b$};
        
            %% Right inductors
            \draw (7,0) to [L, *-*] (7,2);
            \draw (7,0) to [L, *-*] (9,-2);
            \draw (7,0) to [L, *-*] (5,-2);

            \node[below=1mm] at (7,0) {$n$}; % neutral
            \node[above] at (7,2) {$a$};
            \node[left] at (5,-2) {$c$};
            \node[right] at (9,-2) {$b$};
        \end{circuitikz}}
        \caption{Configuração Y/Y}
    \end{subfigure}%
    \begin{subfigure}[b]{0.5\linewidth}
        \centering
        \scalebox{0.6}{%
        \begin{circuitikz}
            % Left inductors
            \draw (0,0) to [L, *-*] (0,2);
            \draw (0,0) to [L, *-*] (2,-2);
            \draw (0,0) to [L, *-*] (-2,-2);

            \node[below=1mm] at (0,0) {$n$}; % neutral
            \node[above] at (0,2) {$a$};
            \node[left] at (-2,-2) {$b$};
            \node[right] at (2,-2) {$c$};
        
            % Right inductors
            \draw (7,0) to[L, *-*] (5,2);
            \draw (7,0) to[L, *-*] (5,-2);
            \draw (5,2) to[L, *-*] (5,-2);

            \node[above] at (5,2) {$a$};
            \node[right] at (7,0) {$b$};
            \node[below] at (5,-2) {$c$};
        \end{circuitikz}}
        \caption{Configuração Y/$\Delta$}
    \end{subfigure}

    \vspace{0.75em}
    
    \begin{subfigure}[b]{0.5\linewidth}
        \centering
        \scalebox{0.6}{%
        \begin{circuitikz}
            % Left inductors
            \draw (0,0) to[L, *-*] (2,2);
            \draw (0,0) to[L, *-*] (2,-2);
            \draw (2,2) to[L, *-*] (2,-2);

            \node[above] at (2,2) {$a$};
            \node[left] at (0,0) {$c$};
            \node[below] at (2,-2) {$b$};

            % Right inductors
            \draw (7,0) to [L, *-*] (7,2);
            \draw (7,0) to [L, *-*] (9,-2);
            \draw (7,0) to [L, *-*] (5,-2);

            \node[below=1mm] at (7,0) {$n$}; % neutral
            \node[above] at (7,2) {$a$};
            \node[left] at (5,-2) {$c$};
            \node[right] at (9,-2) {$b$};
        \end{circuitikz}}
        \caption{Configuração $\Delta$/Y}
    \end{subfigure}%
    \begin{subfigure}[b]{0.5\linewidth}
        \centering
        \scalebox{0.6}{%
        \begin{circuitikz}
            % Left inductors
            \draw (0,0) to[L, *-*] (2,2);
            \draw (0,0) to[L, *-*] (2,-2);
            \draw (2,2) to[L, *-*] (2,-2);

            \node[above] at (2,2) {$a$};
            \node[left] at (0,0) {$c$};
            \node[below] at (2,-2) {$b$};
        
            % Right inductors
            \draw (7,0) to[L, *-*] (5,2);
            \draw (7,0) to[L, *-*] (5,-2);
            \draw (5,2) to[L, *-*] (5,-2);

            \node[above] at (5,2) {$a$};
            \node[right] at (7,0) {$b$};
            \node[below] at (5,-2) {$c$};
        \end{circuitikz}}
        \caption{Configuração $\Delta/\Delta$}
    \end{subfigure}
    \caption{Tipos de ligação de transformadores trifásicos.}
    \label{fig:tipos-ligacao-transformador-trifasico}
\end{figure}

\noindent Em transformadores Y/$\Delta$ ou $\Delta$/Y, existe uma desfasagem entre as tensões no primário e no secundário, o que leva a uma \underline{relação de transformação complexa}.

\begin{mdframed}
    Tomando como referência a configuração Y/$\Delta$, representada na \hyperref[fig:tipos-ligacao-transformador-trifasico]{Fig. 19 (b)}, observa-se que
    $$
        \mathbf{V}^{ac}_2 = \mathbf{V}^{a}_2 - \mathbf{V}^{c}_2 = \mathbf{V}^{a}_2 (1-e^{j120^{\circ}}) = \sqrt{3} \mathbf{V}^{a}_2\, e^{-j30^{\circ}}
    $$
    Sendo $N_1$ o número de espiras do primário e $N_2$ o número de espiras do secundário, temos 
    $$
        \mathbf{V}^{a}_{1} = \frac{N_1}{N_2} \mathbf{V}^{ac}_{2} = \sqrt{3} \frac{N_1}{N_2} \mathbf{V}^{a}_2\, e^{-j30^{\circ}} 
        \implies
        m = \frac{\mathbf{V}^{a}_1}{\mathbf{V}^{a}_2} = \sqrt{3} \frac{N_1}{N_2} e^{-j30^{\circ}} \iff \boxed{\mathbf{V}^{a}_1 = m\, \mathbf{V}^{a}_2}
    $$
    Concluímos que a tensão simples secundária está $30^{\circ}$ em avanço face à correspondente tensão primária.

    De forma análoga para as correntes, em que $\mathbf{I}^{ac}_2 = (N_1/N_2) \mathbf{I}^{a}_1$, e $\mathbf{I}^{ba}_2 = (N_1/N_2) \mathbf{I}^{b}_1 = (N_1/N_2) \mathbf{I}^{a}_1\, e^{-j120^{\circ}}$
    $$
        \mathbf{I}^{a}_2 = \mathbf{I}^{ac}_2 - \mathbf{I}^{ba}_2 = \mathbf{I}^{a}_1 \frac{N_1}{N_2} (1-e^{-j120^{\circ}}) = \sqrt{3} \frac{N_1}{N_2} \mathbf{I}^{a}_1\, e^{j30^{\circ}}
        \implies
        \boxed{\mathbf{I}^{a}_2 = m^*\, \mathbf{I}^{a}_1}
    $$
    Combinando as expressões das tensões e das correntes chegamos a:
    $$
        \boxed{3 \mathbf{V}^{a}_1\mathbf{I}^{a^*}_1 = 3 \mathbf{V}^{a}_2\mathbf{I}^{a^*}_2}
    $$
\end{mdframed}

%//==============================--@--==============================//%
       %//==============================--@--==============================//%
\subsection{Máquina Assíncrona}

\noindent Um motor assíncrono (também designado por \textit{motor de indução}) recebe energia da rede elétrica e fornece energia mecânica a uma carga (também pode funcionar como gerador): é um \underline{conversor eletromecânico}.

\begin{figure}[H]
    \centering
    \begin{tikzpicture}[very thick,sc/.style={circle,draw,inner sep=1.5pt},scale=0.6]
         \begin{scope}
              \clip (4,0) arc(0:-360:4) (0,0) circle[radius=5cm];
              \fill[gray!50,even odd rule] circle[radius=5cm] circle[radius=4cm] 
               foreach \X in {1,...,6} {(\X*60:4) circle[radius=0.5cm]};
         \end{scope}
         
         \foreach \X [count=\Y starting from 0] in {A,B,C,A,B,C} { 
             \path(180-60*\Y:4) node[circle,draw, scale=0.6] (n\Y) {\X};
         }
         
         \foreach \Y in {0,...,4}{
            \draw[dashed,-{Stealth[bend]}] (180-72*\Y-5:3) arc(180-72*\Y-5:180-72*\Y-67:3);
            }
            
         \fill[gray!70] circle[radius=1.8cm];   
         \foreach \X in {1,...,20}{
            \draw (\X*18:2) circle[radius=0.12cm];
         }
         
         \draw (n1) -- (120:5.8) coordinate (aux) -- (-6.7,0|-aux)  node[sc,left] (c1){};
         \draw (n3) -- (0:5.8) |- ([yshift=8mm]c1.east)  node[sc,left] (c3){};
         \draw (n5) -- (-120:5.8) -| ([yshift=-8mm,xshift=5mm]c1.east)  
         -- ++(-0.5,0)node[sc,left] (c5){};
         \draw[decorate,decoration={brace,mirror,raise=2pt}] (c3.north west) -- (c5.south west)
          node[midway,left=3pt,align=right]{3 Phase\\ input};
         \draw (n0) -| (-5.8,-5.8) node[sc,fill] {} -- ++ (-1,0)
          node[left,sc,label=left:Common] (Common){};
         \draw (n4) -- (-60:5.6) coordinate(aux) -- (aux|-Common) node[sc,fill]{};
         \draw (n2) -- (60:5.6) coordinate(aux) -- (aux-|6,0) |- (Common);
    \end{tikzpicture}

    \caption{Motor de indução trifásico.}
\end{figure}

\noindent A máquina assíncrona trifásica é constituída por um estator (camada exterior) e um rotor (camada interior). Da aplicação de um sistema trifásico de tensões ao enrolamento do estator, resulta no entreferro um fluxo magnético girante, o qual induz no enrolamento do rotor uma f.e.m. Uma vez que o rotor está em curto circuito (rotor em gaiola) ou fechado através de circuito exterior (rotor bobinado), esta f.e.m. dá origem a correntes que circulam no rotor, \underline{produzindo um binário motor}.

\begin{minipage}[b]{0.275\linewidth}
   \begin{figure}[H]
        \centering
        \begin{tikzpicture}[scale=2, transform shape]
        
            \def\samples{200}
            
            % Draw sine waveforms
            \draw[domain=pi:2*pi, samples=\samples] plot (\x,{sin(\x r)});
            \draw[domain=pi:2*pi, samples=\samples] plot (\x,{sin((\x - 2*pi/3) r)});
            \draw[domain=pi:2*pi, samples=\samples] plot (\x,{sin((\x + 2*pi/3) r)});
            
            % Function to draw rotating field representation
            \newcommand\rotatingfield[3]{
                \begin{scope}[shift={(#1,#2)}]
                    \clip (0.4,0) arc(0:-360:0.4) (0,0) circle[radius=0.5cm];
                    \fill[gray!50,even odd rule] circle[radius=0.5cm] circle[radius=0.4cm] 
                    foreach \X in {1,...,6} {(\X*60:0.4) circle[radius=0.07cm]};
            
                \end{scope}
                \begin{scope}[shift={(#1,#2)}]
                     \foreach \X [count=\Y starting from 0] in {A,B,C,A,B,C} { 
                         \path(180-60*\Y:0.4) node[circle,draw,scale=0.15] (n\Y) {\X};
                    }
                    \fill[gray!70] circle[radius=0.16cm];   
                    \foreach \X in {1,...,20}
                    {\draw (\X*18:0.2) circle[radius=0.018cm];}
            
                    % Draw an arrow at the specified angle
                    \draw[blue,-{Stealth[length=3.5pt]}] (0,0) -- (#3:0.14);
                \end{scope}
            }
            
            % Draw vertical dotted lines at intersections and motor direction rotating fields
            \draw[dotted, thin] (5.7596,-1.1) -- (5.7596,1.5);
            \rotatingfield{5.7596}{2}{-60} % Example angle of 45 degrees
            \draw[dotted, thin] (3.6652,-1.1) -- (3.6652,1.5);
            \rotatingfield{3.6652}{2}{60} % Example angle of -30 degrees
            \draw[dotted, thin] (4.7124,-1.1) -- (4.7124,1.5);
            \rotatingfield{4.7124}{2}{0} % Example angle of 90 degrees
            
            % Draw y-axis
            \draw[-{Stealth[length=2.0pt]}] (pi,-1.1) -- (pi,1.1);
            \draw (pi-0.15,0.85) node[circle,draw,scale=0.2] {C};
            \draw (pi-0.15,-0.05) node[circle,draw,scale=0.2] {A};
            \draw (pi-0.15,-0.85) node[circle,draw,scale=0.2] {B};
        
        \end{tikzpicture}
    \end{figure} 
\end{minipage}\hfill
\begin{minipage}[b]{0.53\linewidth}
 \noindent \textbf{No motor de indução:}
 \begin{itemize}
     \item A energia AC fornecida ao estator do motor cria um campo magnético que roda em sincronismo com as oscilações AC.
     \item O rotor gira a uma velocidade um pouco mais lenta do que o campo do estator. O campo magnético do estator está, portanto, a mudar ou a rodar relativamente ao rotor.
     \item Isto induz uma corrente oposta no rotor, que é efetivamente a segunda bobina do motor.
     \item O fluxo magnético rotativo induz correntes nas bobinas do rotor, de forma semelhante às correntes induzidas nas bobinas secundárias de um transformador.
 \end{itemize}
\end{minipage}

\vspace{1em}
\noindent Estando o motor em repouso, as correntes no rotor têm uma frequência igual à da tensão de alimentação; à medida que o rotor acelera, por ação do binário motor, aquela frequência vai diminuindo. 

Se o motor estiver em vazio, a frequência e a amplitude das correntes no rotor são muito próximas de zero. Estando o motor a acionar um carga mecânica que oferece um binário resistente, a frequência e a resultante amplitude das \textit{correntes rotóricas} terão um valor correspondente ao binário motor necessário para estabilizar a marcha da máquina.

\begin{mdframed}
    Em termos de \textbf{balanço energético}, a energia recebida da rede elétrica é transferida para o rotor por efeito indutivo, deduzida das perdas no ferro do estator e no cobre do enrolamento respetivo. Subtraindo as perdas no rotor e as perdas mecânicas, obtém-se a \underline{potência mecânica final fornecida à carga}.
\end{mdframed}

%//==============================--@--==============================//%
\subsubsection{Modelo Matemático e Esquema Equivalente}

Definimos a \textit{velocidade síncrona} do rotor como:
$$
    n_s = \frac{60 f}{p}\; [\text{rpm}] \implies \omega_s = n_s \frac{2\pi}{60} = \frac{2\pi f}{p}\; [\text{rad/s}]
$$
em que $f$ é a frequência da tensão de alimentação e $p$ o número de pares de pólos do enrolamento do estator.

A diferença entre a velocidade síncrona e a velocidade de rotação do rotor ($n_r$), expressa em em p.u. (ou percentagem) na base da primeira, designa-se por \textit{escorregamento} (ou \textit{deslizamento}):
$$
    s = \frac{n_s - n_r}{n_s} = \frac{\omega_s - \omega_r}{\omega_s}
$$
onde $\omega_s$ e $\omega_r$ são as velocidades angulares respetivas. O escorregamento tem um valor muito baixo em vazio, e vai aumentando à medida que a carga aumenta.

\vspace{0.5em}
\hrule
\vspace{0.5em}
\noindent Podemos modelar a máquina síncrona através de um esquema equivalente em T, à semelhança do que se faz para o transformador (ver Apêndice X para mais detalhes):

\vspace{-0.75em}
\begin{figure}[H]
    \centering
    \scalebox{0.915}{%
    \begin{circuitikz}[american,>=latex]
        \ctikzset{inductor=cute}    
        \ctikzset{inductors/coils=4}
        \ctikzset{bipoles/resistor/height=0.25}
        \ctikzset{bipoles/resistor/width=0.5}
        \ctikzset{resistors/zigs=4}
        
        % circuit
        \draw 
            (0,0) to[short,*-] 
            (1,0) to[R=$R_s$]
            (2.5,0) to[L=$jX_s$] 
            (4,0) to[short,-*] (5,0)
            ;

        \draw
            (5,0) to[short,-*] 
            (5,-1) to[short] (4.5,-1) to[short] (5.5,-1) 
            ;

        \draw
            (4.5,-1) to[R,l_=$G_m$] (4.5,-2.5)
            ;
            
        \draw
            (5.5,-1) to[L=$jB_m$] (5.5,-2.5)
            ;

        \draw
            (5,-2.5) to[short] (4.5,-2.5) to[short] (5.5,-2.5)
            to[short] (5,-2.5) to[short,*-] (5,-3.5)
            ;

        \draw
            (5,0) to[short] 
            (6,0) to[short] %to[R=$R_r$]
            (7.5,0) to[L=$jX_r$]
            (9,0) to[short] 
            (10,0) to[R,l_=$\displaystyle \frac{R_r}{s}$,*-] (10,-2.5)
            ;

        \draw (0,-3.5) to[short,*-] (11,-3.5) to[short] (11, -1.25);
        \draw[->] (11,-1.25) -- (10.3,-1.25);

        % labels and arrows
        \draw[->] (0,-0.25) -- (0,-3.25) node[midway,left] {$\mathbf{V}_s$};
        \draw[->] (4.45,0) -- (4.5,0) node[midway,above] {$\mathbf{I}_s$};
        \draw[->] (5,-0.5) -- (5,-0.6) node[midway,right] {$\mathbf{I}_m$};
        \draw[->] (5.7,0) -- (5.75,0) node[midway,above] {$\mathbf{I}_r$};

        \draw[->] (7.25,-0.25) -- (7.25,-3.25) node[midway,right] {$\mathbf{E}_s$};
        
    \end{circuitikz}}
    \caption{Esquema equivalente em T da máquina assíncrona.}
    \label{fig:esquema-T-maquina-assincrona}
\end{figure}

\noindent A potência transferida para o rotor corresponde à potência consumida na resistência fictícia ($R_r/s$), que é igual à potência fornecida pela rede menos as perdas no estator e no circuito magnético:
$$
    P_r = 3\frac{R_r}{s} I^2_r   
$$
A potência mecânica é igual à potência $P_r$ menos as perdas no rotor:
$$
    P_M = P_r - 3R_r I^2_r = 3 \frac{1-s}{s} R_r I^2_r
$$
O esquema equivalente pode ser modificado de forma a modelar a carga mecânica:

\begin{minipage}[c]{0.525\linewidth}
\begin{figure}[H]
    \centering
    \scalebox{0.915}{%
    \begin{circuitikz}[american,>=latex]
        \ctikzset{inductor=cute}    
        \ctikzset{inductors/coils=4}
        \ctikzset{bipoles/resistor/height=0.25}
        \ctikzset{bipoles/resistor/width=0.5}
        \ctikzset{resistors/zigs=4}
        
        % circuit
        \draw 
            (2.5,0) to[short,*-*] (5,0)
            ;

        \draw
            (5,0) to[short,-*] 
            (5,-1) to[short] (4.5,-1) to[short] (5.5,-1) 
            ;

        \draw
            (4.5,-1) to[R,l_=$G_m$] (4.5,-2.5)
            ;
            
        \draw
            (5.5,-1) to[L=$jB_m$] (5.5,-2.5)
            ;

        \draw
            (5,-2.5) to[short] (4.5,-2.5) to[short] (5.5,-2.5)
            to[short] (5,-2.5) to[short,*-] (5,-3.5)
            ;

        \draw
            (5,0) to[short] 
            (6,0) to[R, l=$R_s+R_r$]
            (8,0) to[L, l=$j(X_s+X_r)$]
            (9.5,0) to[short] 
            (10.5,0) to[R,l_=$\displaystyle \frac{1-s}{s}R_r$,*-] (10.5,-2.5)
            ;

        \draw (2.5,-3.5) to[short,*-] (11.5,-3.5) to[short] (11.5, -1.25);
        \draw[->] (11.5,-1.25) -- (10.75,-1.25);

        % labels and arrows
        \draw[->] (2.5,-0.25) -- (2.5,-3.25) node[midway,left] {$\mathbf{V}_s$};
        \draw[->] (4.45,0) -- (4.5,0) node[midway,above] {$\mathbf{I}_s$};
        \draw[->] (5,-0.5) -- (5,-0.6) node[midway,right] {$\mathbf{I}_m$};
        \draw[->] (5.7,0) -- (5.75,0) node[midway,above] {$\mathbf{I}_r$};
        
    \end{circuitikz}}
    \caption{Esquema equivalente em L da máquina assíncrona.}
    \label{fig:esquema-L-maquina-assincrona}
\end{figure}
\end{minipage}\hfill
\begin{minipage}[c]{0.4\linewidth}
    Podemos deslocar o ramo longitudinal de modo a obter o esquema equivalente em L. No entanto, $\mathbf{I}_m$ é maior neste contexto (relativamente ao transformador), o que torna o esquema numa aproximação mais grosseira.
    $$
        \left\{
        \begin{aligned}
            \mathbf{I}_r &= \frac{\mathbf{V}_s}{R_s + R_r/s + j(X_s + X_r)} \\
            \mathbf{I}_m &= (G_m + jB_m) \mathbf{V}_s
        \end{aligned}\right.
    $$
    $$
        \mathbf{I}_s = \mathbf{I}_m + \mathbf{I}_r
    $$
\end{minipage}

\vspace{0.5em}
\noindent A potência perdida é dada por
$$
    \therefore P_s = 3G_m V^2_s + 3 \left(R_s + \frac{R_r}{s} \right) I^2_r
$$
Donde resulta o rendimento:
$$
    \eta = \frac{P_M}{P_s} = \frac{(1-s)/s \cdot R_r I^2_r}{3G_m V^2_s + 3(R_s + R_r/s) I^2_r}
$$

\begin{mdframed}
    \noindent \textbf{Nota}: O motor assíncrono representa uma carga indutiva para a rede de alimentação com
    $$
        Q_s = -3B_m V^2_s + 3(X_s + X_r) I^2_r
    $$
\end{mdframed}

%//==============================--@--==============================//%
\subsubsection{Binário e Característica Binário-Velocidade}

Para além da potência, interessa calcular o binário:
$$
    T = \frac{P_M}{\omega_r} = \frac{P_M}{\omega_s (1-s)} = \frac{3 R_r I^2_r}{s \omega_s}
$$
Como $I^2_r = V^2_s/[(R_s + R_r)^2 + (X_s + X_r)^2]$, temos que:
$$
    T = \frac{3 V^2_s}{\omega_s} \frac{R_r}{(R_s + R_r)^2 + (X_s + X_r)^2}
$$
A região de funcionamento como motor corresponde a $s>0$ ou seja, a velocidade de rotação é menor que a do sincronismo; o funcionamento como gerador é caracterizado por $s<0$, uma vez que neste caso $\omega_r>\omega_s$.

\vspace{-0.5em}
\begin{minipage}[c]{.4\linewidth}
\begin{figure}[H]
    \centering
    \scalebox{0.95}{%
        \begin{tikzpicture}[>=stealth]
            \begin{axis}[
                axis y line=left,
                axis x line=middle,
                xmax=3.7, 
                ymax=3, 
                ymin=-3,
                legend pos=north west,
                ticks=none,
                clip mode=individual
            ]
        
            % The torque-slip curve
            \addplot[black, thick, domain=-3:3.6, samples=200] {-x/(0.4*x^2 + 0.1)};
        
            % Vertical dashed line 
            \draw[dashed, black] (axis cs:-0.5,0) -- (axis cs:-0.5,2.7);
            \draw[dashed, black] (axis cs:0,-2.7) -- (axis cs:0,2.7);
        
            % horizontal dotted line 
            \draw[dotted, black] (axis cs:-3,2.5) -- (axis cs:-0.5,2.5);
            
            % Arrows indicating stable and unstable regions
            \draw[<->, thin] (axis cs:-3,2.7) -- (axis cs:-0.5,2.7) node[midway, above, font=\tiny] {Instável};
            \draw[<->, thin] (axis cs:-0.5,2.7) -- (axis cs:0,2.7) node[midway, above, font=\tiny] {Estável};
            
            % Regions
            \draw[<->, thin] (axis cs:-3,-2.7) -- (axis cs:0,-2.7) node[midway, above, font=\tiny] {Motor};
            \draw[<->, thin] (axis cs:0,-2.7) -- (axis cs:3.6,-2.7) node[midway, above, font=\tiny] {Gerador};
        
            % Manual tick marks
            \draw[black] (axis cs:-3.1,2.5) -- (axis cs:-2.9,2.5) node[left=0.9mm,font=\tiny] {$T_M$};
            \draw[black] (axis cs:3,-0.05) -- (axis cs:3,0.05);
            
            % Labels
            \node[yshift=-1mm,xshift=1mm,font=\tiny] at (-3,0) {$0$};
            \node[yshift=1mm,xshift=1mm,font=\tiny] at (-3,0) {$1$};
            
            \node[yshift=-1mm,xshift=2mm,font=\tiny] at (0,0) {$\omega_s$};
            \node[yshift=1mm,xshift=1mm,font=\tiny] at (0,0) {$0$};
        
            \node[yshift=-1.4mm,xshift=0mm,font=\tiny] at (3,0) {$2\omega_s$};
            \node[yshift=1.4mm,xshift=0mm,font=\tiny] at (3,0) {$-1$};
        
            \node[yshift=-1.5mm,xshift=-0.1mm,font=\tiny] at (3.65,0) {$\omega$};
            \node[yshift=1.5mm,font=\tiny] at (3.65,0) {$s$};
        
            % rando arrows
            \draw[->, thin] (axis cs:-3.5,0.2) -- (axis cs:-3.5,1.6) node[above, font=\tiny] {Torque};
            \draw[->, thin] (axis cs:-1.5,-0.5) -- (axis cs:-0.2,-0.5) node[left=13mm, font=\tiny] {Velocidade};
            \draw[->, thin] (axis cs:-1.8,-0.8) -- (axis cs:-2.7,-0.8) node[right=9mm, font=\tiny] {Escorregamento};
            \end{axis}
        \end{tikzpicture}
    }
    \caption{Característica binário-velocidade}
\end{figure}
\end{minipage}\hfill
\begin{minipage}[c]{.5\linewidth}
    O binário máximo pode calcular-se após uma simples análise da equação em ordem a $s$:
    $$
        \begin{aligned}
            s_{T_{max}} &= \frac{R_r}{\sqrt{R_s^2 + (X_s + X_r)^2}} \\
            T_{max} &= \frac{3V^2_s}{2R_r} \frac{1}{R_s + \sqrt{R_s^2 + (X_s + X_r)^2}}
        \end{aligned}
    $$
\end{minipage}

\vspace{0.5em}
\noindent O binário de arranque, com $s=1$, é dado por:
$$
    T_{arr} = \frac{3V^2_s}{\omega_s} \frac{R_r}{(R_s + R_r)^2 + (X_s + X_r)^2}
$$
E a corrente de arranque é
$$
    \mathbf{I}_{arr} = \frac{\mathbf{V}_s}{R_s + R_r + j(X_s + X_r)}
$$
Queremos reduzir esta corrente, quando $\omega_r = 0$ e $s=1$, especialmente em motores de potência elevada, mas isto resulta na diminuição de $T_{arr}$, o que se pode revelar problemático para certas cargas mecânicas.
%//==============================--@--==============================//%
\subsubsection{Representação em Valores p.u.}

No contexto da máquina assíncrona, normalmente especifica-se a tensão e corrente nominais, e a potência mecânica. Tomam-se os valores da tensão e da corrente nominais como base, calcula-se a potência aparente de base:
$$
    S_b = \sqrt{3} V_b I_b
$$
Tomando como valor base para a velocidade angular:
$$
    \omega_b = \omega_s = \frac{2\pi f}{p} \implies \omega_r = 1-s
$$
As restantes equações ficam
$$
        P_r = \frac{R_r}{s} I^2_r  \qquad\rightarrow\qquad P_M = P_r - R_r I^2_r = \frac{1-s}{s} R_r I^2_r
$$
$$
    \left\{
    \begin{aligned}
        P_s &= G_m V^2_s + (R_s + R_r/s) I^2_r \\
        Q_s &= -B_m V^2_s + (X_s + X_r) I^2_r
    \end{aligned}\right.
$$

\vspace*{0.5em}
$$
        T = \frac{P_M}{\omega_r} = \frac{P_M}{1-s} = V^2_s \frac{R_r/s}{(R_s + R_r/s)^2 + (X_s + X_r)^2}
$$
$$
    \left\{
    \begin{aligned}
        T_{arr} &= V^2_s \frac{R_r}{(R_s + R_r)^2 + (X_s + X_r)^2} \\
        T_{max} &= \frac{V^2_s}{2} \frac{1}{R_s + \sqrt{R^2_s + (X_s + X_r)^2}}
    \end{aligned}\right.
$$

%//==============================--@--==============================//%
\clearpage
\subsubsection{Análise da Máquina Assíncrona}
%//==============================--@--==============================//%
\paragraph{Ensaio em Vazio}

Semelhante ao ensaio que se faz no transformador para determinar os parâmetros transversais. Aplica-se a tensão nominal ao estator da máquina sem qualquer carga mecânica no veio ($s \approx 0$).

\begin{figure}[H]
    \centering
    \ctikzset{bipoles/resistor/height=0.20}
    \ctikzset{bipoles/resistor/width=0.5}
    \begin{circuitikz}[=>stealth,american]
        % circuito
        \draw (0,0) to [short, *-] (3,0);

        \draw (3,-0.5) to [short] (3,0);
        \draw (2.5,-0.5) to [short] (3.5,-0.5);
        \draw (2.5,-0.5) to [R, l_=$G_m$] (2.5,-2.5);
        \draw (3.5,-0.5) to [cute inductor, l=$jB_m$] (3.5,-2.5);
        \draw (2.5,-2.5) to [short] (3.5,-2.5);
        \draw (3,-2.5) to [short] (3,-3);

        \draw (3,-3) to [short] (0,-3) node[circ] {};

        % tensão e corrente de magnetização
        \draw[->] (0,-0.25) -- (0,-2.75) node[midway,left] {$V_n$};
        \draw[->] (3,-0.25) -- (3,-0.30) node[midway,right] {$I_m$};
    \end{circuitikz}
    \caption{Esquema equivalente do ensaio em vazio.}
    \label{fig:ensaio-vazio-maq-async}
\end{figure}

\vspace{-0.5em}
\noindent Os valores de $G_m$ e $B_m$ obtêm-se igualmente das medidas da tensão aplicada $V_n$, corrente de magnetização $I_m$ e potência de perdas em vazio $P_0$:
$$
    G_m = \frac{P_0}{V_n^2} 
    \mkern48mu
    B_m = -\sqrt{\left(\frac{I_m}{V_n}\right)^2-G_m^2}
$$

%//==============================--@--==============================//%
\paragraph{Ensaio com Rotor Bloqueado}

Neste ensaio o rotor é mantido parado ($s \approx 1$), o que resulta numa resistência $R (1-s)/s$ nula. O esquema equivalente é igual ao do ensaio em curto-circuito no transformador. Aplica-se uma tensão reduzida ao estator até a corrente atingir o valor nominal.
$$
    \left\{\begin{aligned}
        V_{cc} &= Z_{cc} I_n\\
        I_n &= 1.0\; \text{p.u.}
    \end{aligned}\right.\quad\rightarrow\quad
    \boxed{V_{cc} = Z_{cc}} 
$$
\noindent Para decompor a impedância de curto-circuito nas suas componentes resistiva e reativa, é preciso conhecer a potência $P_{cc}$:
$$
    R_t = \dfrac{P_{cc}}{I_n^2} = P_{cc}\qquad
    Z_{cc} = V_{cc} = \sqrt{R_t^2 + X_t^2}\qquad
    X_t = \sqrt{Z_t^2 - R_t^2}
$$
%//==============================--@--==============================//%
\subsubsection{Funcionamento como Gerador}

A máquina assíncrona, além das suas funções habituais, também pode operar como gerador, sendo especialmente utilizada em centrais de baixa potência alimentadas por fontes renováveis. Ao contrário da máquina síncrona, que possui um sistema de excitação próprio, a máquina assíncrona requer uma corrente de magnetização do exterior, normalmente fornecida pela rede elétrica.

Quando funciona como gerador, a máquina assíncrona tem certas características específicas. Recebe energia mecânica (de um motor, por exemplo) e entrega energia elétrica à rede. No entanto, mesmo fornecendo energia ativa, continua a absorver energia reativa, o que exige compensação através de uma bateria de condensadores. Dependendo das condições, esta bateria pode ser ajustada para que o gerador assíncrono também forneça energia reativa.


\begin{mdframed}
    \vspace{1.5cm}
    \hfil (Adicionar imagem)
    \vspace{1.5cm}
\end{mdframed}


\noindent Um ponto interessante é que um gerador assíncrono pode auto-excitar-se quando está em vazio e ligado a um condensador, dependendo da capacitância do condensador. Na prática, a ligação deste gerador à rede pode ser feita de duas formas distintas: rodando a uma velocidade próxima da nominal ou utilizando uma bateria de condensadores de valor apropriado. A segunda opção evita sobre-correntes indesejadas.

%//==============================--@--==============================//%

    \clearpage
    \section{Máquina Síncrona}%
       %//==============================--@--==============================//%

A maior parte da energia elétrica que consumimos provém de \textit{geradores síncronos} ou \textit{alternadores trifásicos}, fundamentais nos Sistemas de Energia Elétrica. Estas máquinas transformam energia mecânica em elétrica através da lei da indução eletromagnética de Faraday (conversores mecanoelétricos).

O termo \textit{síncrona} refere-se à capacidade destas máquinas operarem a uma velocidade e frequência constantes, em sintonia com outras ligadas à mesma rede. Na prática, um gerador pode receber energia mecânica de várias fontes, como turbinas hidráulicas ou a vapor, e converter essa energia em eletricidade com eficiência elevada. Curiosamente, estas máquinas também podem funcionar como motores, absorvendo energia elétrica e fornecendo energia mecânica (funcionamento como \textit{motor síncrono}).

%//==============================--@--==============================//%
\subsection{Princípio de Funcionamento}

A máquina síncrona é constituída por uma massa metálica fixa (\textit{estator}) na qual está instalado o \textit{enrolamento induzido}, e por uma massa metálica rotativa (\textit{rotor}) no qual está bobinado o \textit{enrolamento indutor} (ou de \textit{excitação}):

O enrolamento indutor é percorrido por uma corrente contínua, que dá origem a um fluxo magnético que se fecha através do entreferro e do estator. Uma vez que o rotor, acionado pela máquina motriz, roda com velocidade constante, cria-se no entreferro um fluxo magnético girante.

O enrolamento do estator é constituído por bobinas, alojadas em cavas; as bobinas correspondente a uma fase são colocadas em cavas diretamente opostas. De acordo com a Lei de Indução, o fluxo magnético girante produz uma tensão que origina uma corrente num circuito externo ligado aos respetivos terminais. Estes enrolamentos são desfasados fisicamente por $120^{\circ}$ para que com a rotação uniforme do rotor sejam produzidas tensões desfasadas de $120^{\circ}$ no tempo, constituindo um sistema trifásico simétrico.

\vspace{0.5em}\hrule\vspace{0.5em}%

\noindent Para uma máquina com um par de pólos, a frequência da tensão induzida em ciclos por segundo (Hz) iguala a velocidade do rotor em rotações por segundo. Assim, para $50\,$Hz a velocidade de rotação deverá ser de $3000\,$ rpm. No entanto, é costume as máquinas terem mais pares de pólos (magnéticos), por exemplo 4:

\vspace{1.5em}
\noindent%
\begin{minipage}[c]{.45\linewidth}

    Cada fase é um par de enrolamentos (4 cavas): $a_1 a'_1$ e $a_2 a'_2$, $b_1 b'_1$ e $b_2 b'_2$, $c_1 c'_1$ e $c_2 c'_2$.

    \vspace{1em}
    Em cada instante são induzidas tensões iguais nos dois enrolamentos de cada fase, as quais se somam por estarem em série.

    \begin{figure}[H]
        \centering
        \includegraphics[width=.55\linewidth]{img/3/Synchronous-machine-rotor.png}
    \end{figure}
\end{minipage}\hfill
\begin{minipage}[c]{.5\linewidth}
    \centering
    \usetikzlibrary{shapes.arrows}
    \scalebox{0.5}{
    \begin{tikzpicture}[>=stealth,cross/.style={path picture={ 
    \draw[black]
    (path picture bounding box.south east) -- (path picture bounding box.north west) (path picture bounding box.south west) -- (path picture bounding box.north east);
    }}]
    
    
        % main circles
        \draw (0,0) circle (5cm);
        \draw (0,0) circle (3cm);
        \fill[gray!40,even odd rule] circle[radius=5cm] circle[radius=3cm];
        
        % small circles
        \foreach \X [count=\Y starting from 0] in {$a_2'$,$b_2$,$c_1'$,$a_2$,$b_1'$,$c_1$,$a_1'$,$b_1$,$c_2'$,$a_1$,$b_2$,$c_2$} { 
        \path(180-30*\Y:4) node[circle,draw, fill=white] (n\Y) {\X};
        }
        
        % Two-way arrow
        \node[blue!40,fill=blue!40,double arrow, draw, minimum height=6cm, minimum width=2.8cm, rotate=14,text=black] at (0,0) {};
        \node[blue!40,fill=blue!40,double arrow, draw, minimum height=6cm, minimum width=2.8cm, rotate=-76,text=black] at (0,0) {};
        
        % very small circles
        \foreach \X [count=\Y starting from 0] in {0,180} { 
            \draw[rotate around={\X:(0,0)}] (-1.5,0.3) circle (0.15cm);
            \draw[rotate around={\X:(0,0)}, fill=black] (-1.5,0.3) circle (0.075cm);
            \draw[rotate around={\X:(0,0)}] (-1.1,0.4) circle (0.15cm);
            \draw[rotate around={\X:(0,0)}, fill=black] (-1.1,0.4) circle (0.075cm);
            
            \draw[rotate around={\X:(0,0)}] (-0.97,1.2) circle (0.15cm);
            \draw[rotate around={\X:(0,0)},fill=black] (-0.97,1.2) circle (0.075cm);
            \draw[rotate around={\X:(0,0)}] (-0.88,0.8) circle (0.15cm);
            \draw[rotate around={\X:(0,0)},fill=black] (-0.88,0.8) circle (0.075cm);
        }
        
        \foreach \X [count=\Y starting from 0] in {-90,90} { 
            \draw[rotate around={\X:(0,0)}, cross] (-1.5,0.3) circle (0.15cm);
            \draw[rotate around={\X:(0,0)},cross] (-1.1,0.4) circle (0.15cm);
            
            \draw[rotate around={\X:(0,0)},cross] (-0.97,1.2) circle (0.15cm);
            \draw[rotate around={\X:(0,0)},cross] (-0.88,0.8) circle (0.15cm);
        }
        
        % Poles
        \node[rotate=14] at (-0.5,2) {$N$};
        \node[rotate=14] at (0.5,-2) {$N$};
        
        \node[rotate=-76] at (2,0.5) {$S$};
        \node[rotate=-76] at (-2,-0.5) {$S$};
    
    \end{tikzpicture}
    }
\end{minipage}

\vspace{1.5em}
\noindent Para esta situação já é necessário criar uma distinção entre o ângulo elétrico e o ângulo mecânico, de acordo com a distribuição espacial do campo magnético:

\begin{figure}[H]
    \centering
    \begin{tikzpicture}
        \begin{axis}[
            xlabel=\empty,
            ylabel={\( B \)},
            axis lines=center,
            ytick=\empty,
            xtick={0,1.5708,3.14159,4.71239,6.28319,7.85398,9.42478,10.99557,12.5664},
            xticklabels=\empty,
            ymin=-1, ymax=1,
            xmin=0, xmax=14,
            domain=0:4*pi,
            samples=500,
            width=12cm, height=4cm,
        ]
        
        \addplot[black, thick] {0.75*sin(deg(x))};

        \node[above, font=\small] at (13.75,0) {$\theta_m$};
        \node[below, font=\small] at (13.725,0) {$\theta$};

        \node[above, xshift=1mm,  font=\small] at (3.14159,0) {$\frac{\pi}{2}$};
        \node[below, xshift=-1mm, font=\small] at (3.14159,0) {$\pi$};

        \node[above, xshift=-1mm,  font=\small] at (6.28319,0) {$\pi$};
        \node[below, xshift=1mm, font=\small] at (6.28319,0) {$2\pi$};

        \node[above, xshift=1mm,  font=\small] at (9.42478,0) {$\frac{3\pi}{2}$};
        \node[below, xshift=-1mm, font=\small] at (9.42478,0) {$3\pi$};

        \node[above, xshift=-1mm,  font=\small] at (12.5664,0) {$2\pi$};
        \node[below, xshift=1mm, font=\small] at (12.5664,0) {$4\pi$};
        
        \end{axis}
    \end{tikzpicture}
    \caption{Distribuição espacial da indução magnética $B$ para uma máquina de 4 pólos ($\theta_m$ --- rad. mecânicos; $\theta$ --- rad. elétricos)}
    \label{fig:distrib-espacial-B}
\end{figure}

\noindent Numa máquina com $p$ pares de pólos temos $\theta = p \theta_m$ em que $\theta$ é o ângulo elétrico e $\theta_m$ o ângulo mecânico. A frequência angular da tensão induzida $\omega$ vem então:
$$
    \omega = \frac{d\theta}{dt} = p\frac{d\theta_m}{dt} = p \omega_r 
$$
em que $\omega_r$ é a velocidade ângular do rotor. E a frequência da tensão (Hz) relaciona-se com a velocidade de rotação do motor, $n_r$ (rpm), pela expressão:
$$
    f = p \frac{n_r}{60}
$$

\clearpage
\noindent A variação espacial da indução magnética $\mathbf{B}$ ao longo do entreferro é sinusoidal, i.e., $B = B_{max}\cos(p\alpha)$, em que $B_{max}$ é o valor máximo medido no centro da cabeça do pólo e $\alpha$ o ângulo que define um ponto ao longo do entreferro, medido em radianos mecânicos a partir do eixo magnético do rotor.

O fluxo magnético por pólo $\Phi$ é o integral da indução magnética ao longo de uma revolução completa
$$
    \Phi = \int_{-\pi/2}^{\pi/2} B_{max}\cos(p\alpha) \cdot lr \,d\alpha = \frac{2B_{max}lr}{p}
$$
onde $l$ é o comprimento axial do estator e $r$ o raio interior. O fluxo ligado $\Psi$ com a fase $a$ do estator (referência), admitindo um enrolamento com $N$ espiras, é dado por
$$
    \Psi = N\Phi\cos(\theta)
$$
onde $\theta$ é o ângulo do eixo do rotor em radianos elétricos, medidos a partir do eixo magnético do enrolamento da fase $a$ do estator.
$$
    \theta = p\omega_r t = \omega t
$$
$$
    \therefore \Psi = N\Phi\cos(\omega t)
$$
Pela Lei de Indução, a tensão induzida na fase $a$ é 
$$
    e = -\frac{d}{dt} \Psi = \omega N \Phi \sin(\omega t)
$$
Esta tensão induzida (\textit{força eletromotriz}), é sinusoidal, com frequência $\omega = 2\pi f$ e está desfasada $\pi/2$ em atraso relativamente ao fluxo. É costume definir o valor eficaz desta grandeza (fase-neutro):
$$
    \boxed{ E = \frac{\omega N \Phi}{\sqrt{2}} = \sqrt{2} \pi fN \Phi }
    \implies 
    e = \sqrt{2} E \sin(\omega t)
$$

%//==============================--@--==============================//%
\subsubsection{Reação do Induzido e Esquema Equivalente}

Quando o gerador está em carga e é alimentado, o sistema trifásico de correntes simétricas que se dá no enrolamento estatórico origina um campo magnético girante no entreferro, uma vez que as correntes em cada fase estão desfasadas por $120^{\circ}$ temporalmente e espacialmente. Este fenómeno designa-se por reação do induzido, i.e., a aparição deste campo magnético à velocidade de sincronismo que se soma ao campo devido à corrente de excitação.

O fluxo magnético resultante é uma combinação dos três fluxos individuais devido às correntes no estator (vamos tomar a fase $a$ como referência):
$$
    \Psi_r = L i_a + M i_b + M i_c
$$
onde $L$ e $M$ são as indutâncias própria e mutua, respetivamente. Em regime trifásico simétrico, $\sum i_x = 0$,
$$
    \therefore \Psi_r = (L - M) i_a
$$
A tensão induzida por estas correntes na fase $a$ é portanto
$$
    e_r = -\frac{d}{dt} \Psi_r = -(L - M) \frac{d i_a}{dt}
$$
A tensão aos terminais do gerador em carga é a soma da f.e.m. devido ao indutor com a tensão devido à reação do induzido
$$
    v = e + e_r = e - (L - M) \frac{d i_a}{dt}
$$
Em regime vetorial temos:
$$
    \begin{aligned}
        \mathbf{V} &= \mathbf{E} - j\omega(L - M) \mathbf{I} \\
                   &= \mathbf{E} - jX_s \mathbf{I}
    \end{aligned}
$$
Conhecendo a tensão aos terminais e a corrente, calcula-se a f.e.m. por
$$
    \mathbf{E} = \mathbf{V} + jX_s \mathbf{I}
$$
onde a grandeza $X_s$ recebe o nome de reatância síncrona. Normalmente também inclui a reatância de dispersão do estator, que não foi considerada na análise anterior. É pratica comum expressar $X_s$ em p.u., referida aos valores nominais $S_n$ e $V_n$:
$$
    X_s = X_{s_{(p.u.)}} \frac{V^2_n}{S_n}\; [\text{p.u.}]
$$

\clearpage
\noindent Em regime estacionário (trifásico simétrico) o esquema equivalente pode ser representada por

\begin{figure}[H]
    \centering
    \begin{subfigure}[b]{.475\linewidth}
        \centering
        \scalebox{0.75}{%
            \begin{circuitikz}
                %% Configure circuitikz
                \ctikzset{inductor=cute}    
                \ctikzset{inductors/coils=4}
                \ctikzset{bipoles/resistor/height=0.25}
                \ctikzset{bipoles/resistor/width=0.5}
                \ctikzset{resistors/zigs=4}
            
                \draw (0,-2) to[sV, l=$\mathbf{E}$] (0,2) 
                to[L, l=$jX_s$, -*] (4,2);
                \draw (0,-2) to[short,-*] (4,-2);

                \draw[>=stealth,->] (4,1.75) -- (4,-1.75) node[midway,right] {$\mathbf{V}$};
                \draw[>=stealth,->] (3.25,2) -- (3.35,2) node[midway,above] {$\mathbf{I}$};
            \end{circuitikz}
        }
        \caption{Esquema monofásico equivalente}
    \end{subfigure}\hfill
    \begin{subfigure}[b]{.475\linewidth}
        \centering
        \scalebox{0.925}{%
            \begin{tikzpicture}[>=latex, scale=3]
                \coordinate (E) at (0:2.5);
                \coordinate (I) at (-53.13:0.5); 
                \coordinate (I') at (-53.13:1.5);
                \coordinate (V) at (-20:1.791);
                
                % Phasors
                \draw[->,black] (0,0) -- (E) node[right, above] {$E$};
                \draw[->,black] (0,0) -- (I) node[below=0.2, left] {$I$};
                \draw[-,gray, dashed] (I) -- (I');
                \draw[->,black] (0,0) -- (V) node[left,below] {$V$};
                \draw[-,gray, dashed] (I') -- (V);
                \draw[->,black] (V) -- (E) node[below=0.25] {$j X_s I$};
            
                % Arcs
                \draw[-] (-20:0.2cm) arc (-20:-53.13:0.2cm) node[yshift=-0.2mm,xshift=2.4mm] {$\phi$};
                \draw[-] (0:0.4cm) arc (0:-20:0.4cm) node[yshift=1.5mm,xshift=2mm] {$\delta$};
            \end{tikzpicture}
        }
        \caption{Diagrama de fasores}
    \end{subfigure}
    
    \caption{Máquina síncrona como gerador}
    \label{fig:esquema-equiv-maq-sincrona}
\end{figure}

\noindent Despreza-se a resistência dos enrolamentos (valor pequeno face à reatância); admite-se que $\mathbf{I}$ está desfasado com um atraso de ângulo $\phi$ relativamente à tensão $\mathbf{V}$, e a f.e.m. $\mathbf{E}$ por um ângulo $\delta$ também relativamente à tensão, designado por ângulo de potência.

%//==============================--@--==============================//%
\subsection{Características de Funcionamento}
\subsubsection{Em Vazio e em Curto-circuito}

A característica em vazio é a curva da f.e.m. $E$ (tensão em vazio) em função da corrente de excitação $I_{exc}$, com a máquina a rodar à velocidade nominal (sincronismo), movida pela máquina de acionamento.

A curva característica em vazio exibe uma zona linear (cuja linha tangente é a reta entreferro), para valores baixos de corrente de excitação. Apresenta a não-linearidade resultante de saturação do ferro, quando o fluxo magnético excede um determinado valor limite.

Na operação próxima à tensão nominal, aproxima-se a máquina a outra fictícia que não exibe saturação, caracterizada pela reta de magnetização que passa pela origem e pelo ponto correspondente à tensão nominal.

\vspace{1em}\hrule\vspace{1em}

\noindent A característica em curto-circuito é a curva da corrente no estator $I$ em função de $I_{exc}$, com a máquina a rodar à velocidade síncrona, e os enrolamentos do estator em curto-circuito.

A curva característica em curto-circuito é linear, uma vez que o fluxo magnético tem um valor muito baixo nestas condições (não se manifesta a saturação).

\vspace{1em}\hrule\vspace{1em}

\noindent Podemos calcular a reatância $X_s$ a partir das características em vazio e em curto-circuito. Lembramos que $E \;\propto\; \omega$, ao fixarmos $I_{exc}$ correspondente à tensão nominal no ensaio em vazio, $\mathbf{E}=\mathbf{V}_n$. No ensaio em curto-circuito, $\mathbf{E}=jX_s \mathbf{I}_{cc}$, então, para o $I_{exc}$ fixo escolhido anteriormente, e como $E \;\propto\; \omega$ é um valor fixo à tensão nominal, temos que $V_n = X_s I_{cc}$, isto é:
$$
    \therefore X_s = \frac{V_n}{I_{cc}}
$$

\vspace{-1.5em}
\begin{minipage}[c]{.5\linewidth}
\begin{figure}[H]
    \centering
    \scalebox{1.15}{%
        \begin{tikzpicture}
            \begin{axis}[
                xmin=0, xmax=1,
                ymin=0, ymax=2.5,
                axis lines=middle,
                width=7cm,
                height=7cm,
                xtick=\empty, 
                ytick={1.291}, 
                yticklabels={$V_n$},
                tick style={font=\tiny},
                ytick align=outside, 
                xtick align=outside,
                ytick pos=left,
                ticklabel style = {font=\tiny},
                clip=false,
                restrict y to domain=*:2.5,
                legend pos=north west, % Position of the legend
                legend style={
                    legend columns=2,
                    font=\tiny,
                    row sep=0.05pt, % Adjust the vertical spacing between legend entries
                    column sep=1pt, % Adjust the horizontal spacing between legend image and text
                    nodes={scale=0.75, transform shape}
                },
            ]
                \node[font=\scriptsize] at (-0.04,2.5) {$E$};
                \node[font=\scriptsize] at (1.035,2.5) {$I$};
                \node[font=\scriptsize] at (1.06,-0.05) {$I_{exc}$};
        
                % OCC
                \addplot[blue!90, domain=0.1692:0.9, samples=100] ({x},{1.5*(1-1.2*exp(-5*x))});
                \addlegendentry{Em Vazio}
                
                % Entreferro
                \addplot[blue!90, domain=0:0.1692, samples=100, forget plot] ({x},{4.3*x});
                \addplot[dashed, black, domain=0:0.485, samples=100] ({x},{4.3*x});
                \addlegendentry{Entreferro}
        
                % No saturation
                \addplot[gray!50, domain=0:0.63, samples=100] ({x},{3*x});
                \addlegendentry{Reta de Magnetização}
        
                % SC
                \addplot[green!50!blue, domain=0:0.9, samples=100] ({x},{1.25*x});
                \addlegendentry{Em Curto-circuito}
        
                % relevant horizontal values
                \addplot[dotted,black, domain=0:1.291, samples=100] ({0.4302},{x});
                \addplot[dotted,black, domain=0:0.4302, samples=100] ({x},{1.291});
                \addplot[dotted,black, domain=0.4302:1, samples=100] ({x},{0.538});
            \end{axis}
            \begin{axis}[
                xmin=0, xmax=1,
                ymin=0, ymax=2.5,
                axis y line*=right,
                axis x line=none,
                axis lines=left, % Adjusted line
                ytick={0.538}, 
                yticklabels={$I_{cc}$},
                width=7cm,
                height=7cm,
                tick style={font=\tiny},
                ytick align=outside, 
                xtick align=outside,
                ytick pos=right,
                ticklabel style = {font=\tiny},
                clip=false,
                restrict y to domain=*:2.5,
            ]
            \end{axis}
        \end{tikzpicture}
    }
    \caption{Caracteristica em vazio e em curto-circuito}
    \label{fig:maq-sincrona-vazio-e-cc}
\end{figure}
\end{minipage}\hfill
\begin{minipage}[c]{.45\linewidth}
    \begin{mdframed}
        \vspace{2.25cm}
        \hfil (imagem)
        \vspace{2.25cm}
    \end{mdframed}
\end{minipage}

%//==============================--@--==============================//%
\clearpage
\subsubsection{Em Carga}

\noindent A potência nominal de uma máquina síncrona  é a máxima potência aparente à tensão nominal com fp $= 0.85,0.90, 0.95$ que pode fornecer continuamente. O fator que a limita é o aquecimento devido às correntes que percorrem os enrolamentos (limite térmico).

\begin{mdframed}
    \hfil Potência ativa $<$ potência nominal $\rightarrow$ limitada pela potência da máquina de acionamento.
\end{mdframed}

\noindent Quando a máquina está a funcionar à velocidade síncrona e está excitada de modo a apresentar a sua tensão nominal em vazio, podemos assumir que a corrente de carga aumentará gradualmente a partir de zero até atingir o seu valor nominal, mantendo um fator de potência constante.

\begin{minipage}[b]{0.5\linewidth}
   \begin{figure}[H]
        \centering
        \begin{tikzpicture}[>=latex, scale=3]
   
            \coordinate (E) at (0:2.5);
            \coordinate (I) at (-53.13:0.5); 
            \coordinate (I') at (-53.13:1.5);
            \coordinate (V) at (-20:1.791);
            
            % Phasors
            \draw[->,black] (0,0) -- (E) node[right, above] {$E$};
            \draw[->,black] (0,0) -- (I) node[below=-0.01] {$I$};
            \draw[-,gray, dashed] (I) -- (I');
            \draw[->,black] (0,0) -- (V) node[right,below] {$V$};
            \draw[-,gray, dashed] (I') -- (V);
            \draw[->,black] (V) -- (E) node[below=0.15] {$j X_s I$};
        
            % Arcs
            \draw[-] (-20:0.2cm) arc (-20:-53.13:0.2cm) node[yshift=-0.2mm,xshift=2.4mm] {$\phi$};
            \draw[-] (0:0.4cm) arc (0:-20:0.4cm) node[yshift=1.5mm,xshift=2mm] {$\delta$};
            
    \end{tikzpicture}
    \end{figure} 
\end{minipage}\hfill
\begin{minipage}[b]{0.45\linewidth}
    \noindent Do diagrama de fasores, retiramos as seguintes equações:
    $$
    \begin{aligned}
        E \sin(\delta) &= X_s I \cos(\phi)\\
        E \cos(\delta) &= V + X_s I \sin(\phi)
    \end{aligned}
    $$
    Resolvendo em ordem a $V$  e eliminando o ângulo $\delta$, obtem-se:
    
    $$
        \boxed{V = \sqrt{E^2 - X_s^2 I^2 \cos(\phi)^2} - X_s I \sin(\phi)}
    $$
\end{minipage}

\vspace{1em}
\noindent Supondo $E$ constante, \underline{a tensão $V$ vai experimentar variação}:


\hspace{-1.5em}\begin{minipage}[c]{0.45\linewidth}
\noindent Pressupondo uma máquina síncrona com reactância síncrona $X = 1$ p.u. e velocidade nominal constante, com corrente de excitação constante para a tensão nominal em vazio, observamos a variação da tensão nos terminais à medida que a corrente de carga varia de 0 a 1 p.u. para diferentes fatores de potência. \underline{Calculando os extremos}:

\vspace{0.75 em}
\noindent\textbf{Para f.p. = 1:}
$$
    \boxed{V^2 + X_s^2 I^2 = E^2}\; \rightarrow\; \text{Elipse}
$$

\noindent\textbf{Para f.p. = 0:}
$$
    \begin{aligned}
       &\text{Indutivo:}\quad &\boxed{V = E - X_s I}\\
        &\text{Capacitivo:}\quad &\boxed{V = E + X_s I}
    \end{aligned}
$$

\noindent A variação da tensão com a corrente é linear.

\end{minipage}
\begin{minipage}[c]{0.5\linewidth}
\begin{figure}[H]
    \centering
    \begin{tikzpicture}[scale=1.4]
        \begin{axis}[
            xmin=0, xmax=1,
            ymin=0, ymax=2,
            axis lines=middle,
            width=7cm,
            height=7cm,
            xtick={0,0.2,0.4,0.6,0.8,1}, 
            ytick={1}, 
            tick style={font=\tiny},
            ytick align=outside, 
            xtick align=outside,
            ytick pos=left,
            ticklabel style = {font=\tiny},
            clip=false
        ]
    
            \node[font=\scriptsize] at (-0.06,1.95) {$V_\text{p.u.}$};
            \node[font=\scriptsize] at (1.06,-0.05) {$I_\text{p.u.}$};
            
             % Plot for delta = 0 ind and cap
            \addplot[black, domain=0:1, samples=100] {1 - x} node[right, font=\tiny] at (axis cs:0.7,0.15) {0 ind};
            \addplot[black, domain=0:1, samples=100] {1 + x} node[right, font=\tiny] at (axis cs:0.7,1.85) {0 cap};
            
            % Plot for delta = 0.85 ind and cap
            \addplot[black,dashed, domain=0:1, samples=100] {sqrt(1 - 0.85^2*x^2) - x*0.52678} node[right, font=\tiny] at (axis cs:0.7,0.43) {0.85 ind};
            \addplot[black,dashed, domain=0:1, samples=100] {sqrt(1 - 0.85^2*x^2) + x*0.52678} node[right, font=\tiny] at (axis cs:0.7,1.2) {0.85 cap};
            
            % Plot for delta = 1
            \addplot[black,densely dotted, domain=0:1, samples=100] {sqrt(1 - x^2)} node[right, font=\tiny] at (axis cs:0.7,0.73) {1};
        \end{axis}
    \end{tikzpicture}
\end{figure} 
\end{minipage}

\vspace{0.5em}
\begin{mdframed}
    \begin{enumerate}
        \item Quando o fator de potência é unitário ou indutivo:
        \begin{itemize}
            \item A tensão diminui à medida que a corrente de carga aumenta.
            \item Isso ocorre devido ao efeito desmagnetizante da reação do induzido, onde o fluxo magnético se subtrai do fluxo principal.
        \end{itemize}
    
        \item Quando o fator de potência é capacitivo:
        \begin{itemize}
            \item A tensão aumenta à medida que a corrente de carga.
            \item Isso ocorre porque a reação do induzido tem um efeito magnetizante, onde o fluxo magnético se soma ao fluxo principal, especialmente em correntes relativamente baixas.
        \end{itemize}
    \end{enumerate}
\end{mdframed}

\clearpage
\noindent Se se pretender  manter constante a tensão então há que \underline{atuar sobre a corrente de excitação que condiciona $E$}. Para uma dada potência ativa, sendo a amplitude de tensão $V$ constante a variação da corrente de excitação altera $E$ donde resulta uma variação de intensidade $I$ e $\sin(\phi)$. A equação anterior pode reescrever-se:

$$
    \boxed{E = \sqrt{V^2 + X_s^2I^2 + 2V X_s I \sin(\phi)}}
$$

\hspace{-1.5em}\begin{minipage}[c]{0.45\linewidth}
\noindent Pressupondo novamente a máquina síncrona com reactância síncrona $X = 1$ p.u. com tensão nominal aos terminais, observamos a variação da f.e.m $E$ à medida que a corrente de carga varia de 0 a 1.0 p.u. para diferentes fatores de potência e potência ativa. \underline{Calculando os extremos}:

\vspace{1 em}
\noindent\textbf{Para P = 0:}
$$
    \boxed{E = V \pm X_s I}\; \rightarrow\; \text{Linear}
$$
$\cos(\phi) = 0$ e consequentemente, $\sin(\phi) = \pm 1$
\noindent\textbf{Para P = 1:}

$$
    \begin{aligned}
            &E = \sqrt{V^2 + X_s^2 I^2}\\
            &\boxed{E^2 - I^2 = 1}\; \rightarrow\; \text{Hiperbole}
    \end{aligned}
$$
$\cos(\phi) = 1$ e consequentemente, $\sin(\phi) = 0$

\vspace{0.5em}
\noindent O fator de potência encontra-se representado a traço interrompido.

\end{minipage}
\begin{minipage}[c]{0.5\linewidth}
\begin{figure}[H]
    \centering
\begin{tikzpicture}[scale=1.4]
    \begin{axis}[
        xmin=0, xmax=1.1,
        ymin=0, ymax=2,
        axis lines=middle,
        width=7cm,
        height=7cm,
        xtick={0,0.2,0.4,0.6,0.8,1}, 
        ytick={1}, 
        tick style={font=\tiny},
        ytick align=outside, 
        xtick align=outside,
        ytick pos=left,
        ticklabel style = {font=\tiny},
        clip=false,
        restrict x to domain=*:1,
    ]

        \node[font=\scriptsize] at (-0.07,1.95) {$E_\text{p.u.}$};
        \node[font=\scriptsize] at (1.16,-0.05) {$I_\text{p.u.}$};

        % Plot for FP = 1
        \addplot [gray,dashed,domain=0:1, samples=100] {sqrt(x^2 + 1)};

        % Plot for FP = 0.85 cap
        \addplot [gray,dashed,domain=0:1, samples=100] {sqrt(x^2 + 2*0.5267*x + 1)};

        % Plot for FP = 0.85 ind
        \addplot [gray,dashed,domain=0:1, samples=100] {sqrt(x^2 - 2*0.5267*x + 1)};
        
         % Plot for P = 0
        \addplot[black, domain=0:1, samples=100] {1 - x};
        \addplot[black, domain=0:1, samples=100] {1 + x};

         % Plot for P = 1
        \addplot [black, samples=200] ({sqrt(x^2 + 1)-0.05},{x+1.45});

        % Plot for P = 0.5
        \addplot [black, samples=200] ({sqrt(x^2 + sqrt(3)*x + 1)},{x + 2});

        % Plot for P = 0.25
       \addplot [black, samples=200] ({sqrt(x^2 + 0.9682*2*x + 1)},{x + 2});

        % Plot for P = 0.75
        \addplot [black, samples=200] ({sqrt(x^2 + 2*0.6613*x + 1)},{x + 2});


        % Lengends
        \node[font=\tiny] at (1.0195,0.03) {$0$};
        \node[font=\tiny] at (1.05,0.15) {$0.25$};
        \node[font=\tiny] at (1.035,0.35) {$0.5$};
        \node[font=\tiny] at (1.05,0.75) {$0.75$};
        \node[font=\tiny] at (1.0195,1.2) {$1$};

        \node[gray, font=\tiny] at (1.1,1) {$0.85$ cap};
        \node[gray, font=\tiny] at (1.1,1.8) {$0.85$ ind};
        \node[gray, font=\tiny] at (1.0195,1.45) {$1$};

    \end{axis}
\end{tikzpicture}
\end{figure} 
\end{minipage}

\vspace{0.5em}
\begin{mdframed}
    \begin{enumerate}
        \item  O valor da corrente te carga é mínimo para f.p $= 1$ (O traço interrompido interceta as curvas hiperbólicas no mínimo relativo ao eixo das abcissas) aumentando à medida que o fator de potência diminui.

        \item Quando traçadas em função da corrente de excitação, estas curvas são conhecidas pela designação de curvas em V. O seu andamento é semelhante ao gráfico da variação da tensão, ainda que não idêntico, por força da saturação da característica em vazio $E(I_\text{exc})$.
    \end{enumerate}
\end{mdframed}

%//==============================--@--==============================//%
\subsubsection{Fórmulas da Potência Ativa e Reativa}

\noindent Tomando a tensão aos terminais de V como referência, a potência complexa fornecida pelo gerador é:

$$
\begin{aligned}
      \mathbf{S}_G = P_G + jQ_G = &\mathbf{V}\mathbf{I}^* = V e^{j 0} e^{j \phi} = VI e^{j \phi}\\
      P_G = V I \cos(\phi)&\qquad
      Q_G = V I \sin(\phi)
\end{aligned}
$$

\noindent Relembrando as expressões do diagrama de fasores, podemos escrever as potências ativa e reativa da seguinte forma:
$$
    \left\{\begin{aligned}
        E \sin(\delta) &= X_s I \cos(\phi)\\
        E \cos(\delta) &= V + X_s I \sin(\phi)
    \end{aligned}\right.\quad
    \rightarrow\quad
    \left\{\begin{aligned}
         P_G &= \frac{E V}{X_s} \sin(\delta)\\
         Q_G &= \frac{V}{X_s} \left(E\cos(\delta) - V\right) =  \frac{V}{X_s} \Delta
    \end{aligned}\right.\qquad
$$

\noindent O ângulo de potência $\delta$ não é uma variável de controlo. Sendo o gerador um conversor mecanoeléctrico, a potência ativa gerada é (à parte as perdas) igual à potência mecânica fornecida pela máquina motriz. Note-se que $\delta$ depende da f.e.m. $E$ e, por conseguinte, da corrente de excitação:

\begin{mdframed}
    \noindent Constata-se que a potência reativa depende da diferença $\Delta$. Admitindo constante a tensão V:
    $$
        \begin{aligned}
            &E \cos(\delta) = V\; \rightarrow\; \text{A potência reativa é controlável através da corrente de excitação que determina a $E$.}\\
            &\mkern140mu \text{A excitação normal é definida para $\delta$ = 0.}\\
            &E \cos(\delta) > V\; \rightarrow\; \text{A máquina fica sobreexcitada e fornece potência reativa.}\\
            &E \cos(\delta) < V\; \rightarrow\; \text{A máquina fica subexcitada e absorve potência reativa.}
        \end{aligned}
    $$
\end{mdframed}

%//==============================--@--==============================//%

    \clearpage
    \section{Linha Elétrica de Energia}%
        %//==============================--@--==============================//%

A transmissão de energia elétrica é realizada pelo campo eletromagnético criado pela tensão entre os condutores e pela corrente que neles flui.

As linhas são normalmente aéreas, constituidas por condutores de alumímio ou de cobre. Os condutores (sujeitos ao peso e uma força longitudinal) descrevem uma linha designada por \textit{catenária}, a qual para pequenas distancias se aproximada de uma parábola.

A tensão nominal de uma linha determina a sua capacidade de transporte, i.e., quanto maior a tensão, maior é a potência transmitida. As tensões mais elevadas requerem naturalmente um isolamento mais pronunciado, bem como maiores distâncias entre condutores e entre estes e a terra. 

%//==============================--@--==============================//%
\subsection{Modelos da Linha em Regime Estacionário}

\subsubsection{Modelo exato}

Considerando um troço de uma fase de uma linha com comprimento infinitesimal $dx$, onde $v$ é a tensão fase-neutro e $i$ a corrente por fase, funções do tempo e da distância $x$ medida a partir do emissor, podemos escrever:
$$
    \left\{
    \begin{aligned}
        v(x) - v(x+dx) &= R\, dx\, i + L\, dx\, \frac{\partial i}{\partial t} \\[8pt]
        i(x) - i(x+dx) &= G\, dx\, v + C\, dx\, \frac{\partial v}{\partial t}
    \end{aligned}\right.
    \quad\implies\quad
    \left\{
    \begin{aligned}
        -\frac{\partial v}{\partial x} &= Ri + L\frac{\partial i}{\partial t} \\[8pt]
        -\frac{\partial i}{\partial x} &= Gv + C\frac{\partial v}{\partial t}
    \end{aligned}\right.
    \quad\implies\quad
    \left\{
    \begin{aligned}
        -\frac{\partial \mathbf{V}}{\partial x} &= (R + j\omega L) \mathbf{I} \\[8pt]
        -\frac{\partial \mathbf{I}}{\partial x} &= (G + j\omega C) \mathbf{V}
    \end{aligned}\right.
$$

% \vspace{-1em}
% \begin{figure}[H]
%     \centering
%     \scalebox{1.0}{%
%         \begin{circuitikz}
%             %% Esquema equivalente em pi
%             \draw (0,0) 
%                 to[short, *-] (2,0)
%                 to[generic, l=$\mathbf{B}$] (7,0)
%                 to[short, -*] (9,0);

%             \draw (2,0) to[generic, l=$\displaystyle \frac{\mathbf{A}-1}{\mathbf{B}}$] (2,-3);
%             \draw (7,0) to[generic, l_=$\displaystyle \frac{\mathbf{A}-1}{\mathbf{B}}$] (7,-3);

%             \draw (0,-3) to[short, *-*] (9,-3);

%             %% Labels e arrows
%             \draw[>=stealth,->] (0,-0.25) -- (0,-2.75) node[midway,left] {$\mathbf{V}_e$};
%             \draw[>=stealth,->] (9,-0.25) -- (9,-2.75) node[midway,right] {$\mathbf{V}_r$};
%         \end{circuitikz}
%     }
%     \caption{Esquema equivalente monofásico de uma linha com parâmetros distribuidos}
%     \label{fig:linha-esq-monofasico}
% \end{figure}

\noindent Definem-se a característica (ou impedância) da onda $\mathbf{Z}_0$ ($\Omega$) e constante de propagação $\gamma$ (m$^{-1}$):
$$
    \begin{aligned}
        \mathbf{Z}_0 &= \sqrt{\frac{R + j\omega L}{G + j\omega C}} = \sqrt{\frac{R + jX}{G + jB}} \\
        \gamma &= \sqrt{(R + j\omega L)(G + j\omega C)} = \sqrt{(R + jX)(G + jB)}
    \end{aligned}
$$
Para obtermos as soluções definimos as seguintes EDOs, após derivar e substituir:
$$
    \left\{
    \begin{aligned}
        \frac{d^2 \mathbf{V}}{dx^2} &= (R + j\omega L)(G + j\omega C) \mathbf{V} \\[6pt]
        \frac{d^2 \mathbf{I}}{dx^2} &= (R + j\omega L)(G + j\omega C) \mathbf{I}
    \end{aligned}\right.
    \quad\iff\quad
    \left\{
    \begin{aligned}
        \frac{d^2 \mathbf{V}}{dx^2} &= \gamma^2\, \mathbf{V} \\[6pt]
        \frac{d^2 \mathbf{I}}{dx^2} &= \gamma^2\, \mathbf{I}
    \end{aligned}\right.
    \quad\implies\quad
    \boxed{%
    \begin{aligned}
        \mathbf{V} &= \mathbf{V}_e \cosh(\gamma x) - \mathbf{Z}_0 \mathbf{I}_e \sinh(\gamma x) \\[4pt]
        \mathbf{I} &= -\frac{\mathbf{V}_e}{\mathbf{Z}_0} \sinh(\gamma x) + \mathbf{I}_e \cosh{\gamma x}
    \end{aligned}}
$$
As soluções obtidas são para uma distância qualquer entre o emissor e o recetor. Interessa o caso específico no extremo do recetor ($\mathbf{V}_r, \mathbf{I}_r$) em que $x = d$. Sob a forma matricial temos:
$$
    \begin{bmatrix}
        \mathbf{V}_r \\[6pt]
        \mathbf{I}_r
    \end{bmatrix}
    =
    \begin{bmatrix}
        \cosh(\gamma d) & -\mathbf{Z}_0 \sinh(\gamma d) \\[6pt]
        -1/\mathbf{Z}_0 \cdot \sinh(\gamma d) & \cosh(\gamma d)
    \end{bmatrix}
    \begin{bmatrix}
        \mathbf{V}_e \\[6pt]
        \mathbf{I}_e
    \end{bmatrix}
    \iff 
    \begin{bmatrix}
        \mathbf{V}_e \\[6pt]
        \mathbf{I}_e
    \end{bmatrix}
    =
    \begin{bmatrix}
        \cosh(\gamma d) & \mathbf{Z}_0 \sinh(\gamma d) \\[6pt]
        1/\mathbf{Z}_0 \cdot \sinh(\gamma d) & \cosh(\gamma d)
    \end{bmatrix}
    \begin{bmatrix}
        \mathbf{V}_r \\[6pt]
        \mathbf{I}_r
    \end{bmatrix}
$$
Podemos apresentar a equação sob a forma:
$$
    \begin{bmatrix}
        \mathbf{V}_e \\[6pt]
        \mathbf{I}_e
    \end{bmatrix}
    =
    \begin{bmatrix}
        \mathbf{A} & \mathbf{B} \\[6pt]
        \mathbf{C} & \mathbf{D}
    \end{bmatrix}
    \begin{bmatrix}
        \mathbf{V}_r \\[6pt]
        \mathbf{I}_r
    \end{bmatrix}
$$
em que os parâmetros $\mathbf{A}$, $\mathbf{B}$, $\mathbf{C}$ e $\mathbf{D}$ são dados por:
$$
    \begin{aligned}
        \mathbf{A} &= \mathbf{D} = \cosh(\gamma d) = \cosh(\sqrt{\mathbf{Z}_L \mathbf{Y}_T}) \\
        \mathbf{B} &= \mathbf{Z}_0 \sinh(\gamma d) = \frac{\mathbf{Z}_0 \sinh(\sqrt{\mathbf{Z}_L \mathbf{Y}_T})}{\sqrt{\mathbf{Z}_L \mathbf{Y}_T}} \\
        \mathbf{C} &= \frac{\mathbf{Z}_0}{\sinh{\gamma d}} = \frac{\mathbf{Y}_T \sinh(\sqrt{\mathbf{Z}_L \mathbf{Y}_T})}{\sqrt{\mathbf{Z}_L \mathbf{Y}_T}}
    \end{aligned}
$$
onde $\mathbf{Z}_L = (R + j\omega L)d$ e $\mathbf{Y}_T = (G + j\omega C)d$ são a impedância longitudinal e admitância transversal totais, respetivamente.

%//==============================--@--==============================//%
\subsubsection{Esquema em $\pmb{\pi}$ exato}

\begin{figure}[H]
    \centering
    \scalebox{1.0}{%
        \begin{circuitikz}
            %% Esquema equivalente em pi
            \draw (0,0) 
                to[short, *-] (2,0)
                to[generic, l=$\mathbf{B}$] (7,0)
                to[short, -*] (9,0);

            \draw (2,0) to[generic, l=$\displaystyle \frac{\mathbf{A}-1}{\mathbf{B}}$] (2,-3);
            \draw (7,0) to[generic, l_=$\displaystyle \frac{\mathbf{A}-1}{\mathbf{B}}$] (7,-3);

            \draw (0,-3) to[short, *-*] (9,-3);

            %% Labels e arrows
            \draw[>=stealth,->] (0,-0.25) -- (0,-2.75) node[midway,left] {$\mathbf{V}_e$};
            \draw[>=stealth,->] (9,-0.25) -- (9,-2.75) node[midway,right] {$\mathbf{V}_r$};
        \end{circuitikz}
    }
    \caption{}
    \label{fig:linha-transmissao-esq-exato}
\end{figure}

%//==============================--@--==============================//%
\subsubsection{Esquema em $\pmb{\pi}$ nominal}

\begin{figure}[H]
    \centering
    \scalebox{1.0}{%
        \begin{circuitikz}
            %% Esquema equivalente em pi
            \draw (0,0) 
                to[short, *-] (2,0)
                to[generic, l=$\mathbf{Z}_L$] (7,0)
                to[short, -*] (9,0);

            \draw (2,0) to[generic, l=$\displaystyle \frac{\mathbf{Y}_T}{2}$] (2,-3);
            \draw (7,0) to[generic, l_=$\displaystyle \frac{\mathbf{Y}_T}{2}$] (7,-3);

            \draw (0,-3) to[short, *-*] (9,-3);

            %% Labels e arrows
            \draw[>=stealth,->] (0,-0.25) -- (0,-2.75) node[midway,left] {$\mathbf{V}_e$};
            \draw[>=stealth,->] (9,-0.25) -- (9,-2.75) node[midway,right] {$\mathbf{V}_r$};
        \end{circuitikz}
    }
    \caption{}
    \label{fig:linha-transmissao-esq-nominal}
\end{figure}


%//==============================--@--==============================//%
\newpage
\subsection{Capacidade de Transporte}

\subsubsection{Limite Térmico}
\subsubsection{Limite de Estabilidade Estática}
\subsubsection{Limite de Estabilidade da Tensão}

%//==============================--@--==============================//%
\subsection{Parâmetros da Linha}

As linhas elétricas são caracterizadas pela impedância longitudinal e admitância transversal, geralmente expressas em $\Omega$/km e S/km, respetivamente. Os modelos comuns consideram a resistência e a reactância longitudinais; a suscetância transversal é considerada em linhas longas, enquanto a condutância transversal é muitas vezes ignorada. 

\begin{mdframed}
    Ao contrário dos circuitos de parâmetros concentrados, estas linhas têm parâmetros distribuídos ao longo do seu comprimento, o que resulta num tempo de propagação não nulo para o campo eletromagnético, que viaja à velocidade da luz:
    $$
        v = \frac{1}{\sqrt{LC}} = \frac{1}{\sqrt{\mu \varepsilon}} = \frac{1}{\sqrt{\mu_r \varepsilon_r}} \cdot c_0
    $$
    onde $c_0 = 300'000$ km/s é a velocidade da luz no vácuo.
\end{mdframed}

%//==============================--@--==============================//%

    \clearpage
    \section{Trânsito de Energia}%
        %//==============================--@--==============================//%
\renewcommand{\thefootnote}{\fnsymbol{footnote}}

O \textit{Trânsito de Energia}\footnotemark[4] é a solução em regime estacionário de um sistema de energia elétrica, compreendendo os geradores, a rede e as cargas. Nos SEE, é preferível especificar as potências ativas e reativas fornecidas pelos geradores como variáveis de controlo, em vez das correntes injetadas, o que resulta num modelo muito simples para aqueles elementos. As tensões nos nós são as variáveis de estado, e as potências ativas e reativas de carga assumem o carácter de variáveis de perturbação.

O número de nós (barramentos) e de ramos (linhas e transformadores) é muito elevado --- da ordem das centenas ou milhares --- para um sistema de grande porte e as equações que o modelam são não-lineares, o que exige o recurso a um método de cálculo \textit{potente}.

\begin{mdframed}
    O trânsito de energia compreende os seguintes passos:
    \begin{enumerate}[label=\arabic*.,font=\small\bfseries]\small
        \item \underline{Formulação de um modelo} matemático que represente com suficiente rigor o sistema físico real.
        \item \underline{Especificação} do tipo de barramentos e das grandezas referentes a cada um.
        \item \underline{Solução numérica das equações} do trânsito de energia, a qual fornece o valor das amplitudes e argumentos das tensões em todos os barramentos.
        \item \underline{Cálculo das potências que transitam} em todos os ramos --- linhas e transformadores.
    \end{enumerate}
\end{mdframed}

\footnotetext[4]{%
    Também designado por \textit{Trânsito de Potência} ou \textit{Fluxo de Potência} (\textit{Power Flow} ou \textit{Load Flow}). 
}
\renewcommand{\thefootnote}{\arabic{footnote}}

%//==============================--@--==============================//%
\subsection[Sistema com $n$ barramentos]{Sistema com $\pmb{n}$ barramentos}

Começamos por analisar o caso mais simples de um sistema com dois barramentos ligados por uma linha ($n=2$), que irá servir para introduzir as características do trânsito de energia.

Cada barramento é alimentado por geradores que fornecem as potências complexas $\mathbf{S}_{G_1}$ e $\mathbf{S}_{G_2}$, respetivamente a cada barramento. A estes barramentos estão ligadas cargas que consomem $\mathbf{S}_{C_1}$ e $\mathbf{S}_{C_2}$. A linha de transmissão pode ser representada pelo equivalente em $\pi$, conforme apresentado na \hyperref[fig:transito-dois-barramentos]{Fig. X (b)}.

\vspace{0.5em}
\begin{figure}[H]
    \ctikzset{bipoles/length=1.2cm}
    \tikzset{
        line/.style = {line width=1pt, draw=black},
        shin/.style = {line, line width=2pt},
        net node/.style = {circle, draw=black,line width=1.2pt,minimum width=0.8cm, inner sep=0pt, outer 
        sep=0pt},
    }

    \centering
    \begin{subfigure}[b]{0.45\linewidth}
        \centering
        \begin{circuitikz}
            \draw (7.5,12.7) to[sinusoidal voltage source, sources/symbol/rotate=auto] (7.5,12) node[yshift=3mm,xshift=15mm,font=\tiny] {$\;\;\mathbf{S}_{G1} = P_{G1} + jQ_{G1}$};
            \path [shin] (6.5,11.5) -- (8.5,11.5);
            \draw [thick, >=stealth,->](7.5,12) to[short] (7.5,11.5);
            \draw [>=stealth,->](7,11.5) to[short] (7,10) node[right, font=\tiny] {$\mathbf{S}_{C1} = P_{C1} + jQ_{C1}$};
             \node[yshift=-1mm,xshift=2mm,font=\tiny] at (6.1,11.6) {$V_1$};
            
            \draw (11,12.7) to[sinusoidal voltage source, sources/symbol/rotate=auto ] (11,12) node[yshift=3mm,xshift=15mm,font=\tiny] {$\;\;\mathbf{S}_{G2} = P_{G2} + jQ_{G2}$};
            \path [shin] (10,11.5) -- (12,11.5);
            \draw [thick, >=stealth,->](11,12) to[short] (11,11.5);
            \draw [>=stealth,->](11.5,11.5) to[short] (11.5,10) node[right, font=\tiny] {$\mathbf{S}_{C2} = P_{C2} + jQ_{C2}$};
             \node[yshift=-1mm,xshift=2mm,font=\tiny] at (12,11.6) {$V_2$};
            
            \draw [-](8,11.5) to[short] (8,11);
            \draw [-](8,11) to[short] (10.5,11);
            \draw [>=stealth,->](9.25,11) to[short] (9.3,11) node[below, font=\tiny] {Linha};
            \draw [-](10.5,11.5) to[short] (10.5,11);
        \end{circuitikz}

        \caption{Esquema unifilar}
    \end{subfigure}\hfill
    \begin{subfigure}[b]{0.45\linewidth}
        \centering
        \begin{circuitikz}
            \ctikzset{resistors/scale=0.7}
            
            %% Esquema equivalente monofásico
            \path [shin] (1,0.5) -- (2.5,0.5);
            \path [shin] (6.5,0.5) -- (8,0.5);
            
            % Gerador ficticio
            \draw (1.75,1.7) to[esource] (1.75,1) node[yshift=3mm,xshift=15mm,font=\tiny] {$\,\mathbf{S}_{1} =\mathbf{S}_{G1} - \mathbf{S}_{C1}$};
            \fill[pattern={north west lines}, pattern color=black] (1.75,1.35) circle (0.35cm);
            
            \draw (7.25,1.7) to[esource] (7.25,1) node[yshift=3mm,xshift=15mm,font=\tiny] {$\,\mathbf{S}_{2} =\mathbf{S}_{G2} - \mathbf{S}_{C2}$};
            \fill[pattern={north west lines}, pattern color=black] (7.25,1.35) circle (0.35cm);
            
            % Setas do gerador ficticio
            \draw [thick, >=stealth,->](1.75,1) to[short] (1.75,0.5);
            \draw [thick, >=stealth,->](7.25,1) to[short] (7.25,0.5);
            
            \draw (2,0) 
                to[short, -] (2,0)
                to[generic, l=$\mathbf{Z}_L$] (7,0)
                to[short, -] (7,0);
            
            \draw (2,0.5) to[generic, l=$\displaystyle \frac{\mathbf{Y}_T}{2}$] (2,-1.75);
            \draw (7,0.5) to[generic, l_=$\displaystyle \frac{\mathbf{Y}_T}{2}$] (7,-1.75);
            \draw[dashed] (1.25,-1.75) -- (7.75,-1.75);
        \end{circuitikz}

        \caption{Esquema monofásico equivalente}
    \end{subfigure}

    \caption{Sistema com dois barramentos}
    \label{fig:transito-dois-barramentos}
\end{figure}

\noindent Definimos a \textit{potência injetada} $\mathbf{S}$ como a diferença entre as potências gerada e consumida em cada barramento. Para este exemplo temos:
$$
    \begin{aligned}
        \mathbf{S}_1 &= P_1 + jQ_1 = (P_{G_1} - P_{C_1}) + j(Q_{G_1} - Q_{C_1})\\
        \mathbf{S}_2 &= P_2 + jQ_2 = (P_{G_2} - P_{C_2}) + j(Q_{G_2} - Q_{C_2})
    \end{aligned}
$$
Conclui-se que se num dado barramento a geração for superior à carga, a potência injetada será positiva; o oposto ocorre na situação inversa. Os geradores fictícios que entregam as potências injetadas nos barramentos são representados na \hyperref[fig:transito-dois-barramentos]{Fig. X (b)} por círculos sombreados.

%//==============================--@--==============================//%
\vspace{0.5em}\hrule\vspace{0.5em}
\noindent A corrente injectada num barramento \( \mathbf{I} \) está relacionada com a potência injetada pela equação:
$$
    \mathbf{I} = \frac{\mathbf{S}^*}{\mathbf{V}^*} = \frac{P - jQ}{\mathbf{V}^*}
$$

\noindent A aplicação da primeira lei de Kirchhoff aos barramentos 1 e 2 conduz às equações nodais:
$$
    \begin{aligned}
        \mathbf{I}_1 &= \frac{\mathbf{S}_1^*}{V_1^*} = \frac{\mathbf{Y}_T}{2} \mathbf{V}_1 + \frac{1}{\mathbf{Z}_L} (\mathbf{V}_1 - \mathbf{V}_2) \\
        \mathbf{I}_2 &= \frac{\mathbf{S}_2^*}{\mathbf{V}_2^*} = \frac{\mathbf{Y}_T}{2} \mathbf{V}_2 + \frac{1}{\mathbf{Z}_L} (\mathbf{V}_2 - \mathbf{V}_1)      
    \end{aligned}
    \qquad\rightarrow\qquad
    \boxed{\begin{aligned}
        \frac{\mathbf{S}_1^*}{\mathbf{V}_1^*} = \mathbf{y}_{11} \mathbf{V}_1 + \mathbf{y}_{12} \mathbf{V}_2\\
        \frac{\mathbf{S}_2^*}{\mathbf{V}_2^*} = \mathbf{y}_{21} \mathbf{V}_1 + \mathbf{y}_{22} \mathbf{V}_2
    \end{aligned}}
$$

\begin{mdframed}
    Podemos assim definir a \textit{matriz das admitâncias nodais}:
    $$
    \left\{
    \begin{aligned}
        y_{11} &= \frac{Y_T}{2} + \frac{1}{Z_L}\\
        y_{12} &= y_{21} = -\frac{1}{Z_L}\\
        y_{22} &= \frac{Y_T}{2} + \frac{1}{Z_L}
    \end{aligned}
    \right.\quad\xrightarrow{}\quad
    \begin{bmatrix}
        \mathbf{Y}
    \end{bmatrix}
    =
    \begin{bmatrix}
        y_{11} & y_{12} \\[6pt]
        y_{21} & y_{22}
    \end{bmatrix}
    $$
    Sob a forma matricial, as equações nodais escrevem-se:
    $$
    \begin{bmatrix}
        \dfrac{\mathbf{S}^*}{\mathbf{V}^*}
    \end{bmatrix} =
    \begin{bmatrix}
        \mathbf{Y}
    \end{bmatrix}
    \begin{bmatrix}
        \mathbf{V}
    \end{bmatrix}
    $$
    Note-se que estas equações relacionam tensões e potências (e não correntes), o que as torna não lineares, uma alternativa será:
    $$
    \begin{bmatrix}
        \mathbf{V}
    \end{bmatrix} =
    \begin{bmatrix}
        \mathbf{Z}
    \end{bmatrix}
    \begin{bmatrix}
        \dfrac{\mathbf{S}^*}{\mathbf{V}^*}
    \end{bmatrix}
    \qquad\qquad
    \begin{bmatrix}
        \mathbf{Y}
    \end{bmatrix}^{-1} = 
    \begin{bmatrix}
        \mathbf{Z}
    \end{bmatrix}
    $$
    onde $[\mathbf{Z}]$ é a \textit{matriz das impedâncias nodais}.
\end{mdframed}

\noindent Em notação compacta, $\mathbf{S}_i^*$, pode ser reescrito como:
$$
    \mathbf{S}_i^* = P_i - jQ_i = \mathbf{V}_i^*\sum_{j = 1}^2 \mathbf{y}_{ij} \mathbf{V}_j,\quad i = 1,2
$$
%//==============================--@--==============================//%
\vspace{0.5em}\hrule\vspace{0.5em}
\noindent As equações do trânsito de energia escrevem-se também na forma real, cabendo a cada barramento duas equações. Normalmente, exprime-se a tensão em notação polar e a admitância complexa em notação retangular:
$$
    \mathbf{V}_i = V_i e^{j\theta_i}\qquad
    \mathbf{y}_{ij} = G_{ij} + j B_{ij}
$$
onde $G_{ij}$ e $B_{ij}$ são a condutância e a suscetância nodais, respetivamente. Por substituição na fórmula de notação compacta obtemos:
$$
    P_i - jQ_i = \sum_{j = 1}^2 \left(G_{ij} + j B_{ij}\right)\, V_iV_je^{(\theta_j - \theta_i)},\quad i = 1,2
$$
Decompondo nas partes real e imaginária:
\begin{mdframed}
    $$
        \begin{aligned}
            P_i = P_{Gi} - P_{Ci} = \sum_{j=1}^{2} V_i V_j\, \Big[G_{ij} \cos(\theta_i - \theta_j) + B_{ij} \sin(\theta_i - \theta_j)\Big], \quad &i = 1,2\\
            Q_i = Q_{Gi} - Q_{Ci} = \sum_{j=1}^{2} V_i V_j\, \Big[G_{ij} \sin(\theta_i - \theta_j) - B_{ij} \cos(\theta_i - \theta_j)\Big], \quad &i = 1,2
        \end{aligned}
    $$
\end{mdframed}
%//==============================--@--==============================//%
\subsubsection{Característica das equações}
As equações do trânsito de energia exibem as seguintes características:

\begin{itemize}
    \item As equações são algébricas, porque modelam matematicamente o sistema em regime estacionário\footnote{Se se tratasse de um regime dinâmico, teríamos equações diferenciais.}.
    
    \item As equações são não-lineares, o que torna impossível uma solução por via analítica. Usando um computador digital, pode no entanto obter-se uma solução numérica.
\end{itemize}

\noindent As equações do balanço de potência ativa e das respetivas perdas $P_L$ são dadas por:
$$
    P_{G1} + P_{G2} = P_{C1} + P_{C2} + P_L
$$
$$
    P_L = (V_1^2 + V_2^2) G_{11} + 2 V_1 V_2 G_{12} \cos(\theta_1 - \theta_2)
$$
\noindent De forma análoga obtemos equação de balanço de potência reativa e as respetivas perdas $Q_L$:
$$
    Q_{G1} + Q_{G2} = Q_{C1} + Q_{C2} + Q_L
$$
$$
    Q_L = - (V_1^2 + V_2^2) B_{11} - 2 V_1 V_2 B_{12} \cos(\theta_1 - \theta_2)
$$

\noindent As perdas são função dos módulos e argumentos das tensões.

\noindent Para o sistema em estudo existem 12 variáveis: quatro potências activas \(P_{G1}\), \(P_{C1}\), \(P_{G2}\) e \(P_{C2}\), quatro potências reactivas \(Q_{G1}\), \(Q_{C1}\), \(Q_{G2}\) e \(Q_{C2}\), das tensões \(V_1\) e \(V_2\) e dois argumentos \(\theta_1\) e \(\theta_2\). Por conseguinte, temos que especificar oito destas variáveis, restando quatro incógnitas que podem ser obtidas por solução das quatro equações do trânsito de energia de que dispomos. Contudo, a solução do trânsito de energia é limitada por dois obstáculos:

\begin{enumerate}
    \item Não é possível especificar as potências geradas nos dois barramentos, uma vez que não se conhecem as perdas, que são função das incógnitas.
    
    \item Não é possível calcular os valores de \(\theta_1\) e \(\theta_2\), somente a sua diferença \(\theta_1 - \theta_2\).
\end{enumerate}

\begin{mdframed}
    A solução do problema requer agora os seguintes passos:
    \begin{enumerate}
        \item Conhecer ou estimar as cargas activas e reactivas impostas pelos consumidores.
        
        \item Especificar a tensão e o argumento no barramento 1 (normalmente toma-se \(\theta_1 = 0\)), que passa a desempenhar o papel de barramento de referência, bem como as potências activa e reactiva geradas no barramento 2.
        
        \item Resolver as equações do trânsito de energia para obter a tensão e o argumento no barramento 2 e as gerações activa e reactiva no barramento 1, que assume assim também o papel de barramento de \textit{balanço}\footnote{Porque permite fechar o balanço energético do sistema: geração = carga + perdas.}.
    \end{enumerate}
\end{mdframed}

%//==============================--@--==============================//%
\subsubsection{Modelo Matemático}

A generalização das equações do trânsito de energia para um sistema com $n$ barramentos é trivial. Considere-se o barramento genérico $i$ do sistema, por definição, a potência injetada $\mathbf{S}_i = P_i + jQ_i$ é dada por
$$
    \mathbf{S}_i = \mathbf{S}_{G_i} - \mathbf{S}_{C_i} = (P_{G_i} - P_{C_i}) + j(Q_{G_i} - Q_{C_i})
$$
Representando a linha $k$, ligada entre os nós $i$ e $j$ pelo esquema equivalente em $\pi$, a aplicação da KCL ao barramento $i$ resulta em:
$$
    \mathbf{I}_i = \frac{\mathbf{S}^{*}_i}{\mathbf{V}^{*}_i} = \mathbf{y}_{ii} \mathbf{V}_i + \sum_{{j = 1}\atop{j \neq i}}^{n} \mathbf{y}_{ij} \mathbf{V}_j = \sum_{j = 1}^{n} \mathbf{y}_{ij} \mathbf{V}_j
$$
onde 
$$
    \begin{aligned}
        \mathbf{y}_{ii} &= \sum_{{j = 1}\atop{j \neq i}}^{n} \left( \frac{(\mathbf{Y}_T)_k}{2} + \frac{1}{(\mathbf{Z}_L)_k} \right)  \\
        \mathbf{y}_{ij} &= \mathbf{y}_{ji} = -\frac{1}{(\mathbf{Z}_L)_k}
    \end{aligned}
$$
A matriz das admitâncias nodais possui dimensão $n \times n$:
$$
    [\mathbf{Y}] =
    \begin{bmatrix}
        \mathbf{y}_{11} & \dots & \mathbf{y}_{1n} \\
        \vdots & \ddots & \vdots \\
        \mathbf{y}_{n1} & \dots & \mathbf{y}_{nn} \\
    \end{bmatrix}
$$
Esta matriz é simétrica e complexa no caso geral, podendo ser decomposta em:
$$
    [\mathbf{Y}] = [\mathbf{G}] + j[\mathbf{B}]
$$
onde $[\mathbf{G}]$ e $[\mathbf{B}]$ são a matriz de condutâncias nodais e matriz de suscetâncias nodais, respetivamente.

O elemento diagonal $\mathbf{y}_{ii}$ é dado pela soma das admitâncias de todos os ramos ligados ao nó $i$ (este valor é sempre diferente de zero), enquanto a elemento não diagonal $\mathbf{y}_{ij}$ $(i \neq j)$ é dado pelo simétrico da admitância do ramo que liga os nós $i$ e $j$ (o valor é nulo caso não estejam ligados). Note-se que esta é uma matriz esparsa, uma vez que numa rede elétrica cada nó só está ligado àqueles que lhe são vizinhos.

\vspace{0.5em}
\noindent Massajando a expressão da corrente injetada, obtemos:
$$
    \mathbf{S}^*_i = P_i - jQ_i = \mathbf{V}_i^* \sum_{j = 1}^{n} \mathbf{y}_{ij} \mathbf{V}_j,\quad i = 1,\dots,n
$$
Esta é a forma complexa das equações do trânsito de energia, em número igual ao de barramentos da rede. 

\noindent As equações do trânsito de energia escrevem-se na forma real e resultam então num conjunto de $2n$ equações que exprimem o balanço de energia ativa e reativa:
$$
    \begin{aligned}
        P_i = P_{G_i} - P_{C_i} = \sum_{j=1}^{n} V_i V_j\, \Big[G_{ij} \sin(\theta_i - \theta_j) - B_{ij} \cos(\theta_i - \theta_j)\Big], \quad &i = 1,\dots,n \\
        Q_i = Q_{G_i} - Q_{C_i} = \sum_{j=1}^{n} V_i V_j\, \Big[G_{ij} \sin(\theta_i - \theta_j) - B_{ij} \cos(\theta_i - \theta_j)\Big], \quad &i = 1,\dots,n
    \end{aligned} 
$$

%//==============================--@--==============================//%
\subsection{Solução do Trânsito de Energia}

%
\subsubsection{Cálculo das Tensões}

O primeiro passo é o cálculo das tensões nos barramentos. Dada a não-linearidade das equações do trânsito de energia, a solução tem de ser numérica, com base num método iterativo. Utiliza-se normalmente:

\begin{enumerate}
    \item Método de Gauss-Seidel.
    \item Método de Newton-Raphson.
    \item Método do Desacoplamento.
\end{enumerate}

\noindent Nos métodos computacionais, independentemente do escolhido, tudo começa com uma estimativa inicial das tensões nos barramentos. A partir desta, é feita uma correção para que se aproxime mais da solução final.

Após a primeira iteração, o processo é repetido até que as amplitudes das tensões atinjam a precisão desejada. Se o método for convergente, cada iteração aprimora a solução. No entanto, há situações em que o método pode não levar a uma solução.

%
\subsubsection{Potência Injetada no Nó de Balanço}

Uma vez obtida a convergência das tensões, é possível calcular a potência injectada no barramento de balanço. Para isso, utiliza-se a equação seguinte, que é excluída do processo iterativo de cálculo das tensões, uma vez que é especificada a tensão no nó de balanço:
$$
    P_1 - jQ_1 = \mathbf{V}_1^* \sum_{j=1}^{n} \mathbf{y}_{1j} V_j
$$
Se existirem múltiplos barramentos de balanço, então é preciso calcular a potência complexa injetada em cada um, utilizando as respetivas equações de trânsito de energia.

%
\subsubsection{Potência Transitada nas Linhas}
\label{subsubsec:potencia-transitada}

Finalmente, é necessário calcular os trânsitos de potência activa e reactiva nas linhas. A potência complexa \( \mathbf{S}^k_{ij} \) que transita na linha \( k \), ligada entre os nós \( i \) e \( j \), medida no nó \( i \) e definida como positiva na direcção \( i \rightarrow j \), é dada por:
$$
    \mathbf{S}^k_{ij} = P^k_{ij} + j Q^k_{ij} = \mathbf{V}_i (\mathbf{I}^k_{ij})^* = \left( \frac{1}{\mathbf{Z}^*_{L_k}} + \frac{\mathbf{Y}^*_{T_k}}{2} \right) V^2_i - \frac{1}{\mathbf{Z}^*_{L_k}} \mathbf{V}_i \mathbf{V}^*_j
$$
Analogamente, a potência complexa no extremo da linha ligado ao nó \( j \), definida como positiva na direcção \( j \rightarrow i \), é dada por:
$$
    \mathbf{S}^k_{ji} = P^k_{ji} + j Q^k_{ji} = \mathbf{V}_j (\mathbf{I}^k_{ji})^* = \left( \frac{1}{\mathbf{Z}^*_{L_k}} + \frac{\mathbf{Y}^*_{T_k}}{2} \right) V^2_j - \frac{1}{\mathbf{Z}^*_{L_k}} \mathbf{V}_i \mathbf{V}^*_j
$$
onde:
$$
    \begin{aligned}
        \mathbf{Z}_{L_k} &= R_k + jX_k  \\
        \mathbf{Y}_{T_k} &= j\omega C_k 
    \end{aligned}
$$
\( R_k \), \( X_k \) e \( C_k \) são, respetivamente, a resistência, a reactância e a capacitância \underline{totais} da linha.

\vspace{0.5em}
\noindent Definindo:
$$
    G_k = \text{Re} \left( \frac{1}{\mathbf{Z}_{L_k}} \right) = \frac{R_k}{R^2_k + X^2_k} 
    \qquad\;
    B_k = \text{Im} \left( \frac{1}{\mathbf{Z}_{L_k}} \right) = -\frac{X_k}{R^2_k + X^2_k} 
    \qquad\;
    B'_k = \text{Im} \left( \frac{\mathbf{Y}_{T_k}}{2} \right) = \frac{\omega C_k}{2}
$$
e somando $\mathbf{S}^k_{ij}$ e $\mathbf{S}^k_{ji}$, obtêm-se as perdas de potência ativa $P_L$ e reativa $Q_L$ na linha:
$$
    \mathbf{S}^k_{ij} + \mathbf{S}^k_{ji} = \underbrace{(P^k_{ij} + P^k_{ji})}_{P_L} + j\underbrace{(Q^k_{ij} + Q^k_{ji})}_{Q_L}
    \quad\implies\quad
    \left\{\begin{aligned}
        P_L &= G_k [V^2_i + V^2_j - 2 V_i V_j \cos (\theta_i - \theta_j)] \\
        Q_L &= -(B_k + B'_k) (V^2_i + V^2_j) + 2 B_k V_i V_j \cos (\theta_i - \theta_j)
    \end{aligned}\right.  
$$

%//==============================--@--==============================//%
\subsection{Modelo de Corrente Contínua}

Em estudos de planeamento, que requerem um elevado número de trânsitos de energia, usa-se por vezes um modelo simplificado do sistema de energia eléctrica, designado por \textit{modelo de corrente contínua} (\textit{D.C. Load Flow}), que dispensa o recurso a métodos iterativos. Trata-se de um modelo linearizado que permite calcular um valor aproximado dos trânsitos de potência activa na rede, baseando-se nas seguintes hipóteses simplificativas:

\begin{enumerate}
    \item Toma-se a amplitude de \underline{tensão igual a 1,0 p.u.} em todos os barramentos.
    \item Não se considera o trânsito de potência reactiva.
    \item Consideram-se nulas a resistência e a admitância transversal dos ramos.
    \item Admite-se que a diferença entre os argumentos das tensões nos barramentos de ligação de cada ramo é pequena, donde:
    $$
        \begin{aligned}
            \cos(\theta_i - \theta_j) &\approx 1 \\
            \sin(\theta_i - \theta_j) &\approx \theta_i - \theta_j
        \end{aligned}
    $$
\end{enumerate}

\noindent Uma vez que se desprezam a resistência e a admitância transversal dos ramos, a matriz das condutâncias nodais é nula, e os elementos da matriz das suscetâncias nodais são dados por:
$$
\begin{aligned}
    B_{ij} &= - \left( \frac{1}{X_{i1}} + \dots + \frac{1}{X_{in}} \right) & \quad \text{se } i = j \\
    B_{ij} &= \frac{1}{X_{ij}} & \quad \text{se } i \neq j
\end{aligned}
$$
Donde:
$$
    \sum_{j=1}^{n} B_{ij} = 0
$$
O número de nós da rede determina os elementos da matriz de susceptâncias. Os elementos da diagonal principal são iguais à soma das susceptâncias ligadas ao barramento. Note-se que a susceptância dos elementos indutivos é negativa; os elementos fora da diagonal principal são o simétrico da susceptância que liga os barramentos. É importante ressaltar que, no caso de elementos indutivos, esta susceptância é positiva.

\vspace{0.5em}\hrule\vspace{0.5em}

\noindent As equações do trânsito de energia escrevem-se então:
$$
\begin{aligned}
    P_i &= \sum_{j=1}^{n} V_i V_j [G_{ij} \cos(\theta_i - \theta_j) + B_{ij} \sin(\theta_i - \theta_j)] \\
    &\approx \sum_{j=1}^{n} B_{ij} (\theta_i - \theta_j) = \theta_i \sum_{j=1}^{n} B_{ij} - \sum_{j=1}^{n} B_{ij} \theta_j = - \sum_{j=1}^{n} B_{ij} \theta_j  \\
    Q_i &= \sum_{j=1}^{n} V_i V_j [G_{ij} \sin(\theta_i - \theta_j) + B_{ij} \cos(\theta_i - \theta_j)] \\
    &\approx \sum_{j=1}^{n} -B_{ij} = 0
\end{aligned}
$$
Em notação matricial:
$$
    [P] = [B'] [\theta] \;\iff\; [\theta] = [B']^{-1} [P]
$$
onde $[B']$ representa a matriz das suscetâncias nodais \underline{negada} (i.e., $[B'] = -[B]$), $[P]$ o vetor das potências injetadas e $[\theta]$ o vetor dos argumentos das tensões. Note-se que a matriz $[B]$ não inclui a linha e a coluna correspondentes ao nó de balanço (caso contrário seria uma matriz singular, i.e., não teria inversa).

\vspace{0.5em}\hrule\vspace{0.5em}

\noindent Nestes moldes, a \underline{potência ativa transitada} no ramo \( i - j \) é dada por:
$$
    \begin{aligned}
        P_{ij} &= B_{ij} V_i V_j \sin(\theta_i - \theta_j) = \frac{V_i V_j}{X_{ij}} \sin(\theta_i - \theta_j) \\
        &\approx B_{ij} (\theta_i - \theta_j) = \frac{1}{X_{ij}} (\theta_i - \theta_j)
    \end{aligned}
$$
Este modelo é análogo ao de uma rede em corrente contínua se se interpretar \( P \) como a corrente, \( \theta \) como a tensão e \( X \) como a resistência. Uma vez que não é considerada a resistência dos ramos, o modelo conduz a perdas nulas na rede. Podemos mitigar isto através de uma aproximação. As perdas no ramo \( i - j \) calculam-se fazendo \( V_i = V_j = 1,0 \) p.u., de onde vem (\hyperref[subsubsec:potencia-transitada]{vista a dedução acima}):
$$
    P_{Lij} = 2G_{ij} [1 - \cos(\theta_i - \theta_j)]
$$

%//==============================--@--==============================//%

    \clearpage
    \section{Correntes de Curto-Circuito}%
        %//==============================--@--==============================//%
\noindent Curto-circuitos em Sistemas de Energia Elétrica representam uma situação atípica originada por falhas, levando à formação de correntes elevadas. A maioria destes defeitos acontece em linhas aéreas devido à sua grande exposição a fenómenos físicos naturais. Estes podem ser:

\begin{itemize}
    \item \textbf{Trifásicos} quando afetam simultaneamente as três fases do sistema, sendo simétricos no caso de a impedância do defeito ser igual em todas as fases. Se esta impedância for nula, o curto circuito designa-se \textit{franco} (ou \textit{sólido}).

    \item \textbf{Assimétricos} onde podem envolver uma fase e a terra --- curto-circuito fase-terra ou monofásico —, que é o mais habitual, ou duas fases --- curto-circuito fase-fase ---, ou ainda duas fases e a terra --- curto-circuito fase-fase-terra.
\end{itemize}

\noindent Dada a sua perigosidade e potencial de dano a equipamentos, torna-se, por conseguinte, importante desligar no mais curto tempo possível a secção da rede onde se deu o defeito. Esta manobra exige a utilização de interruptores, ditos dijuntores capazes de cortar as correntes de curto-circuito.

%//==============================--@--==============================//%
\subsection{Esquema Monofásico Equivalente}

\noindent Um curto-circuito pode ser modelado pela ligação de uma impedância de baixo (ou nulo) valor no ponto de defeito. Considere-se um defeito trifásico simétrio no barramento \textit{i}, com uma impedância $\mathbf{Z}_{def}$ do qual resultam correntes de curto-circuito iguais em módulo nas três fases e desfasadas de $\pm 120$º. Já que a corrente de curto-circuito é simétrica, podemos usar o esquema monofásico equivalente que se representa na Fig. 34 (b):
\begin{figure}[H]
    \ctikzset{bipoles/length=1.2cm}
    \tikzset{
        line/.style = {line width=1pt, draw=black},
        shin/.style = {line, line width=2pt},
        net node/.style = {circle, draw=black,line width=1.2pt,minimum width=0.8cm, inner sep=0pt, outer 
        sep=0pt},
    }

    \centering
    \begin{subfigure}[b]{0.45\linewidth}
        \centering
        \begin{circuitikz}
        
            \draw[-] (1, 1.5) -- (7.5,1.5) node[yshift=1.7mm,xshift=-63mm] {$a$};
            \draw[-] (1, 1) -- (7.5,1) node[yshift=1.7mm,xshift=-63mm] {$b$};
            \draw[-] (1, 0.5) -- (7.5,0.5) node[yshift=1.7mm,xshift=-63mm] {$c$};
        
            \draw (1.5,0.5) to[generic, l=$\displaystyle \mathbf{Z}_{def}$] (1.5,-1.5);
            \draw (4.25,0.5) to[generic, l=$\displaystyle \mathbf{Z}_{def}$] (4.25,-1.5);
            \draw (7,0.5) to[generic, l=$\displaystyle \mathbf{Z}_{def}$] (7,-1.5);
        
            \draw[-stealth, thick] (1.5, 1.5) -- (1.5,0.1) node[yshift=2mm,xshift=4mm, font=\footnotesize] {$\mathbf{I}_{ai}^{cc}$};
            \draw[-stealth, thick] (4.25, 1) -- (4.25,0.1) node[yshift=2mm,xshift=4mm,font=\footnotesize] {$\mathbf{I}_{bi}^{cc}$};
            \draw[-stealth, thick] (7, 0.5) -- (7,0.1) node[yshift=2mm,xshift=4mm,font=\footnotesize] {$\mathbf{I}_{ci}^{cc}$};
            \draw[dashed] (1,-1.5) -- (7.5,-1.5);
        
        \end{circuitikz}
        \caption{Curto-circuito trifásico simétrico no barramento \textit{i}.}
    \end{subfigure}\hfill
    \begin{subfigure}[b]{0.45\linewidth}
        \begin{circuitikz}
        
            %% Esquema equivalente monofásico
            \path [shin] (1,0.5) -- (7.5,0.5) node[yshift=0mm,xshift=-67mm] {$i$};
            \path [shin] (4.25,1.5) -- (4.25,0.5);
            
            \draw[-] (1.5, 0.5) -- (1.5,-0.5) -- (1,-0.5);
            \draw (7,0.5) to[generic, l=$\displaystyle \mathbf{Z}_{def}$] (7,-1.5);
            \draw[dashed] (1,-1.5) -- (7.5,-1.5);
            
            \draw[-stealth] (4.25, 0.3) -- (4.25,-1.3) node[yshift=8mm,xshift=4mm] {$\mathbf{V}_i^{cc}$};
            \draw[-stealth] (7, 0.2) -- (7,0.1) node[yshift=1mm,xshift=4mm] {$\mathbf{I}_i^{cc}$};
            
            \end{circuitikz}
    \caption{Esquema monofásico equivalente para o defeito.}
    \end{subfigure}

    \caption{Curto-circuito trifásico simétrico}
    \label{fig:transito-dois-barramentos}
\end{figure}

\noindent Usando o \textbf{teorema da sobreposição} é possível considerar o estado da rede após o defeito como a \underline{sobreposição de dois estados}.

\begin{figure}[H]
    \ctikzset{bipoles/length=1.2cm}
    \tikzset{
        line/.style = {line width=1pt, draw=black},
        shin/.style = {line, line width=2pt},
        net node/.style = {circle, draw=black,line width=1.2pt,minimum width=0.8cm, inner sep=0pt, outer 
        sep=0pt},
    }

    \centering
    \begin{subfigure}[b]{0.45\linewidth}
        \centering
        \begin{circuitikz}
            
            %% Esquema equivalente monofásico
            \path [shin] (1,0.5) -- (7.5,0.5) node[yshift=0mm,xshift=-67mm] {$i$};
            \path [shin] (4.25,1.5) -- (4.25,0.5);
        
            \draw[-] (1.5, 0.5) -- (1.5,-1) -- (1,-1);
            \draw[dashed] (7,0.1) to[sinusoidal voltage source, sources/symbol/rotate=auto] (7,-1)  node[yshift=5mm,xshift=7.5mm] {$\mathbf{V}_i^{0}$};
            \node[yshift=0mm,xshift=65mm] (7, 0.5) {$+$};
            \node[yshift=-9mm,xshift=65mm] (7, 0.5) {$-$};
            \draw[dashed] (7,-1) to[generic, l=$\displaystyle \;\;\,\mathbf{Z}_{def}$] (7,-2.5);
            \draw[dashed] (1,-2.5) -- (7.5,-2.5);
        
            \draw[-stealth] (4.25, 0.3) -- (4.25,-2.3) node[yshift=13mm,xshift=4mm] {$\mathbf{V}_i^{0}$};
            \draw[dashed, -stealth] (7, 0.5) -- (7,0.1) node[yshift=1mm,xshift=9mm] {$\mathbf{I} = 0$};
        
        \end{circuitikz}
        \caption{Estado 1.}
    \end{subfigure}\hfill
    \begin{subfigure}[b]{0.45\linewidth}
      \begin{circuitikz}
        
        %% Esquema equivalente monofásico
        \path [shin] (1,0.5) -- (7.5,0.5) node[yshift=0mm,xshift=-67mm] {$i$};
        \path [shin] (4.25,1.5) -- (4.25,0.5);
    
        \draw[-] (1.5, 0.5) -- (1.5,-1) -- (1,-1);
        \draw (7,0.1) to[sinusoidal voltage source, sources/symbol/rotate=auto] (7,-1)  node[yshift=5mm,xshift=7.5mm] {$\mathbf{V}_i^{0}$};
        \node[yshift=0mm,xshift=65mm] (7, 0.5) {$-$};
        \node[yshift=-9mm,xshift=65mm] (7, 0.5) {$+$};
        \draw  (7,-1) to[generic, l=$\displaystyle \;\;\,\mathbf{Z}_{def}$] (7,-2.5);
        \draw[dashed] (1,-2.5) -- (7.5,-2.5);
    
        \draw[-stealth] (4.25, 0.3) -- (4.25,-2.3) node[yshift=13mm,xshift=4mm] {$\mathbf{V}_i^{T}$};
        \draw[-stealth] (7, 0.5) -- (7,0.1) node[yshift=1mm,xshift=7.3mm] {$\mathbf{I}_{ci}^{cc}$};
    
    \end{circuitikz}
    \caption{Estado 2.}
    \end{subfigure}

    \caption{Teorema da sobreposição sob os dois estados do barramento}
    \label{fig:transito-dois-barramentos2}
\end{figure}

\begin{enumerate}
    \item O \textbf{estado 1} corresponde à situação pré-defeito. Uma vez que a f.e.m do gerador fictício é igual à tensão no barramento, a corrente é nula, pelo que pode ser retirado.

    \item O \textbf{estado 2} corresponde à ligação do gerador fictício com polaridade invertida. Os geradores reais são representados unicamente pelas respetivas impedâncias internas.
\end{enumerate}
%//==============================--@--==============================//%
\subsection{Teorema de Thévenin}

O \textbf{estado 2} corresponde à aplicação do teorema de Thévenin, o qual permite estabelecer para uma rede elétrica, vista de qualquer nó $i$, o esquema equivalente representado na Fig. 36, onde $Z_T$ é a impedância equivalente (de Thévenin) da rede vista do nó $i$ quando se anulam as fontes de tensão e/ou de corrente e $V_i^0$ é a tensão pré-defeito.

\vspace{-1em}
\begin{minipage}[b]{.45\linewidth}
\begin{figure}[H]
    \begin{circuitikz}[scale=1]
        \draw (1,0) to[sinusoidal voltage source, sources/symbol/rotate=auto] (7.5,0)  node[yshift=7mm,xshift=-32mm] {$\mathbf{V}_i^{0}$};
        \node[circ] at (2.5, 0) {};
        \node[] at (2.5, 0.2) {$i$};
        \draw  (1,0) to[generic, l=$\displaystyle \mathbf{Z}_{T}$] (1,-4);
        \draw  (7.5,0) to[generic, l=$\displaystyle \mathbf{Z}_{def}$] (7.5,-4);
        \draw[-stealth] (7.5, 0) -- (7.5,-0.8) node[yshift=1mm,xshift=4mm] {$\mathbf{I}_{ci}^{cc}$};
        \draw (1,-4) -- (7.5,-4);
        
        \node[yshift=0mm,xshift=-5mm] at (4.25, 0.5) {$-$};
        \node[yshift=0mm,xshift=5mm] at (4.25, 0.5) {$+$};
    \end{circuitikz}
    \caption{Esquema Equivalente de Thévenin.}
\end{figure}
\end{minipage}\hfill
\begin{minipage}[b]{.45\linewidth}
 \begin{figure}[H]
    Com defeito no nó $i$ com impedância $\mathbf{Z}_{def}$, $\mathbf{I}_{cc}$ é:
    $$
        \mathbf{I}_{cc} = \frac{\mathbf{V}_i^0}{\mathbf{Z}_{def} + \mathbf{Z}_T}
    $$
    Para um sistema trifásico (em unidades do S.I.) será:
    $$
       \mathbf{ I}_{cc} = \frac{\mathbf{V}_i^0}{\sqrt{3}(\mathbf{Z}_{def} + \mathbf{Z}_T)}\;\rightarrow\;
        \boxed{I_{cc} = \frac{\mathbf{V}_i^0}{\sqrt{3}\mathbf{Z}_T}}
    $$
    Sendo $\mathbf{Z}_{def}$ nula (curto-circuito franco):
\end{figure}
\end{minipage}

\vspace{1em}
\noindent Define-se a potência de curto-circuito $S_{cc}$ no nó $i$ por:
$$
    S_i^{cc} = \sqrt{3} V_i^0 I_i^{cc} = \frac{{V_i^0}^2}{Z_T}
$$
Se se tomar para $V_i^0$ a tensão nominal $V_n$:
$$
    S_i^{cc} = \frac{V^2_n}{Z_T}
$$
Em valores p.u.:
$$
    \boxed{S_i^{cc} = I_i^{cc} = \frac{1}{Z_T}}
$$
isto é, \underline{a corrente e a potência de curto-circuito são iguais ao inverso da impedância equivalente da rede} vista do ponto de defeito.
%//==============================--@--==============================//%
\subsection{Curto-Circuito de um Gerador Síncrono}

Considere um gerador síncrono que opera à velocidade nominal, sem carga, excitado por uma corrente constante. Se um curto-circuito trifásico ocorrer no instante $t=0$, a corrente é dada por:
$$
    i_{cc} = \sqrt{2} \frac{E}{X_d'} \cos(\omega t + \alpha_0) - \frac{E}{\sqrt{2}}  \left( \frac{1}{X_d'} + \frac{1}{X_q} \right) \cos \alpha_0 - \frac{E}{\sqrt{2}}  \left( \frac{1}{X_d'} - \frac{1}{X_q} \right) \cos(2\omega t + \alpha_0)
$$
em que as grandezas representam: $X_d'$: reactância transitória ao longo do eixo $d$; $X_q$: reactância síncrona ao longo do eixo $q$; $E$: f.e.m. da máquina; $\omega$: frequência angular nominal; $\alpha_0$: ângulo do rotor no instante do curto-circuito.

\vspace{0.5em}
\noindent Verifica-se portanto que:
\begin{itemize}
    \item A corrente de curto-circuito tem três componentes: uma componente fundamental, uma componente contínua e uma com dupla frequência.
    \item A componente contínua varia com a posição do rotor no momento do defeito.
    \item Desconsiderando as resistências dos enrolamentos, as componentes da corrente de curto-circuito são constantes.
    \item O valor eficaz da componente fundamental é $E/X_d'$. Em regime estacionário, o curto-circuito é maior que a corrente estacionária.
\end{itemize}

\noindent Após a ocorrência de um curto-circuito, o fluxo magnético no gerador não desaparece de imediato, ao contrário do que poderíamos intuitivamente pensar. Uma das principais razões para esta persistência do fluxo magnético é a presença de barras de cobre localizadas na superfície do rotor. Estas barras desempenham um papel crucial na manutenção de um fluxo magnético residual.

\noindent Em situações normais, quando o gerador opera em regime estacionário, o enrolamento no rotor não é percorrido por corrente. No entanto, quando ocorre um desequilíbrio, como no caso de um curto-circuito, a máquina começa a experimentar oscilações que são resultantes de desequilíbrios de potência. Estas oscilações podem ser atribuídas, em parte, à indução de correntes nestes enrolamentos devido às flutuações no fluxo magnético. Estas correntes induzidas atuam como um amortecedor, ajudando a estabilizar a máquina.

Adicionalmente, é imperativo mencionar a reatância subtransitória $X_d^{\prime\prime}$. Durante um curto-circuito, o comportamento do gerador é amplamente influenciado por esta reatância, particularmente nos primeiros ciclos do evento. A reatância $X_d^{\prime\prime}$ reflete a resposta rápida do gerador ao distúrbio, determinada em grande parte pelas correntes no enrolamento amortecedor. Estas correntes, em combinação com a reatância subtransitória, determinam a magnitude inicial e a dinâmica da corrente durante o curto-circuito como apresentado:

\begin{figure}[H]
    \centering
    \begin{tikzpicture}
        \begin{axis}[
            axis lines=middle,
            xlabel={$t$},
            ylabel={$i_{cc}$},
            xmin=0, xmax=3, xtick=\empty,
            ymin=-4, ymax=4, ytick={2.8}, yticklabel={$I_{max}$},
            axis line style={thick},
            grid=both, minor tick num=1,
            width=0.7\textwidth,
            height=0.5\textwidth,
            samples=500,
            clip=false,
        ]

        % Parameters (change these to fit your data)
        \def\alpha{0.5}
        \def\omega{30}
        \def\A{1}
        \def\B{1}
        \def\phi{-90}

        % Damped sinusoidal waveform with decaying DC component
        \addplot[black, thick, domain=0:2.75] { exp(-\alpha*x) * 2*\A*cos(deg(\omega*x) + exp(-7.5*\alpha*x) * \phi) + exp(-7.5*\alpha*x) * \B };

        \addplot[dashed, domain=0:2.75] { 2*exp(-\alpha*x) + exp(-7.5*\alpha*x) * \B };
        \addplot[dashed, domain=0:2.75] { 2*exp(-\alpha*x) };
        \addplot[dashed, domain=0:2.75] { -2*exp(-\alpha*x) - exp(-7.5*\alpha*x) * \B };
        \addplot[dashed, blue!, thick, domain=0:1] {ln(x+2) * exp(-8.88*\alpha*x) * \B * 1.5 * 1/ln(2) * (-x*x+1)};
            
        % Vertical arrow denoting the value 1.5, next to the y-axis
        \draw[>=stealth, <->, thick] (-0.05,0) -- (-0.05,1.95) node[midway,left] {$\sqrt{2} \displaystyle\frac{E}{X_d''}$};

        \draw[>=stealth, ->, blue!90] (1,1.5) -- (0.75,0.25);
        \node[blue!95, above, xshift=1mm, align=center, font=\footnotesize] at (1,1.5) {Componente\\contínua};
        
        \end{axis}
    \end{tikzpicture}
\end{figure}

\noindent A corrente que circula no estator durante o curto-circuito também é afetada. De fato, nas primeiras alternâncias após o início do curto-circuito, observa-se um aumento substancial na corrente no estator. Uma das principais contribuições para este aumento é a componente contínua da corrente. A sua presença resulta num acréscimo significativo no valor de pico das primeiras alternâncias da corrente de curto-circuito. Em termos numéricos, este aumento pode chegar a ser até $1.8\sqrt{2} = 2.55$ vezes o valor eficaz da componente alternada. É crucial entender a interação entre esta componente contínua e a corrente alternada, uma vez que determina a resposta dinâmica da máquina em situações de falha e influencia as medidas de proteção a serem adotadas. 

O efeito desmagnetizante desta corrente, que tende a enfraquecer o fluxo, é compensado por um aumento da corrente do enrolamento de excitação, que tem um efeito magnetizante. Dado que este enrolamento tem uma resistência não nula, esta corrente vai diminuindo com uma constante de tempo \(T_d'\) (aproximadamente \(T_d' \approx T_{d0}' \cdot X_d'/X_d''\)), onde \(T_{d0}'\) é a constante de tempo do enrolamento de excitação (da ordem de 5-10 s). Isto origina um enfraquecimento do fluxo no entreferro e, portanto, da tensão do gerador. A corrente no estator vai, consequentemente, diminuindo também até atingir o seu valor em regime estacionário com uma constante de tempo \(T_d'\) (cerca de 1-2 s).

\begin{figure}[H]
    \begin{subfigure}[b]{.475\linewidth}
        \centering
        \begin{tikzpicture}
            \begin{axis}[
                axis lines=middle,
                xlabel={$t$},
                ylabel={$\phi$},
                xmin=0, xmax=5, xtick=\empty,
                ymin=0, ymax=5, ytick=\empty,
                axis line style={thick},
                grid=both, minor tick num=1,
                width=\textwidth,
                samples=500,
            ]

            \addplot[black, thick] {1.75*exp(-0.5*x)+1.05};
            
            \end{axis}
        \end{tikzpicture}
    \end{subfigure}\hfill
    \begin{subfigure}[b]{.475\linewidth}
        \centering
        \begin{tikzpicture}
            \begin{axis}[
                axis lines=middle,
                xlabel={$t$},
                ylabel={$i_r$},
                xmin=0, xmax=5, xtick=\empty,
                ymin=0, ymax=2.75, ytick=\empty,
                axis line style={thick},
                grid=both, minor tick num=1,
                width=\textwidth,
                samples=500,
            ]

            \addplot[black, thick, domain=0:5] {2.25*(1-exp(-5*x)) * exp(-0.75*x) * (ln(x+1.25)) + 0.5};
            \addplot[dashed, thick, domain=0:5] {0.5};
                
            \end{axis}
        \end{tikzpicture}
    \end{subfigure}
\end{figure}

%//==============================--@--==============================//%
\newpage
\subsection{Modelos dos Elementos da Rede}
\subsubsection{Gerador}

Tendo em conta o discutido na secção precedente, o modelo da máquina síncrona para o cálculo de correntes de curto-circuito simétrico é ilustrado na figura seguinte:

\begin{figure}[H]
    \centering
    \scalebox{0.75}{%
        \begin{circuitikz}
            %% Configuração do circuitikz
            \ctikzset{inductor=cute}    
            \ctikzset{inductors/coils=4}
            \ctikzset{bipoles/resistor/height=0.25}
            \ctikzset{bipoles/resistor/width=0.5}
            \ctikzset{resistors/zigs=4}
        
            \draw (0,-2) to[sV, l=$\mathbf{E}^{\prime}$] (0,2) 
            to[L, l=$jX_d^{\prime}$ (ou $jX_d^{\prime\prime}$), -*] (4,2);
            \draw (0,-2) to[short,-*] (4,-2);
        \end{circuitikz}
    }
    \caption{Modelo do Gerador Síncrono}
    \label{fig:modelo-gerador-sincrono}
\end{figure}

\noindent No contexto deste modelo, salientamos o seguinte:
\begin{enumerate}
    \item Desconsiderou-se a resistência dos enrolamentos.
    \item Desprezaram-se todas as componentes da corrente de curto-circuito, à excepção da componente à frequência fundamental.
    \item Ainda que a componente à frequência fundamental decresça exponencialmente, tendo em conta que a constante de tempo é da ordem do segundo (50 ciclos), o regime é considerado quase-estacionário.
    \item Em disjuntores de actuação rápida, tipicamente empregues na rede de transporte (1,5 a 2 ciclos), deve-se optar pela reactância subtransitória, resultando num valor mais acentuado da corrente de curto-circuito. Para disjuntores de actuação mais lenta (4 a 5 ciclos), comuns na distribuição, a reactância transitória é suficiente.
    \item No cálculo das forças electrodinâmicas induzidas pela corrente de curto-circuito, recorre-se à reactância subtransitória, visto ser crucial determinar o seu valor máximo.
\end{enumerate}

\subsubsection{Transformador e Linha}

O modelo do transformador é congruente com o usado no trânsito de energia. Omite-se o ramo transversal associado à impedância de magnetização e mantém-se o ramo longitudinal com a impedância de curto-circuito (frequentemente desconsiderando a resistência). Ao considerar a rede em vazio e no estado pré-falha, assume-se uma relação de transformação unitária, mesmo se o transformador tiver um comutador de tomadas.

A representação da linha coincide igualmente com a adotada no trânsito de energia, isto é, o esquema equivalente em $\pi$. É importante notar que a admitância transversal tem um impacto diminuto, pelo que é viável desprezá-la sem incorrer em erros significativos. Relativamente à resistência, esta pode ser omitida em linhas de alta tensão, mas não em linhas de média ou baixa tensão.

\subsubsection{Cargas}

Ao calcular a corrente de curto-circuito, frequentemente desconsideram-se as cargas, que influenciam minimamente o valor da referida corrente. Esta suposição implica que a rede esteja em vazio, com um perfil de tensão homogéneo, omitindo simultaneamente todos os elementos transversais (como capacitâncias das linhas e baterias de condensadores, bem como reactâncias indutivas).

Quando se modelam as cargas, estas são geralmente vistas como passivas (i.e., com elasticidade 2), o que facilita a sua representação através de impedâncias constantes. Evidentemente, uma carga passiva não influencia a corrente de curto-circuito.

Destaca-se que as impedâncias equivalentes das cargas apresentam valores substanciais quando contrastadas com as impedâncias dos elementos da rede, manifestando uma predominância resistiva, ao contrário das últimas, que demonstram um carácter reactivo dominante.

Em situações específicas, como instalações industriais dotadas de motores (síncronos ou assíncronos) de alta potência, é imperativo modelizar estes de forma mais meticulosa, o que implica adotar um modelo análogo ao da máquina síncrona (f.e.m. em série com a reactância transitória). De facto, nos momentos pós-falha, os motores operam como geradores, capitalizando a energia cinética armazenada nas suas massas rotativas, contribuindo assim para a corrente de curto-circuito.

%//==============================--@--==============================//%
\subsection{Matriz das Impedâncias Nodais}

Para calcular correntes de curto-circuito em redes extensas, é necessário uma formulação adequada para solução digital. Essa abordagem utiliza a matriz de \textit{impedâncias nodais} $[\mathbf{Z}]$, que é o inverso da matriz das \textit{admitâncias nodais} $[\mathbf{Y}]$. Em valores p.u.:
$$
    [\mathbf{Z}] = [\mathbf{Y}]^{-1}
$$
No trânsito de energia, tanto os geradores como as cargas são modelados como fontes e drenos de potência constantes. No entanto, para o cálculo de correntes de curto-circuito, o gerador é representado por uma fonte de corrente $\mathbf{I}_i$ em paralelo com a sua admitância transitória $\mathbf{Y}_{di}'$ (ou subtransitória $\mathbf{Y}_{di}''$). A carga, tratada como passiva (elasticidade 2), é representada pela admitância equivalente $\mathbf{Y}_{Ci}$, como ilustrado na figura:

\begin{figure}[h]
    \centering
    \tikzset{
        line/.style = {line width=1pt, draw=black},
        shin/.style = {line, line width=2pt},
        net node/.style = {circle, draw=black,line width=1.2pt,minimum width=0.8cm, inner sep=0pt, outer sep=0pt},
    }
        
    \begin{circuitikz}[scale=0.93]
        %% Configure circuitikz
        \ctikzset{inductor=cute}    
        \ctikzset{inductors/coils=4}
        \ctikzset{bipoles/resistor/height=0.25}
        \ctikzset{bipoles/resistor/width=0.5}
        \ctikzset{bipoles/length=1.2cm}
        \ctikzset{resistors/zigs=4}
        
        %% Esquema equivalente monofásico
        \path [shin] (1,0.5) -- (2.5,0.5) node[right] {$i$};
        \path [shin] (6.5,0.5) -- (8,0.5) node[right] {$j$};
        
        % Gerador ficticio
        \draw (1.75,2.2) to[esource] (1.75,1.5) node[yshift=3mm,xshift=6mm] {$\,\mathbf{I}_i$};
        \draw [-](1.75,2.2) to[short] (1.75,2.5);
        \draw [thick, >=stealth,->](1.75,2) to[short] (1.75,1.6);
        \draw [thick, >=stealth,->](1.75,1.5) to[short] (1.75,0.5);
        
        % Admitância transitória
        \draw (1.2,1.5) to[L, l=$\mathbf{Y}_{di}'$] (1.2,2.2);
        \draw [-](1.2,0.5) to[short] (1.2,1.5);
        \draw [-](1.2,2.2) to[short] (1.2,2.5);
    
        \draw[dashed] (1,2.5) -- (8,2.5);
    
        % Impedâncias
        \draw (2,0) 
            to[short, -] (2,0)
            to[generic, l=$\mathbf{Z}_L$] (7,0)
            to[short, -] (7,0);
    
        \draw[-] (2,0.5) -- (2,0);
        \draw[-] (7,0.5) -- (7,0);
        \draw (2.5,0) to[generic, l=$\displaystyle \frac{\mathbf{Y}_{TK}}{2}$] (2.5,-3);
        \draw (6.5,0) to[generic, l_=$\displaystyle \frac{\mathbf{Y}_{TK}}{2}$] (6.5,-3);
        \draw (1.475,0) to[generic, l_=$\displaystyle \mathbf{Y}_{Ci}$] (1.475,-3);
        \draw [-](1.475,0.5) to[short] (1.475,0);
    
        \draw[dashed] (1,-3) -- (8, -3);
    
        % Setas tensão
        \draw [thick, >=stealth,->](2,-0.5) to[short] (2,-2.5) node[below, font=\footnotesize] {$\mathbf{V}_i$};
        \draw [thick, >=stealth,->](7,-0.5) to[short] (7,-2.5) node[below, font=\footnotesize] {$\mathbf{V}_j$};
        
    \end{circuitikz}
    \caption{Barramento i com geração, carga e linha ligada ao barramento j.}
\end{figure}

\noindent A matriz de admitâncias utilizada para o trânsito de energia deve ser ajustada nos seus elementos diagonais:

\begin{itemize}
    \item Adição da admitância transitória de cada gerador ao elemento correspondente ao seu nó de ligação.
    \item Integração das admitâncias equivalentes das cargas nos respetivos nós a que estão associadas.
\end{itemize}

\noindent Através da aplicação do teorema da sobreposição, o vetor de tensões nodais após o curto-circuito $[\mathbf{V}^{cc}]$ é dado pela soma do vetor das tensões preexistentes $[\mathbf{V}^0]$ com o vetor das variações de tensão $[\mathbf{V}^T]$ resultantes da ligação do gerador equivalente de Thévenin no nó $i$, onde ocorre o defeito (não se consideram em simultâneo):
$$
    [\mathbf{V}^{cc}] = [\mathbf{V}^0] + [\mathbf{V}^T]
$$
O vetor $[\mathbf{V}^T]$ pode ser obtido a partir da equação:
$$
    [\mathbf{I}^{cc}] = [\mathbf{Y}] [\mathbf{V}^T]
$$
ou:
$$
    [\mathbf{V}^T] = [\mathbf{Z}][\mathbf{I}^{cc}]
$$
A matriz de impedâncias nodais $[\mathbf{Z}]$ é simétrica, mas é consideravelmente menos esparsa do que a matriz $[\mathbf{Y}]$, visto que a sua inversão afeta negativamente a esparsidade.

O vetor $[\mathbf{I}^{cc}]$ representa as correntes de curto-circuito injetadas, e todos os seus elementos são nulos, exceto o que corresponde ao nó de defeito $i$:
$$
    [\mathbf{I}^{cc}] = 
    \begin{bmatrix}
        0 \\
        \vdots \\
        -I^{cc}_i \\
        \vdots \\
        0
    \end{bmatrix}
$$
Observa-se o sinal negativo da corrente injetada, que resulta da corrente de curto-circuito ter o sentido convencional de uma corrente de carga.

\vspace{0.5em}
\noindent Substituindo, obtém-se:
$$
    [\mathbf{V}^{cc}] = [\mathbf{V}^0] + [\mathbf{Z}] [\mathbf{I}^{cc}]
    \quad\implies\quad
    \left\{\begin{aligned}
        \mathbf{V}^{cc}_1 &= \mathbf{V}^0_1 - z_{1i} \mathbf{I}^{cc}_i \\
        &\qquad\vdots \\
        \mathbf{V}^{cc}_i &= \mathbf{V}^0_i - z_{ii} \mathbf{I}^{cc}_i \\
        &\qquad\vdots \\
        \mathbf{V}^{cc}_n &= \mathbf{V}^0_n - z_{ni} \mathbf{I}^{cc}_i \\
    \end{aligned}\right.
$$
%//==============================--@--==============================//%
\noindent Neste momento, a corrente de curto-circuito $I^{cc}_i$ é desconhecida. No entanto, podemos relacioná-la com a tensão $\mathbf{V}^{cc}_i$ através da equação:
$$
\mathbf{V}^{cc}_i = \mathbf{Z}_{def} \mathbf{I}_i^{cc}
$$
onde $\mathbf{Z}_{def}$ é a impedância do defeito.

\noindent Determinando o valor da corrente de curto-circuito através da conjugação das duas equações anteriores:
$$
\mathbf{I}_i^{cc} = \frac{\mathbf{V}^0_i}{z_{ii} + \mathbf{Z}_{def}}
$$
Para um curto-circuito franco ($\mathbf{Z}_{def} = 0$) e $\mathbf{V}^{cc}_i = 0$, a equação anterior reduz-se a:
$$
\mathbf{I}_i^{cc} = \frac{\mathbf{V}^0_i}{z_{ii}}
$$
Onde $z_{ii}$, o elemento diagonal da matriz de impedâncias nodais correspondente ao barramento $i$, coincide com a impedância equivalente de Thévenin da rede vista desse barramento.

Tendo determinada a corrente de curto-circuito no barramento $i$, as tensões nos outros barramentos são obtidas a partir das equações resultantes da soma de matrizes já acima mencionadas:
$$
\mathbf{V}^{cc}_j = \mathbf{V}^0_j - \frac{z_{ji} \mathbf{V}^0_i}{z_{ii}}
$$
Uma vez conhecidas as tensões nos barramentos, podem calcular-se as correntes nos ramos da rede. Em geral, interessa sobretudo as correntes que circulam nos ramos que convergem no nó do defeito $i$. No caso de um curto-circuito franco, a corrente no ramo $k$, ligado entre os nós $i$ e $j$, e considerada positiva no sentido $j \to i$, é dada por:
$$
    \mathbf{I}^{cc}_{ji} = \frac{\mathbf{V}^{cc}_j - \mathbf{V}^{cc}_i}{\mathbf{Z}_{Lk}} = \frac{1}{\mathbf{Z}_{Lk}} \left( \mathbf{V}^0_j - \frac{z_{ji} \mathbf{V}^0_i}{z_{ii}} \right)
$$
Note-se, finalmente, que para o cálculo do curto-circuito no barramento $i$ é necessário conhecer apenas os elementos da coluna $[\mathbf{Z}_i] = \left[ z_{1i}, z_{2i}, \dots, z_{ni} \right]$ da matriz de impedâncias nodais, a qual pode ser obtida sem a necessidade de inversão completa da matriz $[\mathbf{Y}]$, operação que frequentemente é complexa para redes de grande dimensão. As diferentes colunas podem ser calculadas uma a uma, à medida que se percorre sequencialmente os barramentos da rede, nos quais se pretende calcular a corrente de curto-circuito.

\begin{figure}[H]
    \centering
    \scalebox{1.0}{%
        \begin{circuitikz}
            %% Esquema equivalente em pi
            \draw (0,0) 
                to[short, *-] (2,0)
                to[generic, l=$\mathbf{Z}_{L_k}$] (7,0)
                to[short, -*] (9,0);

            \draw (2,0) to[generic, l=$\displaystyle \frac{\mathbf{Y}_{T_k}}{2}$] (2,-3);
            \draw (7,0) to[generic, l_=$\displaystyle \frac{\mathbf{Y}_{T_k}}{2}$] (7,-3);

            \draw[dashed] (0,-3) -- (9,-3);

            %% Labels e arrows
            \draw[>=stealth,->] (0,-0.25) -- (0,-2.75) node[midway,left] {$\mathbf{V}^{cc}_i$};
            \draw[>=stealth,-] (9,0) -- (9,-3) node[midway,right] {};

            \draw[>=stealth,->] (7.9,0) -- (8,0) node[midway,above] {$\mathbf{I}^{cc}_{ji}$};

            % Label nodes
            \node[above] at (0,0) {$j$};
            \node[above] at (9,0) {$i$};
        \end{circuitikz}
    }
    \caption{Barramento $i$ com geração, carga e linha ligada ao barramento $j$.}
\end{figure}

%//==============================--@--==============================//%


    \clearpage \pagestyle{empty} \setcounter{secnumdepth}{-2}
    \bibliographystyle{unsrtnat} \nocite{*}
    \bibliography{refs}
\end{document}%